%% LyX 2.1.2.1 created this file.  For more info, see http://www.lyx.org/.
%% Do not edit unless you really know what you are doing.
\documentclass[11pt,english]{article}
\usepackage[T1]{fontenc}
\usepackage[latin9]{inputenc}
\usepackage{geometry}
\geometry{verbose,tmargin=1.5cm,bmargin=1.5cm,lmargin=1.5cm,rmargin=1.5cm}
\usepackage{color}
\usepackage{amsmath}
\usepackage{amssymb}
\usepackage{setspace}
\onehalfspacing

\makeatletter
%%%%%%%%%%%%%%%%%%%%%%%%%%%%%% User specified LaTeX commands.
%\documentclass[opre,blindrev]{informs3} % current default for manuscript submission

% current default line spacing
%%\OneAndAHalfSpacedXII 
%%\DoubleSpacedXII
%%\DoubleSpacedXI

% If hyperref is used, dvi-to-ps driver of choice must be declared as
%   an additional option to the \documentclass. For example
%\documentclass[dvips,opre]{informs3}      % if dvips is used 
%\documentclass[dvipsone,opre]{informs3}   % if dvipsone is used, etc. 

%%% OPRE uses endnotes
\usepackage{endnotes}
\let\footnote=\endnote
\let\enotesize=\normalsize
\def\notesname{Endnotes}%
\def\makeenmark{\hbox to1.275em{\theenmark.\enskip\hss}}
\def\enoteformat{\rightskip0pt\leftskip0pt\parindent=1.275em
  \leavevmode\llap{\makeenmark}}

\date{}

% Private macros here (check that there is no clash with the style)
\newcommand{\lambdahat}{\widehat{\lambda}}
\newcommand{\xtilde}{\tilde{x}}
\newcommand{\qhat}{\widehat{q}}
\newcommand{\as}{\ a.s.}
\newcommand{\zap}[1]{}
\newcommand{\Ncal}{\mathcal{N}}
\newcommand{\sigmahat}{\hat{\sigma}}
\newcommand{\Nat}{\mathbb{N}}
\renewcommand{\P}{\mathbb{P}}
\newcommand{\Pb}[2]{\mathbb{P}_{#1}\left\{#2\right\}}
\newcommand{\Ybar}{\overline{Y}}
\newcommand{\Z}{\mathbb{Z}}
\newcommand{\Q}{\mathbb{Q}}
\newcommand{\R}{\mathbb{R}}
% \newcommand{\argmax}{\operatornamewithlimits{arg\,max}}
% \newcommand{\argmin}{\operatornamewithlimits{arg\,min}}
\newcommand{\g}{\,\vert\,}
\newcommand{\ind}[1]{1_{\left\{#1\right\}}}
\newcommand{\Fcal}{\mathcal{F}}
\newcommand{\Gcal}{\mathcal{G}}
\newcommand{\E}{\mathbb{E}}
\newcommand{\xhat}{\hat{x}}
\newcommand{\xwig}{\tilde{x}}
\newcommand{\PZ}{\mbox{PZ}}
\newcommand{\PCS}{\mbox{PCS}}
\newcommand{\PGS}{\mbox{PGS}}
\newcommand{\uLFC}{\underline{u}}
\newcommand{\e}[1]{\left\{ #1 \right\}}
\newcommand{\T}{\mathbb{T}} % Time index set.
\newcommand{\CS}{\mbox{CS}}
\newcommand{\Ft}{\mathcal{F}_t}
\newcommand{\F}[1]{\mathcal{F}_{#1}}
\newcommand{\Ftau}{\mathcal{F}_\tau}
\newcommand{\Rcal}{\mathcal{R}}
\newcommand{\upthresh}{P}
\newcommand{\xstar}{X^*}
\newcommand{\cmax}{1-(P^*)^{\frac1{k-1}}}
\newcommand{\NewN}{M}
\newcommand{\thetavec}{\vec{\theta}}
\newcommand{\uvec}{\vec{u}}
\newcommand{\sigmavec}{\lambda}
\newcommand{\sigmacom}{\sigma}
\newcommand{\sigmascal}{\lambda}
\newcommand{\PstarB}{\mathcal{P}^*_B}
\newcommand{\Ymod}{Y'}
\newcommand{\ceil}{\mathrm{ceil}}
% How to state the common variance assumption at the beginning of lemmas, propositions and theorems
\newcommand{\homog}{Suppose $\lambda^2_x = \sigma^2>0\ \forall x$.  }
\newcommand{\algref}[1]{Alg.~\ref{#1}}


\usepackage{algorithm,algorithmic,enumerate}




%% Setup of theorem styles. Outcomment only one. 
%% Preferred default is the first option.
  % Preferred (Theorem 1, Lemma 1, Theorem 2)
%\TheoremsNumberedByChapter  % (Theorem 1.1, Lema 1.1, Theorem 1.2)

\makeatother

\usepackage{babel}
\begin{document}

\title{Asymptotic Validity of a Fully Sequential Elimination Procedure for
Indifference-Zone Ranking and Selection with Tight Bounds on Probability
of Correct Selection}


\author{Saul Toscano\\
{\small{}School of Operations Research and Information Engineering,
Cornell University}\\
Peter I. Frazier\\
{\small{}School of Operations Research and Information Engineering,
Cornell University}}

\maketitle
We consider the indifference-zone (IZ) formulation of the ranking
and selection problem. Conservatism leads classical IZ procedures
to take too many samples in problems with many alternatives. The Bayes-inspired
Indifference Zone (BIZ) procedure, proposed in Frazier (2014), is
less conservative than previous procedures, but its proof of validity
requires strong assumptions. In this paper, we present a new proof
of asymptotic validity that relaxes these assumptions. 


\section{Introduction}

One common problem in simulation is that of choosing the best among
several simulated systems. The problem of deciding how many samples
to use from each alternative to support our selection as the best
is the ranking and selection problem. An efficient solution to this
problem has to balance between the time spent simulating and the quality
of the selection.

This paper will consider the indifference-zone (IZ) formulation of
the ranking and selection problem, in which we choose the best system
with probability larger than a pre-specified threshold, whenever the
distance between the best system and the others is sufficiently large.
We say that a sampling procedure having this property satisfies the
IZ guarantee and the set of system configurations under which the
best alternative is better than the second best by at least some given
$\delta>0$ is called the preference zone (PZ). The seminal work dates
back to Bechhofer (1954), with early work compiled in the monograph
Bechhofer et al. (1968). The progress in the area has been summarized
in Bechhofer et al. (1995), Swisher et al. (2003), Kim and Nelson
(2006, 2007) and Frazier (2014).

The goal of an IZ algorithm is to take as few samples as possible
while the IZ guarantee is satisfied. The first IZ procedures presented
in Bechhofer (1954), Paulson (1964), Fabian (1974), Rinott (1978),
Hartmann (1988, 1991), Paulson (1994) satisfy the IZ guarantee, but
they usually take too many samples when there are many alternatives,
in part because their probability of correct selection (PCS) is much
larger than the probability specified by the user. A reason of this
is the use of the Bonferroni's inequality which leads to sample more
than necessary. More recent algorithms in Kim and Nelson (2001), Goldsman
et al. (2002), Hong (2006) improve the performance but they still
use the Bonferroni's inequality, and so the methods are inefficient
when there are several systems. Procedures in Kim and Dieker (2011),
Dieker and Kim (2012) do not use the Bonferroni's inequality only
when compare groups of three alternatives. 

Since the classic IZ procedures take too many samples with many alternatives,
these methods are unpopular when there are more than a few hundred
alternatives. However, Frazier (2014) presented a new sequential elimination
IZ procedure, called BIZ (Bayes-inspired Indifference Zone), whose
lower bound on worst-case probability of correct selection in the
preference zone is tight in continuous time, and almost tight in the
discrete time. In numerical experiments, the number of samples required
by BIZ is significantly smaller than that of the most popular IZ procedures,
especially on problems with many alternatives. Unfortunately, the
proof that the BIZ procedure satisfies the IZ guarantee for the discrete-time
case assumes that variances are known and have an integer multiple
structure which is not very realistic. In practice, variances are
unknown. However, asymptotically we can use a central limit theorem
that allows us to prove the asymptotic validity of the BIZ procedure
for the discrete-time case. Moreover, we only need to assume that
the systems are independent, identically distributed and have finite
variance.

Kim et al. (2006) also proves the asymptotical validity of a IZ procedure.
Our proof is larger because the BIZ procedure is more sophisticated
and the bound for the PCS is tighter. To prove the case when the variances
are known, we use a theorem for Ergodic processes that shows how to
standardize the output data to make them behave like Brownian motion
processes in the limit. We also use an extension of the Continuous
Mapping Theorem (Theorem 5.5 of Billingsley 1968) to see that the
algorithm behaves like a sequential elimination IZ procedure with
a Brownian motion process instead of the standardized sum of the output
data in the limit, and then we use the results of the paper of Frazier
\cite{key-5} to prove the validity of this algorithm in the limit.
Finally, we use a random change of time argument to prove the case
when the variances are unknown.

This paper is organized as follows: In $\text{�}2$, we recall the
indifference-zone ranking and selection problem. In $\text{�}3$,
we recall the Bayes-inspired IZ (BIZ) procedure. In $\text{�}4$,
we present the proof of the validity of the algorithm when the variances
are known. In $\text{�}5$, we prove the case when the variances are
unknown. In $\text{�}6$, we present some simulations showing the
effectiveness of the algorithm. In $\text{�}7$, we conclude.


\section{Indifference-Zone Ranking and Selection }

Ranking and Selection is a procedure for selecting the best system
among a finite set of alternatives, i.e. the system with the largest
mean. The method selects a system as the best based on the samples
that are observed sequentially over the time. If the best system is
selected, we say that the procedure has made the \emph{correct selection}
(CS). We define the \emph{probability of correct selection} as 
\[
\mbox{PCS}\left(\mu,\lambda\right)=\mathbb{P}_{\mu,\lambda}\left(\hat{x}\in\mbox{arg max}_{x}\mu_{x}\right)
\]
where $\hat{x}$ is the alternative chosen by the procedure and $\mathbb{P}_{\mu,\lambda}$
is the probability measure under which samples from system $x$ have
mean $\mu_{x}$ and finite variance $\lambda_{x}^{2}$.

In the Indifference-Zone Ranking and Selection, the procedure is indifferent
in the selection of a system whenever the means of the populations
are nearly the same. Formally, let $\mu=\left[\mu_{k},\ldots,\mu_{1}\right]$
be the vector of the true means, the \emph{indifference zone} is defined
as the set $\left\{ \mu\in\mathbb{R}^{k}:\mu_{\left[k\right]}-\mu_{\left[k-1\right]}<\delta\right\} $.
The complement of the indifference zone is called the \emph{preference
zone} (PZ) and $\delta>0$ is called the indifference zone parameter.
We say that a procedure meets the \emph{indifference-zone (IZ) guarantee
}at $P^{*}\in\left(1/k,1\right)$ and $\delta>0$ if
\[
\mbox{PCS}\left(\mu,\lambda\right)\geq P^{*}\mbox{ for all }\mu\in\mbox{PZ}\left(\delta\right).
\]
We assume $P^{*}>1/k$ because IZ guarantees can be meet by choosing
$\hat{x}$ uniformly at random from among $\left\{ 1,\ldots,k\right\} $.


\section{The Bayes-inspired IZ (BIZ) Procedure}

BIZ is an elimination procedure. This procedure maintains a set of
alternatives that are in contention, and it takes samples from each
alternative in this set at each point in time. At beginning, all alternatives
are in contention, and over the time alternatives are eliminated.
The procedure ends when there is only one alternative in the contention
set and this remain alternative is chosen as the best. 

Frazier (2014) showed that the BIZ procedure with known common variance
satisfies the IZ guarantee when the systems follow the normal distribution
, with tight bounds on worst-case preference-zone in continuous time.
He also proved that this procedure retains the IZ guarantee when the
systems follow the normal distribution, and the variances are known
and are integer multiples of a common value. The continuous time version
of this procedure also satisfies the IZ guarantee, with a tight worst-case
preference-zone PCS bound.

The discrete-time BIZ procedure for unknown and/or heterogeneous sampling
variances is given in Alg. 2. It takes a variable number of samples
from each alternative, and $n_{tx}$ is this number. This algorithm
depends on a collection of integers $B_{1},\ldots,B_{k}$, $P^{*},c,\delta$
and $n_{0}$. $n_{0}$ is the number of samples to use in the first
stage of samples, and $100$ is the recommended value for $n_{0}$.
$B_{x}$ controls the number of samples taken from system $x$ in
each stage. 

For each $t$, $x\in\left\{ 1,\ldots,k\right\} $, and subset $A\subset\left\{ 1,\ldots,k\right\} $,
we define a function
\[
q'_{tx}\left(A\right)=\mbox{exp}\left(\delta\beta_{t}\frac{Z_{tx}}{n_{tx}}\right)\left/\sum_{x'\in A}\mbox{exp}\left(\delta\beta_{t}\frac{Z_{tx'}}{n_{tx'}}\right),\right.\mbox{ }\beta_{t}=\frac{\sum_{x'\in A}n_{tx'}}{\sum_{x'\in A}\hat{\lambda}_{tx'}^{2}}
\]
where $\hat{\lambda}_{tx'}^{2}$ is the sample variance of all samples
from alternative $x$ thus far and $Z_{tx}=Y_{n_{tx},x}$.

   
\paragraph{Algorithm: Discrete-time implementation of BIZ, for unknown and/or heterogeneous variances.}    
\begin{algorithmic}[1]   
\label{alg:hetero-BIZ}   
\REQUIRE $c \in [0,\cmax]$, $\delta>0$, $P^*\in(1/k,1)$, $n_0\ge0$ an integer, $B_1,\ldots,B_k$ strictly positive integers.  Recommended choices are $c=\cmax$, $B_1=\cdots=B_k=1$ and $n_0$ between $10$ and $30$.     If the sampling variances $\lambda^2_x$ are known, replace the estimators     
$\lambdahat^2_{tx}$ with the true values $\lambda^2_x$, and set $n_0=0$.     

\STATE For each $x$, sample alternative $x$ $n_0$ times and set $n_{0x} \leftarrow n_0$.     
Let $W_{0x}$ and $\lambdahat^2_{0x}$ be the sample mean and sample variance respectively of these samples.     Let $t\leftarrow 0$.Let $z \in \mbox{arg max}_{x\in A} \lambdahat^2_{tx}$.     
\STATE Let $A \leftarrow \{ 1,\ldots, k\}$, $\upthresh \leftarrow P^*$.
\WHILE{$x\in\mbox{max}_{x\in A} q'_{tx}\left(A\right)<P$}
\WHILE{$\mbox{min}_{x\in A} q'_{tx}\left(A\right) \le c$}
 \STATE Let $x\in\mbox{arg min}_{x\in A} q_{tx}\left(A\right)$.
    \STATE Let $\upthresh \leftarrow \upthresh/(1-q_{tx}\left(A\right))$.     
\STATE Remove $x$ from $A$.
\ENDWHILE    
\STATE For each $x\in A$, let      $n_{t+1,x} = \ceil\left( \lambdahat^2_{tx} (n_{tz} + B_z) / \lambdahat^2_{tz} \right)$.     \STATE For each $x\in A$, if $n_{t+1,x}>n_{tx}$, take $n_{t+1,x}-n_{tx}$ additional samples from alternative $x$.  Let $W_{t+1,x}$ and $\lambdahat^2_{t+1,x}$ be the sample mean and sample variance respectively of all samples from alternative $x$ thus far.    
\STATE Increment $t$.
 \ENDWHILE
  \STATE Select $\xhat \in\mbox{arg max}_{x\in A} Z_{tx} / n_{tx}$ as our estimate of the best.

   
\end{algorithmic} 

\vspace{10mm}
This algorithm generalizes the BIZ procedure with known common variance. In that case, we have that$B_1=\cdots=B_k=1$ and $n_{tx}=t$. The algorithm 2 can be generalized to the continuous case (See appendix B and Frazier (2014)). This procedure is a slightly modification of the original BIZ procedure where $z \in \mbox{arg max}_{x\in A} \lambdahat^2_{tx}$, instead of $z \in \mbox{arg min}_{x\in A} n_{tx} / \lambdahat^2_{tx}$.



\section{Asymptotic Validity when the Variances are Known}

In this section we prove that the BIZ procedure satisfies the IZ guarantee
when the variances are known and $\delta\rightarrow0$, where the
vector of the true means $\mu=\left[\mu_{k},\ldots,\mu_{1}\right]$
is equal to $\delta a$ for some $a\in\mathbb{R}^{k}$ and $\delta>0$.
Without loss of generality suppose that the true means satisfy that
$\mu_{k}>\mu_{k-1}\geq\cdots\geq\mu_{1}$. We suppose that samples
from system $x\in\left\{ 1\ldots,k\right\} $ are identically distributed
and independent, over time and across alternatives. We also define
$\lambda_{z}^{2}:=\max_{i\in\left\{ 1\ldots,k\right\} }\lambda_{i}^{2}$
and we suppose that $\mbox{min}{}_{i\in\left\{ 1\ldots,k\right\} }\lambda_{i}^{2}>0$
and $\mbox{max}{}_{i\in\left\{ 1\ldots,k\right\} }\lambda_{i}^{2}<\infty$. 

Any ranking and selection algorithm can be viewed as mapping from
paths of the $k$-dimensional discrete-time random walk $\left(Y_{tx}:t\in\mathbb{N},x\in\left\{ 1,\ldots,k\right\} \right)$
onto selection decisions. The BIZ procedure's mapping from paths onto
selections decisions can be understood as the composition of three
simpler maps.

The first is the mapping from the raw discrete-time random walk $\left(Y_{tx}:t\in\mathbb{N},x\in\left\{ 1,\ldots,k\right\} \right)$
onto a time changed version of this random walk, written as $\left(Z_{tx}:t\in\mathbb{N},x\in\left\{ 1,\ldots,k\right\} \right)$,
where we recall $Z_{tx}=Y_{n_{x}\left(t\right),t}$ is the sum of
the samples from alternative $x$ observed by stage $t$.

The second maps this time-changed random walk through a non-linear
mapping for each $t,x$ and subset $A\subset\left\{ 1,\ldots,k\right\} $,
to obtain $\left(q_{tx}^{'}\left(A\right):t\in\mathbb{N},A\subset\left\{ 1,\ldots,k\right\} ,x\in A\right)$,
where

\[
q'_{tx}\left(A\right)=\mbox{exp}\left(\delta\beta_{t}\frac{Z_{tx}}{n_{tx}}\right)\left/\sum_{x'\in A}\mbox{exp}\left(\delta\beta_{t}\frac{Z_{tx}}{n_{tx}}\right)\right.:=q'\left(\left(Z_{tx},t:x\in A\right),\delta,t\right)
\]
where we note that $n_{x}\left(t\right)$ and $\beta_{t}$ are deterministic
in the version of the known-variance BIZ procedure that we consider
here.

The third maps the paths of $\left(q_{tx}^{'}\left(A\right):t\in\mathbb{N},A\subset\left\{ 1,\ldots,k\right\} ,x\in A\right)$
onto selection decisions. Specifically, this mapping finds the first
time that $\mbox{max}_{x\in A}q_{tx}\left(A\right)$ is larger than
some threshold $P$, and then it chooses the system with the largest
empirical mean $Z_{tx}/n_{tx}$.

Later, we can approximate the previous process by the continuous time
process
\begin{eqnarray}
q_{tx}\left(A\right) & = & q\left(\left(Z_{\frac{t}{\delta^{2}}x},\frac{t}{\delta^{2}}:x\in A\right),\delta,A\subset\left\{ 1,\ldots,k\right\} \right)\label{eq:0.001-1}
\end{eqnarray}
where $t\geq0$ . Then we will show the difference in selection decisions
vanishes as $\delta\rightarrow0$.

After that, we would like to use a kind of central limit theorem for
$Z_{\frac{t}{\delta^{2}}x}$ when $\delta$ goes to zero to work with
a Brownian motion instead. In order to do this, we are going to rewrite
equation (1) in terms of a centralized version of $Z$:

\[
\mathcal{C}_{x}\left(\delta,t\right)=\frac{Y_{n_{x}\left(t\right),x}-t\lambda_{x}^{2}\mu_{x}}{\frac{\lambda_{x}^{2}}{\lambda_{z}\delta}}
\]
and the resulting expresion is

\[
q_{tx}\left(A\right)=q\left(\left(C_{x}\left(\delta,t\right)\frac{\lambda_{x}^{2}}{\delta\lambda_{z}^{2}}+\frac{\lambda_{x}^{2}}{\lambda_{z}^{2}}\left(n_{0}+\frac{t}{\delta^{2}}\right)\delta a_{x},\frac{t}{\delta^{2}}:x\in A\right),\delta,A\subset\left\{ 1,\ldots,k\right\} \right).
\]


Thus, we can see the selection decision resulting from the BIZ algorithm
as the result of applying some mapping $f\left(\cdot,\delta\right)$
to the paths $t\mapsto\mathcal{C}_{x}\left(\delta,t\right)$ .

The last part of our proof is to observe that (1) $\mathcal{C}\left(\delta,\cdot\right)$
converges to a multivariate Brownian motion $W$ as $\delta$ goes
to 0; (2) the function $f$ has a continuity property that causes
\[
f\left(\mathcal{C}\left(\delta,\cdot\right),\delta\right)\Rightarrow g\left(W\right)
\]
where g is the selection decision from applying the BIZ procedure
in continuous time; and (3) the BIZ procedure satsifies the IZ guarantee
when applied in continuous time.

First, we are going to see that the the standardized sum of the output
data converges to a Brownian motion in the sense of $D_{\infty}:=D[0,\infty)$,
which is the set of functions from $\left[0,\infty\right)$ to $\mathbb{R}$
that are right-continuous and have left-hand limits, with the Skorohod
topology. The definition and the properties of this topology may be
found in Chapter 3 of Billingsley 1999 and the appendix. 

We briefly recall the definition of convergence of random paths in
the sense of $D_{\infty}$. Suppose that we have a sequence of random
paths $\left(\mathcal{X}_{n}\right)_{n\geq0}^{\infty}$ such that
$\mathcal{X}_{n}:\varOmega\rightarrow D_{\infty}$ where $\left(\Omega,\mathcal{F},\mathbb{P}\right)$
is our probability space. We say that $\mathcal{X}_{n}\Rightarrow\mathcal{X}_{0}$
in the sense of $D_{\infty}$ if $P_{n}\Rightarrow P_{0}$ where $P_{n}:\mathcal{D}_{\infty}\rightarrow\left[0,1\right]$
are defined as $P_{n}\left[A\right]=\mathbb{P}\left[\mathcal{X}_{n}^{-1}\left(A\right)\right]$
for all $n\geq0$ and $\mathcal{D}_{\infty}$ are the Borel subsets
for the Skorohod topology.

The following lemma shows that the standardized sum of the output
data with $t$ changed by $t/\delta^{2}$ converges to a Brownian
motion in the sense of $D_{\infty}$. 


\paragraph*{Lemma 1.}

For each $x\in\left\{ 1\ldots,k\right\} $, we define
\[
\mathcal{C}_{x}\left(\delta,t\right)=\frac{Y_{n_{x}\left(t\right),x}-t\lambda_{x}^{2}\mu_{x}}{\frac{\lambda_{x}^{2}}{\lambda_{z}\delta}}
\]
where $Y_{n,x}$ is the sum of the first $n$ samples from alternative
$x$. Let $x\in\left\{ 1\ldots,k\right\} $, then
\[
\mathcal{C}_{x}\left(\delta,\cdot\right)\Rightarrow W_{x}\left(\cdot\right)
\]
as $\delta\rightarrow0$ in the sense of $D[0,\infty)$, where $W_{x}$
is a standard Brownian motion.


\paragraph*{Proof.}

By Theorem 19.1 of Billingsley 1999,

\[
\frac{Y_{n_{x}\left(t\right),x}-\mbox{floor}\left(\frac{\lambda_{x}^{2}}{\lambda_{z}^{2}}\left(\cdot\frac{1}{\delta^{2}}\right)\right)\mu_{x}}{\frac{\lambda_{x}^{2}}{\lambda_{z}}\sqrt{\frac{1}{\delta^{2}}}}\Rightarrow W_{x}\left(\cdot\right)
\]
in the sense of $D[0,\infty)$. 

Fix $w\in\Omega$. Observe that
\[
\frac{Y_{\mbox{floor}\left(\frac{\lambda_{x}^{2}}{\lambda_{z}^{2}}\left(\cdot\frac{1}{\delta^{2}}\right)\right),x}-\mbox{floor}\left(\frac{\lambda_{x}^{2}}{\lambda_{z}^{2}}\left(\cdot\frac{1}{\delta^{2}}\right)\right)\mu_{x}}{\frac{\lambda_{x}^{2}}{\lambda_{z}}\sqrt{\frac{1}{\delta^{2}}}}-\frac{Y_{\mbox{ceil}\left(\frac{\lambda_{x}^{2}}{\lambda_{z}^{2}}\left(\cdot\frac{1}{\delta^{2}}\right)\right),x}-\mbox{ceil}\left(\frac{\lambda_{x}^{2}}{\lambda_{z}^{2}}\left(\cdot\frac{1}{\delta^{2}}\right)\right)\mu_{x}}{\frac{\lambda_{x}^{2}}{\lambda_{z}}\sqrt{\frac{1}{\delta^{2}}}}\rightarrow0
\]
uniformly in $\left[0,s\right]$ for all $s\geq0$ and then by Theorem
A.2 
\[
\frac{Y_{\mbox{ceil}\left(\frac{\lambda_{x}^{2}}{\lambda_{z}^{2}}\left(\cdot\frac{1}{\delta^{2}}\right)\right),x}-\mbox{ceil}\left(\frac{\lambda_{x}^{2}}{\lambda_{z}^{2}}\left(\cdot\frac{1}{\delta^{2}}\right)\right)\mu_{x}}{\frac{\lambda_{x}^{2}}{\lambda_{z}}\sqrt{\frac{1}{\delta^{2}}}}\Rightarrow W_{x}\left(\cdot\right)
\]
in the sense of $D[0,\infty)$. 

Since $\frac{\frac{\lambda_{x}^{2}}{\lambda_{z}^{2}}t\frac{1}{\delta^{2}}-ceil\left(\frac{\lambda_{x}^{2}}{\lambda_{z}^{2}}t\frac{1}{\delta^{2}}\right)}{\frac{\lambda_{x}^{2}}{\lambda_{z}}\sqrt{\frac{1}{\delta^{2}}}}\rightarrow0$
uniformly on $[0,s]$ for every $s\geq0$, then by Theorem A.2

\[
\frac{Y_{\mbox{ceil}\left(\frac{\lambda_{x}^{2}}{\lambda_{z}^{2}}\left(\cdot\frac{1}{\delta^{2}}\right)\right),x}-\mbox{}\left(\frac{\lambda_{x}^{2}}{\lambda_{z}^{2}}\left(\cdot\frac{1}{\delta^{2}}\right)\right)\mu_{x}}{\frac{\lambda_{x}^{2}}{\lambda_{z}}\sqrt{\frac{1}{\delta^{2}}}}\Rightarrow W_{x}\left(\cdot\right).
\]


Finally, observe that for fixed $\omega\in\Omega$,
\begin{eqnarray*}
 &  & \frac{Y_{\mbox{ceil}\left(\frac{\lambda_{x}^{2}}{\lambda_{z}^{2}}\left(\cdot\frac{1}{\delta^{2}}\right)\right),x}-\mbox{}\left(\frac{\lambda_{x}^{2}}{\lambda_{z}^{2}}\left(\cdot\frac{1}{\delta^{2}}\right)\right)\mu_{x}}{\frac{\lambda_{x}^{2}}{\lambda_{z}}\sqrt{\frac{1}{\delta^{2}}}}-\frac{Y_{\mbox{ceil}\left(\frac{\lambda_{x}^{2}}{\lambda_{z}^{2}}\left(\cdot\frac{1}{\delta^{2}}\right)+n_{0}\frac{\lambda_{x}^{2}}{\lambda_{z}^{2}}\right),x}-\mbox{}\left(n_{0}\frac{\lambda_{x}^{2}}{\lambda_{z}^{2}}+\frac{\lambda_{x}^{2}}{\lambda_{z}^{2}}\left(\cdot\frac{1}{\delta^{2}}\right)\right)\mu_{x}}{\frac{\lambda_{x}^{2}}{\lambda_{z}}\sqrt{\frac{1}{\delta^{2}}}}\\
 & = & \frac{Y_{\mbox{ceil}\left(\frac{\lambda_{x}^{2}}{\lambda_{z}^{2}}\left(\cdot\frac{1}{\delta^{2}}\right)\right),x}-Y_{\mbox{ceil}\left(\frac{\lambda_{x}^{2}}{\lambda_{z}^{2}}\left(\cdot\frac{1}{\delta^{2}}\right)+n_{0}\frac{\lambda_{x}^{2}}{\lambda_{z}^{2}}\right),x}+\left(n_{0}\frac{\lambda_{x}^{2}}{\lambda_{z}^{2}}\right)\mu_{x}}{\frac{\lambda_{x}^{2}}{\lambda_{z}}\sqrt{\frac{1}{\delta^{2}}}}\\
 & \rightarrow & 0
\end{eqnarray*}
uniformly in $\left[0,t\right]$ for all $t\geq0$, and so by Theorem
A.2 the result follows.

$\;\;\;\;\;\;\;\;\;\;\;\;\;\;\;\;\;\;\;\;\;\;\;\;\;\;\;\;\;\;\;\;\;\;\;\;\;\;\;\;\;\;\;\;\;\;\;\;\;\;\;\;\;\;\;\;\;\;\;\;\;\;\;\;\;\;\;\;\;\;\;\;\;\;\;\;\;\;\;\;\;\;\;\;\;\;\;\;\;\;\;\;\;\;\;\;\;\;\;\;\;\;\;\;\;\;\;\;\;\;\;\;\;\;\;\;\;\;\;\;\;\;\;\;\;\;\;\;\;\;\;\;\;\;\;\;\;\;\;\text{\ensuremath{\blacksquare}}$

${\color{white}sssssssssssssssssssssssssssssssssssssssssssssssssssssssssssssssssssssssssssssssssssssssssssss}$$\;\;\;\;\;\;\;\;\;\;\;\;\;\;\;\;\;\;\;\;\;\;\;\;\;\;\;\;\;\;\;\;\;\;\;\;\;\;\;\;\;\;\;\;\;\;\;\;\;\;\;\;\;\;\;\;\;\;\;\;\;\;\;\;\;\;\;\;\;\;\;\;\;\;\;\;\;\;\;\;\;\;\;\;\;\;\;\;\;\;\;\;\;\;\;\;\;\;\;\;\;\;\;\;\;\;\;\;\;\;\;\;\;\;\;\;\;\;\;\;\;\;\;\;\;\;\;\;\;\;\;\;\;\;\;\;\;\;\;\text{\ensuremath{\blacksquare}}$

Now, we use the product topology in $D^{k}\left[0,\infty\right)$
for $k\in\mathbb{N}$. This topology may be described as the one under
which $\left(Z_{n}^{1},\ldots,Z_{n}^{k}\right)\rightarrow\left(Z_{0}^{1},\ldots,Z_{0}^{k}\right)$
if and only if $Z_{n}^{i}\rightarrow Z_{0}^{i}$ for all $i\in\left\{ 1,\ldots,k\right\} $.
See the Miscellany of Billingsley 1968. The following corollary follows
from the previous result and independence.


\paragraph*{Corollary 1.}

We have that
\[
\mathcal{C}\left(\delta,\cdot\right):=\left(\mathcal{C}_{x}\left(\delta,\cdot\right)\right)_{x\in A}\Rightarrow W\left(\cdot\right):=\left(W_{x}\left(\cdot\right)\right)_{x\in A}
\]
as $\delta\rightarrow0$ in the sense of $D_{\infty}^{k}$.\\


Observe that $\mathcal{C}\left(\delta,\cdot\right)\in D\left[0,\infty\right)^{k}$
and the convergence of Lemma 1 is in $D\left[0,\infty\right)^{k}$,
consequently the following natural step is to generalize the continuous-time
procedure for the heterogeneous variance setting, see Appendix B,
to the space $D\left[0,\infty\right)^{k}$. Specifically, for each
$F\in D\left[0,\infty\right)^{k}$, we define $q_{tx}^{F,\delta}\left(A\right)$
as
\[
q_{tx}^{F,\delta}\left(A\right)=q'\left(\left(F_{x}\left(t\right)\frac{\lambda_{x}^{2}}{\delta\lambda_{z}^{2}}+\frac{\lambda_{x}^{2}}{\lambda_{z}^{2}}\left(n_{0}+\frac{t}{\delta^{2}}\right)\delta a_{x},\frac{t}{\delta^{2}}:x\in A\right),\delta,A\subset\left\{ 1,\ldots,k\right\} \right).
\]


Note that if we replace $F$ by $\mathcal{C}\left(\delta,t\right)$,
we get $q{}_{tx}\left(A\right)$ in the previous equation. Using the
continuous-time procedure for the heterogeneous variance setting with
the paths $\left(q_{tx}^{F,\delta}\left(A\right):t\geq0\right)$,
we see that there exists a selection decision for each $F\in D\left[0,\infty\right)^{k}$.
Since we are interested in the right selection, ( i.e. in choosing
$k$) for each $F\in D\left[0,\infty\right)^{k}$, we define the following
function 
\begin{eqnarray*}
f\left(F,\delta\right) & = & \begin{cases}
1 & \mbox{if }k\mbox{ is chosen}\mbox{ }\\
0 & \mbox{otherwise}
\end{cases}
\end{eqnarray*}
 Observe that $f$ is a function that depends in the path $\left\{ q_{tx}^{F,\delta}\left(A\right):t\geq0\right\} $.
We are doing all this because we want to see that $f\left(\mathcal{C}\left(\delta,\cdot\right),\delta\right)$
looks like the right or wrong selection decision of an algorithm that
depends on a standard Brownian motion instead of $\mathcal{C}\left(\delta,\cdot\right)$,
and this will give us the proof because we will be able to use the
theory developed by Frazier (2014) to prove the theorem. Thus, the
following step is to find the limit of $q_{tx}^{F,\delta}\left(A\right)$
when $\delta$ goes to zero, which is called $q_{tx}^{F}\left(A\right)$
and equal to

\[
q_{tx}^{F}\left(A\right):=\mbox{exp}\left(\frac{F_{x}\left(t\right)}{\lambda_{z}}+\frac{1}{\lambda_{z}^{2}}ta_{x}\right)/\sum_{x^{'}\in A}\mbox{exp}\left(\frac{F_{x'}\left(t\right)}{\lambda_{z}}+\frac{1}{\lambda_{z}^{2}}ta_{x^{'}}\right).
\]
 Using the same previous idea, for each $F\in D\left[0,\infty\right)^{k}$,
we define the new functions
\[
\]
\begin{eqnarray*}
g\left(F\right) & = & \begin{cases}
1 & \mbox{if }k\mbox{ is chosen}\mbox{ }\\
0 & \mbox{otherwise}
\end{cases}.
\end{eqnarray*}


So, we want to prove that 
\[
f\left(\mathcal{C}\left(\delta,\cdot\right),\delta\right)\Rightarrow g\left(W\right)
\]
as $\delta\rightarrow0$ in distribution.

In order to prove this, we first prove a continuity property in Lemma
2, which will allow us to use Theorem 5.5 of Billingsley 1968, which
implies the desired result.


\paragraph*{Lemma 2.}

Let $\left\{ \delta_{n}\right\} \subset\left(0,\infty\right)$ such
that $\delta_{n}\rightarrow0$. If $D_{s}\equiv\{Z\in D\left[0,\infty\right)^{k}:\mbox{ if }\left\{ Z_{n}\right\} \subset D\left[0,\infty\right)^{k}\mbox{ and }$
$\mbox{lim}{}_{n}d_{\infty}\left(Z_{n},Z\right)=0$ , then the sequence
$\left\{ f\left(Z_{n},\delta_{n}\right)\right\} $ converges to $\left\{ g\left(Z\right)\right\} $$\left.\right\} $$ $,
then $ $$\mathbb{P}\left(W\text{\ensuremath{\in}}D_{s}\right)=1$.\\


First, we are going to prove the following five results.\\



\paragraph*{Proposition 1.}

Suppose $\left\{ f_{n}\right\} $ and $\left\{ g_{n}\right\} $ are
two sequences of functions on $D_{\infty}$ such that $f_{n}\rightarrow f$
and $g_{n}\rightarrow g$ in $D_{\infty}$. If $f$ and $g$ are continuous,
then
\[
\mbox{min}\left(f_{n},g_{n}\right)\rightarrow\mbox{min}\left(f,g\right)
\]
in $D_{\infty}$.


\paragraph*{Proof.}

Let $t^{*}>0$. We will prove that $\mbox{min}\left(f_{n},g_{n}\right)\rightarrow\mbox{min}\left(f,g\right)$
in $D_{t^{*}}$ and the theorem will follow from Theorem 16.2 of Billingsley
1999. 

Since $f$ and $g$ are uniformly continuous in $\left[0,t^{*}\right]$
and $f_{n}\rightarrow f$ and $g_{n}\rightarrow g$ in $\left[0,t^{*}\right]$,
then $ $ $f_{n}$ and $g_{n}$ converge uniformly to $f$ and $g$,
respectively. Consequently, $\left(f_{n},g_{n}\right)\rightarrow\left(f,g\right)$
uniformly in $\left[0,t^{*}\right]$. Let $a_{f}^{+}=\mbox{max}_{t\in\left[0,t^{*}\right]}\left|f(t)\right|$,
$a_{f}^{-}=-a_{f}^{+}$, $a_{g}^{+}=\mbox{max}_{t\in\left[0,t^{*}\right]}\left|g(t)\right|$
and $a_{g}^{-}=-a_{g}^{+}$. Let $N$ such that if $n\geq N$, implies
$\left|f_{n}\left(t\right)-f\left(t\right)\right|<1$ and $\left|g_{n}\left(t\right)-g\left(t\right)\right|<1$
for all $t\in\left[0,t^{*}\right]$. Consequently, if $n\geq N$,$ $
$f_{n}\left(t\right)\in\left[a_{f}^{-}-1,a_{f}^{+}+1\right]$ and
$g_{n}\left(t\right)\in\left[a_{g}^{-}-1,a_{g}^{+}+1\right]$ for
all $t$ in $\left[0,t^{*}\right]$. Since $\mbox{min}\left(x,y\right)$
is continuous, then it is uniformly continuous in $A=\left[a_{f}^{-}-1,a_{f}^{+}+1\right]\times\left[a_{g}^{-}-1,a_{g}^{+}+1\right]$.

Let $\epsilon>0$, then there exists $\delta>0$ such that if $\left\Vert \left(u,v\right)\right\Vert _{2}<\delta$
and $ $$\left(u:=\left(u_{1},u_{2}\right),v:=\left(v_{1},v_{2}\right)\right)\in A$,
then 
\begin{equation}
\left|\mbox{min}\left(u_{1},u_{2}\right)-\mbox{min}\left(v_{1},v_{2}\right)\right|<\epsilon.\label{eq:1.10}
\end{equation}
Let $M>N$ such that if $n>M$, then
\[
\left\Vert \left(f_{n}\left(t\right),g_{n}\left(t\right)\right)-\left(f\left(t\right),g\left(t\right)\right)\right\Vert _{2}<\delta
\]
for all $t\in\left[0,t^{*}\right]$. Consequently, if $n>M$, by (\ref{eq:1.10}),
\[
\left|\mbox{min}\left(f_{n}\left(t\right),g_{n}\left(t\right)\right)-\mbox{min}\left(f\left(t\right),g\left(t\right)\right)\right|<\epsilon
\]
for all $t\in\left[0,t^{*}\right]$. Since uniform convergence implies
Skorohod convergence, then $\mbox{min}\left(f_{n},g_{n}\right)\rightarrow\mbox{min}\left(f,g\right)$
in $D_{t^{*}}$, and the result follows.

$\;\;\;\;\;\;\;\;\;\;\;\;\;\;\;\;\;\;\;\;\;\;\;\;\;\;\;\;\;\;\;\;\;\;\;\;\;\;\;\;\;\;\;\;\;\;\;\;\;\;\;\;\;\;\;\;\;\;\;\;\;\;\;\;\;\;\;\;\;\;\;\;\;\;\;\;\;\;\;\;\;\;\;\;\;\;\;\;\;\;\;\;\;\;\;\;\;\;\;\;\;\;\;\;\;\;\;\;\;\;\;\;\;\;\;\;\;\;\;\;\;\;\;\;\;\;\;\;\;\;\;\;\;\;\;\;\;\;\;\text{\ensuremath{\blacksquare}}$

${\color{white}sssssssssssssssssssssssssssssssssssssssssssssssssssssssssssssssssssssssssssssssssssssssssssss}$$\;\;\;\;\;\;\;\;\;\;\;\;\;\;\;\;\;\;\;\;\;\;\;\;\;\;\;\;\;\;\;\;\;\;\;\;\;\;\;\;\;\;\;\;\;\;\;\;\;\;\;\;\;\;\;\;\;\;\;\;\;\;\;\;\;\;\;\;\;\;\;\;\;\;\;\;\;\;\;\;\;\;\;\;\;\;\;\;\;\;\;\;\;\;\;\;\;\;\;\;\;\;\;\;\;\;\;\;\;\;\;\;\;\;\;\;\;\;\;\;\;\;\;\;\;\;\;\;\;\;\;\;\;\;\;\;\;\;\;\text{\ensuremath{\blacksquare}}$


\paragraph*{Proposition 2.}

Suppose $\left\{ f_{n}\right\} $ is a sequence of functions in $D_{\infty}$
such that $f_{n}\rightarrow f$ in $D_{\infty}$ , $f$ is continuous,
and $\left\{ T_{n}\right\} \subset\left[0,\infty\right)$ is a sequence
such that $T_{n}\rightarrow T$. We define $T\left(a\right):=\mbox{inf}\left\{ t\geq T:f\left(t\right)\geq a\right\} $
for each $a\in\mathbb{R}$. Suppose $T\left(0\right)\in\mathbb{R}$.
Furthermore, we suppose that there exists $\left\{ \epsilon_{n}\right\} \subset\left(0,\infty\right)$
such that $\epsilon_{n}\rightarrow0$ , $\epsilon_{n}\geq\epsilon_{n+1}$,
and$ $
\begin{equation}
\left\Vert f_{n}\left(t\right)-f\left(t\right)\right\Vert _{2}<\epsilon_{n}\label{eq:2.11}
\end{equation}
for all $t\in\left[0,T\left(0\right)\right]$. We also suppose that
$\mbox{lim sup}_{n}T\left(\epsilon_{n}\right)\leq T\left(0\right)$.
Thus we have that

\[
\mbox{inf}\left\{ t\geq T_{n}:f_{n}\left(t\right)\geq0\right\} \rightarrow\mbox{inf}\left\{ t\geq T:f\left(t\right)\geq0\right\} 
\]
if $T\left(0\right)>T$ or $T_{n}=T$.


\paragraph*{Proof.}

We are going to suppose $T\left(0\right)>T$, and the case $T_{n}=T$
can be proved using almost the same ideas. We introduce the notation
$T_{n}\left(a\right):\mbox{inf}\left\{ t\geq T_{n}:f_{n}\left(t\right)\geq a\right\} $
for $a\in\mathbb{R}$. Since $T\left(0\right)>T$, we can take $N$
such that if $n>N$, then $T_{n}<T+T\left(0\right)-T=T\left(0\right)$
and $\epsilon>\epsilon_{n}$ where $\epsilon:=sup_{n}\epsilon_{n}$.
Let $n>N$. Note that $T_{n}\left(0\right)\leq T_{n}\left(\epsilon_{n}\right)$. 

We also have that
\begin{eqnarray}
T\left(0\right) & \leq & \mbox{lim inf}_{n}T\left(\epsilon_{n}\right)\nonumber \\
 & \leq & \mbox{lim sup}_{n}T\left(\epsilon_{n}\right)\label{eq:2.3}
\end{eqnarray}
Now, since $\mbox{lim sup}_{n}T\left(\epsilon_{n}\right)\leq T\left(0\right)$,
\begin{eqnarray*}
T\left(0\right) & \geq & \mbox{lim sup}_{n}T\left(\epsilon_{n}\right)\\
 & \geq & \mbox{lim inf}_{n}T\left(\epsilon_{n}\right)\\
 & \geq & T\left(0\right)
\end{eqnarray*}
and so 
\[
\mbox{lim inf}_{n}T_{n}\left(0\right)\leq\mbox{lim sup}_{n}T_{n}\left(0\right)\leq\mbox{lim sup}_{n}T_{n}\left(\epsilon_{n}\right)=\mbox{lim}_{n}T\left(\epsilon_{n}\right)=T\left(0\right)
\]


Now, let's prove that $\mbox{lim inf}_{n}T_{n}\left(0\right)\geq T\left(0\right)$.
Since $T\left(0\right)>T$, there exists $M$ such that $T\left(0\right)-\frac{1}{m}\geq T$
if $m>M$, and let $t_{m}=T\left(0\right)-\frac{1}{m}$ and $\alpha_{m}=\mbox{max}\left\{ f\left(t\right):t\in\left[T,t_{m}\right]\right\} $.
Note that $\alpha_{m}<0$ because $t_{m}<T\left(0\right)$. Since
$\epsilon_{n}\rightarrow0$, there exists $N$ such that if $n>N$,
then
\[
\epsilon_{n}\leq-\alpha_{m}
\]
Thus, since $\left\Vert f_{n}\left(t\right)-f\left(t\right)\right\Vert _{2}<\epsilon_{n}$
if $t\in\left[T,t_{m}\right]$, then
\begin{equation}
f_{n}\left(t\right)<f\left(t\right)+\epsilon_{n}\leq f\left(t\right)-\alpha_{m}\leq0\label{eq:20.1}
\end{equation}


If $T\leq T_{n}\left(0\right)$ for all $n>N$, then $T_{n}\left(0\right)\geq t_{m}$
by (\ref{eq:20.1}) and so $\mbox{lim inf}_{n}T_{n}\left(0\right)\geq t_{m}$.
Taking the limit $m\rightarrow\infty$, we conclude that $\mbox{lim inf}_{n}T_{n}\left(0\right)\geq T\left(0\right)$. 

So, we only need to prove that $T\leq T_{n}\left(0\right)$ for $n$
large. We proceed by contradiction. Let $s>0$ be any number. Take
$N$ such that if $n>N$, then $\epsilon_{n}<s$. Since we are supposing
that for all $N'$ there exists $n>N'$ such that $T>T_{n}\left(0\right)$,
and $T_{n}\left(0\right)\geq T_{n}$, $T_{n}\rightarrow T$, then
we have that for every $\epsilon>0$ there exists $n$ large such
that $T-\epsilon<T_{n}\left(0\right)<T$ . By (\ref{eq:2.11}), 
\begin{eqnarray*}
f\left(T_{n}\left(0\right)\right) & \geq & f_{n}\left(T_{n}\left(0\right)\right)-\epsilon_{n}\\
 & \geq & -\epsilon_{n}>-s
\end{eqnarray*}
Let $a>0$, since $f$ is continuous there exists $\epsilon>0$ such
that $f\left(x\right)<f\left(T\right)+a$ if $\left|x-T\right|<\epsilon$.
Thus, there exists $n>N$ such that $T-\epsilon<T_{n}\left(0\right)<T$
and so 
\[
-s<f\left(T_{n}\left(0\right)\right)<f\left(T\right)+a
\]


Since $s$ and $a$ are arbitrary, 
\[
f\left(T\right)\geq0
\]
which is a contradiction because $T\left(0\right)>T$. Consequently,
$T\leq T_{n}\left(0\right)$ and so
\[
\mbox{lim}_{n}T_{n}\left(0\right)=T\left(0\right).
\]


$\;\;\;\;\;\;\;\;\;\;\;\;\;\;\;\;\;\;\;\;\;\;\;\;\;\;\;\;\;\;\;\;\;\;\;\;\;\;\;\;\;\;\;\;\;\;\;\;\;\;\;\;\;\;\;\;\;\;\;\;\;\;\;\;\;\;\;\;\;\;\;\;\;\;\;\;\;\;\;\;\;\;\;\;\;\;\;\;\;\;\;\;\;\;\;\;\;\;\;\;\;\;\;\;\;\;\;\;\;\;\;\;\;\;\;\;\;\;\;\;\;\;\;\;\;\;\;\;\;\;\;\;\;\;\;\;\;\;\;\text{\ensuremath{\blacksquare}}$

${\color{white}sssssssssssssssssssssssssssssssssssssssssssssssssssssssssssssssssssssssssssssssssssssssssssss}$$\;\;\;\;\;\;\;\;\;\;\;\;\;\;\;\;\;\;\;\;\;\;\;\;\;\;\;\;\;\;\;\;\;\;\;\;\;\;\;\;\;\;\;\;\;\;\;\;\;\;\;\;\;\;\;\;\;\;\;\;\;\;\;\;\;\;\;\;\;\;\;\;\;\;\;\;\;\;\;\;\;\;\;\;\;\;\;\;\;\;\;\;\;\;\;\;\;\;\;\;\;\;\;\;\;\;\;\;\;\;\;\;\;\;\;\;\;\;\;\;\;\;\;\;\;\;\;\;\;\;\;\;\;\;\;\;\;\;\;\text{\ensuremath{\blacksquare}}$


\paragraph*{Proposition 3.}

Let $\mathcal{S}$ be the event such that $W$ is continuous and $\mathbb{P}\left[\mathcal{S}\right]=1$.
Fix $\omega\in\mathcal{S}$ and let $\left\{ Z_{n}\right\} \subset D\left[0,\infty\right)^{k}$
be a sequence of functions such that $Z_{n}\rightarrow W$ in $D^{k}\left[0,\infty\right)$,
then 
\begin{eqnarray*}
f_{Z_{n}}\left(\cdot\right): & = & \mbox{max}\left\{ c-\mbox{min}_{x\in A}q_{\cdot x}^{Z_{n},\delta_{n}}\left(A\right),\mbox{max}_{x\in A}q_{\cdot x}^{Z_{n},\delta_{n}}\left(A\right)-P\right\} \\
 & \rightarrow & f_{W}\left(\cdot\right):=\mbox{max}\left\{ c-\mbox{min}_{x\in A}q_{\cdot x}^{W}\left(A\right),\mbox{max}_{x\in A}q_{\cdot x}^{W}\left(A\right)-P\right\} 
\end{eqnarray*}


in $D\left[0,\infty\right)$ for any $P\in\left(0,1\right)$ and $ $$A\subset\left\{ 1,\ldots,k\right\} $.
Furthermore, if $m\in\mathbb{N}$, there exists a sequence $\left\{ \epsilon_{k}\right\} $
such that $\epsilon_{k}\downarrow0$ and 
\[
\left|f_{Z_{k}}\left(t\right)-f_{W}\left(t\right)\right|<\epsilon_{k}
\]
for all $t\in\left[0,m\right]$.


\paragraph*{Proof.}


\paragraph*{\textmd{Since $Z_{n}\rightarrow W$ in $D\left[0,\infty\right)$,
by Theorem 16.2 of Billingsley 1999, for each $s\geq0$ there exist
functions $\lambda_{s}^{n}$ in $\Lambda_{s}$ such that 
\[
\mbox{lim}_{n}Z_{n}\left(\lambda_{s}^{n}t\right)=W\left(t\right)
\]
uniformly in $t$ and
\[
\mbox{lim}_{n}\lambda_{s}^{n}t=t
\]
uniformly in $t$. Then
\[
\mbox{lim}_{n}\frac{\lambda_{x}^{2}\beta_{\delta}\left(\lambda_{s}^{n}t\right)}{\lambda_{z}\mbox{ceil}\left(\frac{\lambda_{x}^{2}}{\lambda_{z}^{2}}\left(n_{0}+\frac{\lambda_{s}^{n}\left(t\right)}{\delta^{2}}\right)\right)}Z_{n}^{x}\left(\lambda_{s}^{n}t\right)+\delta_{n}^{2}\lambda_{x}^{2}\frac{\left(n_{0}+\lambda_{s}^{n}\left(t\right)\frac{1}{\delta_{n}^{2}}\right)\beta_{\delta}\left(\lambda_{s}^{n}\left(t\right)\right)}{\lambda_{z}^{2}\mbox{ceil}\left(\frac{\lambda_{x}^{2}}{\lambda_{z}^{2}}\left(n_{0}+\frac{\lambda_{s}^{n}\left(t\right)}{\delta^{2}}\right)\right)}a_{x}=W_{x}\left(t\right)\frac{1}{\lambda_{z}}+\frac{t}{\lambda_{z}^{2}}a_{x}
\]
uniformly in $t$, and so
\[
\mbox{lim}_{n}\mbox{exp}\left(\frac{\lambda_{x}^{2}\beta_{\delta}\left(\lambda_{s}^{n}t\right)}{\lambda_{z}\mbox{ceil}\left(\frac{\lambda_{x}^{2}}{\lambda_{z}^{2}}\left(n_{0}+\frac{\lambda_{s}^{n}\left(t\right)}{\delta^{2}}\right)\right)}Z_{n}^{x}\left(\lambda_{s}^{n}t\right)+\delta_{n}^{2}\lambda_{x}^{2}\frac{\left(n_{0}+\lambda_{s}^{n}\left(t\right)\frac{1}{\delta_{n}^{2}}\right)\beta_{\delta}\left(\lambda_{s}^{n}\left(t\right)\right)}{\lambda_{z}^{2}\mbox{ceil}\left(\frac{\lambda_{x}^{2}}{\lambda_{z}^{2}}\left(n_{0}+\frac{\lambda_{s}^{n}\left(t\right)}{\delta^{2}}\right)\right)}\right)=\mbox{exp}\left(W_{x}\left(t\right)\frac{1}{\lambda_{z}}+\frac{t}{\lambda_{z}^{2}}a_{x}\right)
\]
uniformly in $t$ since $\mbox{exp}$ is uniformly continuous in $\left[0,s\right]$.
Consequently,
\[
q_{\lambda_{s}^{n}\left(t\right)x}^{Z_{n},\delta_{n}}\left(A\right)\rightarrow q_{tx}^{W}\left(A\right)
\]
uniformly in $t\in\left[0,s\right]$. Thus, $q_{\cdot x}^{Z_{n},\delta_{n}}\left(A\right)\rightarrow q_{\cdot x}^{W}\left(A\right)$
in $D\left[0,s\right]$ for any set $A\subset\left\{ 1,\ldots,k\right\} $
and $s\geq0$. Consequently, $q_{\cdot x}^{Z_{n},\delta_{n}}\left(A\right)\rightarrow q_{\cdot x}^{W}\left(A\right)$
in $D\left[0,\infty\right)$ by Theorem 16.2 of Billingsley 1999.
By Proposition 1, $\mbox{min}_{x\in A}q_{\cdot x}^{Z_{n},\delta_{n}}\left(A\right)\rightarrow\mbox{min}_{x\in A}q_{\cdot x}^{W}\left(A\right)$
and $\mbox{max}_{x\in A}q_{\cdot x}^{Z_{n},\delta_{n}}\left(A\right)\rightarrow\mbox{max}_{x\in A}q_{\cdot x}^{W}\left(A\right)$
in $D\left[0,\infty\right)$, and so by Proposition 1,
\[
f_{Z_{n}}:=\mbox{max}\left\{ c-\mbox{min}_{x\in A}q_{\cdot x}^{Z_{n},\delta_{n}}\left(A\right),\mbox{max}_{x\in A}q_{\cdot x}^{Z_{n},\delta_{n}}\left(A\right)-P\right\} \rightarrow f_{W}:=\mbox{max}\left\{ c-\mbox{min}_{x\in A}q_{\cdot x}^{W}\left(A\right),\mbox{max}_{x\in A}q_{\cdot x}^{W}\left(A\right)-P\right\} 
\]
in $D\left[0,\infty\right)$ for any $P\in\left(0,1\right)$.}}

Now, let's prove the second part. Fix $m\in\mathbb{N}$. By the definition
of $d_{m}\left(f_{Z_{n}},f_{W}\right)$, we have that there exists
$\lambda_{n}\in\Lambda_{\infty}$ such that 
\begin{eqnarray*}
\mbox{sup}_{t\leq m}\left\Vert \lambda_{n}\left(t\right)-t\right\Vert _{2} & \leq & d_{m}\left(f_{Z_{n}},f_{W}\right)+\frac{1}{n}\\
\mbox{sup}_{t\leq m}\left\Vert f_{Z_{n}}\left(t\right)-f_{W}\left(\lambda_{n}t\right)\right\Vert _{2} & \leq & d_{m}\left(f_{Z_{n}},f_{W}\right)+\frac{1}{n}
\end{eqnarray*}
for all $n$. Taking $g_{n}\equiv\mbox{sup}_{t\leq m}\left\Vert f_{W}\left(t\right)-f_{W}\left(\lambda_{n}t\right)\right\Vert _{2}$,
we see from the uniform continuity of $f_{W}$ on $\left[0,m\right]$
($f_{W}$ is uniformly continuous because it's continuos in a compact
set) and the definition of $g_{n}$ that $\mbox{lim}_{n\rightarrow\infty}g_{n}=0.$
Moreover, if we take $\epsilon_{n}=2n^{-1}+2\mbox{sup}\left\{ d_{m}\left(f_{Z_{l}},f_{W}\right)+g_{l}:l=n,n+1,\ldots\right\} $,
then $\left\{ \epsilon_{n}\right\} $ is a monotonically decreasing
sequence of positive numbers with limit zero by the previous result.

From the definition of $\epsilon_{n}$ we have that $d_{m}\left(f_{Z_{n}},f_{W}\right)<\epsilon_{n}/2-\frac{1}{n}$
and $g_{n}<\epsilon_{n}/2$ for $n=1,2,\ldots$ Consequently, we have
\begin{eqnarray*}
\left\Vert f_{Z_{n}}\left(t\right)-f_{W}\left(t\right)\right\Vert  & \leq & \left\Vert f_{Z_{n}}\left(t\right)-f_{W}\left(\lambda_{n}t\right)\right\Vert +\left\Vert f_{W}\left(\lambda_{n}t\right)-f_{W}\left(t\right)\right\Vert \\
 & < & d_{m}\left(f_{Z_{n}},f_{W}\right)+\frac{1}{n}+g_{n}\\
 & < & \epsilon_{n}
\end{eqnarray*}
for all $t\in\left[0,m\right]$.

$\;\;\;\;\;\;\;\;\;\;\;\;\;\;\;\;\;\;\;\;\;\;\;\;\;\;\;\;\;\;\;\;\;\;\;\;\;\;\;\;\;\;\;\;\;\;\;\;\;\;\;\;\;\;\;\;\;\;\;\;\;\;\;\;\;\;\;\;\;\;\;\;\;\;\;\;\;\;\;\;\;\;\;\;\;\;\;\;\;\;\;\;\;\;\;\;\;\;\;\;\;\;\;\;\;\;\;\;\;\;\;\;\;\;\;\;\;\;\;\;\;\;\;\;\;\;\;\;\;\;\;\;\;\;\;\;\;\;\;\text{\ensuremath{\blacksquare}}$

${\color{white}sssssssssssssssssssssssssssssssssssssssssssssssssssssssssssssssssssssssssssssssssssssssssssss}$$\;\;\;\;\;\;\;\;\;\;\;\;\;\;\;\;\;\;\;\;\;\;\;\;\;\;\;\;\;\;\;\;\;\;\;\;\;\;\;\;\;\;\;\;\;\;\;\;\;\;\;\;\;\;\;\;\;\;\;\;\;\;\;\;\;\;\;\;\;\;\;\;\;\;\;\;\;\;\;\;\;\;\;\;\;\;\;\;\;\;\;\;\;\;\;\;\;\;\;\;\;\;\;\;\;\;\;\;\;\;\;\;\;\;\;\;\;\;\;\;\;\;\;\;\;\;\;\;\;\;\;\;\;\;\;\;\;\;\;\text{\ensuremath{\blacksquare}}$


\paragraph*{Proposition 4.}

Let $W=\left(W_{1},\ldots,W_{k}\right)$ be a $k-$dimensional independent
standard Brownian motion. Let $q_{tx}^{W}\left(A\right):=\mbox{exp}\left(\frac{1}{\lambda_{z}^{2}}\left(\lambda_{z}W_{x}\left(t\right)+ta_{x}\right)\right)/\sum_{x^{'}\in A}\mbox{exp}\left(\frac{1}{\lambda_{z}^{2}}\left(\lambda_{z}W_{x^{'}}\left(t\right)+ta_{x^{'}}\right)\right)$
for all $x\in A$ and $A\subset\left\{ 1,\ldots,k\right\} $. Fix
$m\in\left\{ 1,\ldots,k-1\right\} $. We have that for all $N\in\mathbb{N}$,
there exists $t$ such that $\tau_{m}+\frac{1}{N}\geq t>\tau_{m}$
and $q_{tx}^{W}\left(A_{m-1}\right)>P_{m-1}$ for some $x\in A_{m-1}$
almost surely given that $M=m$. (See the definitions in p. 4).


\paragraph*{Proof.}

The proof has the same spirit than the proof of Lemma 4 of Frazier
\cite{key-5}, there Frazier proved that $\tau_{m}<\infty$ almost
surely. Furthermore, $M\leq k-1$ almost surely. We should note that
the Lemma 4 supposes that all systems have the same variance which
is the case here because $\mbox{Var}\left(\lambda_{z}W_{x}\left(t\right)+ta_{x}\right)=\lambda_{z}^{2}t$.
We condition on the event $\mathcal{S}$ such that the previous properties
hold and has probability $1$. 

Define $a=\lambda_{z}\mbox{log}\left(\frac{P_{m-1}\left(k-1\right)}{1-P_{m-1}}\right)$.
If $m=1$, then $P_{m-1}=P^{*}\in\left(0,1\right)$, and so $a$ is
finite. If $m>1$, then 
\[
P^{*}\leq P_{m-1}\leq P^{*}/\left(1-c\right)^{m-1}\leq P^{*}/\left(1-c\right)^{k-2}\leq P^{*}/\left(P^{*}\right)^{k-2/k-1}<1
\]
and so $P_{m-1}\in\left(0,1\right)$ and $a$ is finite.

We define $T^{*}=\mbox{inf}\left\{ t\geq\tau_{m}:q_{t,x}^{W}\left(A_{m-1}\right)>P_{m-1}\mbox{ for some }x\in A_{m-1}\right\} $.
Fix $n\in\mathbb{N}$, $n>2$, and define $\tau=\tau_{m}+\frac{t^{*}}{nN}$
for a deterministic $t^{*}\in\left\{ 1,\ldots,n-1\right\} $. Let
$x$ be any $\mathcal{F}_{\tau_{m}}$-measurable random variable that
is almost surely in $A_{m-1}$ and we define $\Gamma_{t,x}=W_{x}\left(t\right)+t\frac{a_{x}}{\lambda_{z}}$.
Consider the event $ $$a<\Gamma_{\tau+\frac{1}{nN},x}-\Gamma_{\tau+\frac{1}{nN},y}$
for each $y\in A_{m-1}-\left\{ x\right\} $. On this event, $q_{\tau+\frac{1}{mN},y}^{W}\left(A_{m-1}\right)/q_{\tau+\frac{1}{mN},x}^{W}\left(A_{m-1}\right)=\mbox{exp}\left(\frac{1}{\lambda_{z}}\left(\Gamma_{\tau+\frac{1}{nN},y}-\Gamma_{\tau+\frac{1}{nN},x}\right)\right)<\mbox{exp}\left(-a/\lambda_{z}\right)$
for $ $ $y\in A_{m-1}-\left\{ x\right\} $ and
\[
q_{\tau+\frac{1}{nN},x}\left(A_{m-1}\right)=\left[1+\sum_{y\in A_{m-1}-\left\{ x\right\} }q_{\tau+\frac{1}{nN},y}\left(A_{m-1}\right)/q_{\tau+\frac{1}{nN},x}\left(A_{m-1}\right)\right]^{-1}>\left[1+\left(k-1\right)\mbox{exp}\left(-a/\lambda_{z}\right)\right]^{-1}=P_{m-1}.
\]
Thus, on the event considered, $T^{*}\leq\tau+\frac{1}{nN}$. 

We now define $\tilde{x}\in\mbox{arg max}_{x\in A_{m-1}}\mbox{ }\Gamma_{\tau,x}$,
which is $\mathcal{F}_{\tau}$-measurable and is almost surely in
$A_{m-1}$. Then we have that
\begin{eqnarray*}
\mathbb{P}\left\{ T^{*}\leq\tau+\frac{1}{nN}\mid\mathcal{F}_{\tau},T^{*}>\tau\right\}  & \geq & \mathbb{P}\left\{ a<\Gamma_{\tau+\frac{1}{nN},\tilde{x}}-\Gamma_{\tau+\frac{1}{nN},x}\mbox{ \mbox{}}\forall x\in A_{m-1}-\left\{ \tilde{x}\right\} \mid\mathcal{F}_{\tau},T^{*}>\tau\right\} \\
 & \geq & \mathbb{P}\left\{ \Gamma_{\tau+\frac{1}{nN},\tilde{x}}\geq\Gamma_{\tau,\tilde{x}},\Gamma_{\tau,\tilde{x}}-\Gamma_{\tau+\frac{1}{nN},x}>a\mbox{ \mbox{}}\forall x\in A_{m-1}-\left\{ \tilde{x}\right\} \mid\mathcal{F}_{\tau},T^{*}>\tau\right\} \\
 & \geq & \mathbb{P}\left\{ \Gamma_{\tau+\frac{1}{nN},\tilde{x}}\geq\Gamma_{\tau,\tilde{x}},\Gamma_{\tau,x}-\Gamma_{\tau+\frac{1}{nN},x}>a\mbox{ \mbox{}}\forall x\in A_{m-1}-\left\{ \tilde{x}\right\} \mid\mathcal{F}_{\tau},T^{*}>\tau\right\} \\
 & = & \mathbb{P}\left\{ \Gamma_{\tau+\frac{1}{nN},\tilde{x}}\geq\Gamma_{\tau_{m},\tilde{x}}\mid\mathcal{F}_{\tau}\right\} \prod_{x\in A_{m-1}\setminus\left\{ \tilde{x}\right\} }\mathbb{P}\left\{ \Gamma_{\tau,x}-\Gamma_{\tau+\frac{1}{nN},x}>a\mid\mathcal{F}_{\tau}\right\} .
\end{eqnarray*}
Note that $\Gamma_{t,x}$ is a Brownian motion, and so $\mathbb{P}\left\{ \Gamma_{\tau+\frac{1}{nN},\tilde{x}}\geq\Gamma_{\tau_{m},\tilde{x}}\mid\mathcal{F}_{\tau_{m}}\right\} $
is the probability of a conditionally $N\left(\frac{a_{\tilde{x}}}{nN}\frac{1}{\lambda_{z}},1\right)$
random variable $\Gamma_{\tau+\frac{1}{nN},\tilde{x}}-\Gamma_{\tau,\tilde{x}}$,
exceeding $0$. This probability is $\Phi\left(\frac{a_{\tilde{x}}}{nN}\frac{1}{\lambda_{z}}\right)$,
which is bounded below by $ $$\Phi\left(\frac{\mbox{min}_{x}a_{x}}{N}\frac{x_{0}}{\lambda_{z}}\right)$
where $x_{0}=0$ if $\mbox{min}_{x}a_{x}\geq0$, and $x_{0}=\frac{1}{3}$
otherwise. Here, $\Phi$ is the normal cumulative distribution function.
Similarly, the probability $\mathbb{P}\left\{ \Gamma_{\tau,x}-\Gamma_{\tau+\frac{1}{nN},x}>a\mid\mathcal{F}_{\tau_{m}}\right\} $
is the probability of a conditionally $N\left(-a-\frac{a_{x}}{nN}\frac{1}{\lambda_{z}},1\right)$
random variable, $\Gamma_{\tau,x}-\Gamma_{\tau+\frac{1}{nN},x}-a$,
exceeding $0$. This probability is $\Phi\left(-a-\frac{a_{x}}{nN}\frac{1}{\lambda_{z}}\right)$,
and is bounded below by $\Phi\left(-a-\frac{\mbox{max}_{x}a_{x}}{N}\frac{x_{0}^{+}}{\lambda_{z}}\right)$
where $x_{0}^{+}=0$ if $\mbox{max}_{x}a_{x}\leq0$ and $x_{0}^{+}=\frac{1}{3}$
otherwise.

Thus, replacing $\tau$ with $\tau_{m}+\frac{t^{*}}{nN}$,
\[
\mathbb{P}\left\{ T^{*}\leq\tau_{m}+\frac{t^{*}}{nN}+\frac{1}{nN}\mid\mathcal{F}_{\tau_{m}+\frac{t^{*}}{nN}},T^{*}>\tau_{m}+\frac{t^{*}}{nN}\right\} \geq\Phi\left(\frac{\mbox{min}_{x}a_{x}}{N}\frac{x_{0}}{\lambda_{z}}\right)\Phi\left(-a-\frac{\mbox{max}_{x}a_{x}}{N}\frac{x_{0}^{+}}{\lambda_{z}}\right)^{k-1}
\]
Let $\epsilon$ be the quantity on the right-hand side of this inequality,
then $\epsilon<1$ and it does not depend on $t^{*}$ and $n$. 

By repeated application of this inequality, we have that $\mathbb{P}\left\{ T^{*}>T_{W}^{1}\left(P\right)+\frac{1}{N}\mid\mathcal{F}_{\tau_{m}}\right\} \leq\left(1-\epsilon\right)^{n}$
for all $n>2$. This is true because
\begin{eqnarray*}
\mathbb{P}\left\{ T^{*}>T_{W}^{1}\left(P\right)+\frac{1}{N}\mid\mathcal{F}_{\tau_{m}+\frac{n-1}{nN}},T^{*}>\tau_{m}+\frac{n-1}{nN}\right\}  & < & \left(1-\epsilon\right)\\
\Rightarrow\mathbb{P}\left\{ T^{*}>\tau_{m}+\frac{1}{N}\mid\mathcal{F}_{T_{W}^{1}\left(P\right)+\frac{n-2}{nN}},T^{*}>\tau_{m}+\frac{n-2}{nN}\right\}  & < & \left(1-\epsilon\right)\mathbb{P}\left\{ T^{*}>\tau_{m}+\frac{n-1}{nN}\mid\mathcal{F}_{\tau_{m}++\frac{n-2}{nN}},T^{*}>\tau_{m}+\frac{n-2}{nN}\right\} \\
 & \leq & \left(1-\epsilon\right)^{2}\\
 & \vdots\\
\Rightarrow\mathbb{P}\left\{ T^{*}>\tau_{m}+\frac{1}{N}\mid\mathcal{F}_{T_{W}^{1}\left(P\right)}\right\}  & \leq & \left(1-\epsilon\right)^{n}
\end{eqnarray*}
and $\left(1-\epsilon\right)^{n}$ vanishes in the limit as $n\rightarrow\infty$.
Then, $\mathbb{P}\left\{ T^{*}>\tau_{m}+\frac{1}{N}\mid\mathcal{F}_{T_{W}^{1}\left(P\right)}\right\} =0$
and then 
\[
\mathbb{P}\left\{ \tau_{m}\leq T^{*}\leq\tau_{m}+\frac{1}{N}\mid\mathcal{F}_{T_{W}^{1}\left(P\right)}\right\} =1
\]
 for all $N$. 

$\;\;\;\;\;\;\;\;\;\;\;\;\;\;\;\;\;\;\;\;\;\;\;\;\;\;\;\;\;\;\;\;\;\;\;\;\;\;\;\;\;\;\;\;\;\;\;\;\;\;\;\;\;\;\;\;\;\;\;\;\;\;\;\;\;\;\;\;\;\;\;\;\;\;\;\;\;\;\;\;\;\;\;\;\;\;\;\;\;\;\;\;\;\;\;\;\;\;\;\;\;\;\;\;\;\;\;\;\;\;\;\;\;\;\;\;\;\;\;\;\;\;\;\;\;\;\;\;\;\;\;\;\;\;\;\;\;\;\;\text{\ensuremath{\blacksquare}}$

${\color{white}sssssssssssssssssssssssssssssssssssssssssssssssssssssssssssssssssssssssssssssssssssssssssssss}$$\;\;\;\;\;\;\;\;\;\;\;\;\;\;\;\;\;\;\;\;\;\;\;\;\;\;\;\;\;\;\;\;\;\;\;\;\;\;\;\;\;\;\;\;\;\;\;\;\;\;\;\;\;\;\;\;\;\;\;\;\;\;\;\;\;\;\;\;\;\;\;\;\;\;\;\;\;\;\;\;\;\;\;\;\;\;\;\;\;\;\;\;\;\;\;\;\;\;\;\;\;\;\;\;\;\;\;\;\;\;\;\;\;\;\;\;\;\;\;\;\;\;\;\;\;\;\;\;\;\;\;\;\;\;\;\;\;\;\;\text{\ensuremath{\blacksquare}}$

The following proposition can be proved using a similar argument.


\paragraph*{Proposition 5.}

Let $W=\left(W_{1},\ldots,W_{k}\right)$ be a $k-$dimensional independent
standard Brownian motion. Let $q_{tx}^{W}\left(A\right):=\mbox{exp}\left(\frac{1}{\lambda_{z}^{2}}\left(\lambda_{z}W_{x}\left(t\right)+ta_{x}\right)\right)/\sum_{x^{'}\in A}\mbox{exp}\left(\frac{1}{\lambda_{z}^{2}}\left(\lambda_{z}W_{x^{'}}\left(t\right)+ta_{x^{'}}\right)\right)$
for all $x\in A$ and $A\subset\left\{ 1,\ldots,k\right\} $. Fix
$m\in\left\{ 1,\ldots,k-1\right\} $. We have that for all $N\in\mathbb{N}$,
there exists $t$ such that $\tau_{m}-\frac{1}{N}\leq t<\tau_{m}$
and $q_{tx}^{W}\left(A_{m-1}\right)<c$ for some $x\in A_{m-1}$ almost
surely given that $M=m$. \\


We are now ready to prove the lemma 2.


\paragraph*{Lemma 2.}

Let $\left\{ \delta_{n}\right\} \subset\left(0,\infty\right)$ such
that $\delta_{n}\rightarrow0$. If $D_{s}\equiv\{Z\in D\left[0,\infty\right)^{k}:\mbox{ if }\left\{ Z_{n}\right\} \subset D\left[0,\infty\right)^{k}\mbox{ and }$
$\mbox{lim}{}_{n}d_{\infty}\left(Z_{n},Z\right)=0$ , then the sequence
$\left\{ f\left(Z_{n},\delta_{n}\right)\right\} $ converges to $\left\{ g\left(Z\right)\right\} $$\left.\right\} $$ $,
then $ $$\mathbb{P}\left(W\text{\ensuremath{\in}}D_{s}\right)=1$.


\paragraph*{Proof of Lemma 2.}

By Lemma 4 and 6 of Frazier, $\tau_{n}<\infty$ for $n=0,1,\ldots,k-1$,
$M\leq k-1$,$\tau_{n}=\tau_{M}$ for all $n\geq M$, $\hat{x}\in\mbox{arg max}_{x\in A_{m-1}}\left(\lambda_{z}W_{x}\left(\tau_{M}\right)+a_{x}\tau_{M}\right)$
and $\tau_{M-1}<\tau_{M}$ with probability 1. Then there exists a
measurable set $\mathcal{L}$ such that $\mathbb{P}\left[\mathcal{L}\right]=1$
and the previous properties are true, $W$ is continuous and Proposition
4 and 5 hold. We will show that $\mathbb{P}\left(W\text{\ensuremath{\in}}D_{s}\mid M_{W}=i,\mathcal{L}\right)=1$
for $i\in\left\{ 1,\ldots,k-1\right\} $ so that the desired conclusion
follows. It will be useful to use the same notation than in Proposition
2: 
\begin{eqnarray*}
T_{W}^{m}\left(a\right) & = & \begin{cases}
\mbox{inf}\left\{ t\geq\tau_{m-1}:\mbox{max}_{x\in A_{m-1}}q_{tx}^{W}\left(A_{m-1}\right)\geq a\right\}  & \mbox{if }m\geq M_{W}\\
\mbox{inf}\left\{ t\geq\tau_{m-1}:\mbox{min}_{x\in A_{m-1}}q_{tx}^{W}\left(A_{m-1}\right)\leq a\right\}  & \mbox{if }m<M_{W}
\end{cases}
\end{eqnarray*}
 and 
\[
T_{n}^{m}\left(a\right):=\mbox{inf}\left\{ t\geq\tau_{m-1}:\mbox{max}_{x\in A_{m-1}}q_{tx}^{Z_{n},\delta_{n}}\left(A_{m-1}\right)\geq a\mbox{ or }\mbox{min}_{x\in A_{m-1}}q_{tx}^{Z_{n},\delta_{n}}\left(A_{m-1}\right)\leq c\right\} 
\]
 for any $a\in\mathbb{R}$ and $1\leq m\leq k-1$.


\subparagraph*{Case I.}

First, we fix $\omega\in\mathcal{L}\bigcap\left\{ M_{W}=1\right\} $.
Let $\left\{ Z_{n}\right\} \subset D\left[0,\infty\right)^{k}$ such
that $Z_{n}\rightarrow W$. We want to prove that $f\left(Z_{n},\delta_{n}\right)\rightarrow g\left(W\right)$.
Let's prove that $\tau_{1}^{Z_{n},\delta_{n}}\rightarrow\tau_{1}^{W}$
as $n\rightarrow\infty$. 

First, we are going to prove that $ $$T_{W}^{1}\left(P_{0}\right)=\mbox{lim}_{n}T_{W}^{1}\left(P_{0}+\epsilon_{n}\right)$.
Note that $ $ $\mbox{lim}_{n}T_{W}^{1}\left(P_{0}+\epsilon_{n}\right)=\mbox{inf}_{n}T_{W}^{1}\left(P_{0}+\epsilon_{n}\right)$
and so we have to prove that for all $M>0$ there exists $n$ such
that $T_{W}^{1}\left(P_{0}+\epsilon_{n}\right)<T_{W}^{1}\left(P_{0}\right)+M.$
Equivalently, we should prove that for all $N\in\mathbb{N}$ there
exists $t\in\left(T_{W}^{1}\left(P_{0}\right),T_{W}^{1}\left(P_{0}\right)+\frac{1}{N}\right]$
such that for some $x\in A_{0}$, $q_{t,x}^{W}\left(A_{0}\right)>P_{0}$.
However, this is true by proposition 4. 

By Proposition 2 and 3,
\begin{equation}
T_{W}^{1}\left(P_{0}\right)=\mbox{lim}_{n}T_{n}^{1}\left(P_{0}\right).\label{eq:10.1}
\end{equation}


By Proposition 3, we have that $f_{Z_{n}}\left(\cdot\right)\rightarrow f_{W}\left(\cdot\right)$
uniformly in $\left[0,T_{W}^{1}\left(P_{0}\right)+1\right]$ and so
it should be the case that $\mbox{max}_{x\in A_{0}}q_{T_{n}^{1}\left(P_{0}\right)x}^{Z_{n},\delta_{n}}\left(A_{0}\right)\geq P_{0}$
for $n$ large.

Recall that $\hat{x}\in\gamma:=\mbox{arg max}_{x\in A_{0}}\left(\lambda_{z}W_{x}\left(T_{W}^{1}\left(P_{0}\right)\right)+a_{x}T_{W}^{1}\left(P_{0}\right)\right)$. 

Let $\epsilon=\left(q_{T_{W}^{1}\left(P_{0}\right),\hat{x}}^{W}\left(A_{0}\right)-\mbox{arg max}_{x\in A_{0}-\gamma}q_{T_{W}^{1}\left(P_{0}\right),x}^{W}\left(A_{0}\right)\right)/4$.
Since the function $q_{\cdot,x}^{W}\left(A_{0}\right)$ is uniformly
continuous in $\left[0,T_{W}^{1}\left(P_{0}\right)+1\right]$ for
all $x\in A_{0}$, then there exists $\delta>0$ such that if $\left|t-s\right|<\delta$
and $t,s\in\left[0,T_{W}^{1}\left(P_{0}\right)+1\right]$, then 
\[
\left|q_{t,x}^{W}\left(A_{0}\right)-q_{s,x}^{W}\left(A_{0}\right)\right|<\frac{\epsilon}{2}
\]
for all $x\in A_{0}$. By the proof of Proposition 3, $q_{\cdot,x}^{Z_{n},\delta_{n}}\left(A_{0}\right)\rightarrow q_{\cdot,x}^{W}\left(A_{0}\right)$
in $\left[0,T_{W}^{1}\left(P_{0}\right)+1\right]$ with the Skorohod
topology, and then $q_{\cdot,x}^{Z_{n},\delta_{n}}\left(A_{0}\right)\rightarrow q_{\cdot,x}^{W}\left(A_{0}\right)$
uniformly in $\left[0,T_{W}^{1}\left(P_{0}\right)+1\right]$ because
$q_{\cdot,x}^{W}\left(A_{0}\right)$ is uniformly continuous in $\left[0,T_{W}^{1}\left(P_{0}\right)+1\right]$
for all $x\in A_{0}$. Consequently, there exists $N_{1}$ such that
if $n>N_{1}$, then $\left|q_{t,x}^{W}\left(A_{0}\right)-q_{t,x}^{Z_{n},\delta_{n}}\left(A_{0}\right)\right|<\frac{\epsilon}{2}$
for all $t\in\left[0,T_{W}^{1}\left(P_{0}\right)+1\right]$ and all
$x\in A_{0}$. By (\ref{eq:10.1}), there exists $N_{2}$ such that
if $n>N_{2}$, then $\left|T_{W}^{1}\left(P_{0}\right)-T_{n}^{1}\left(P_{0}\right)\right|<\delta$
and $T_{n}^{1}\left(P_{0}\right)\in\left[0,T_{W}^{1}\left(P_{0}\right)+1\right]$.
So if $n>\mbox{max}\left\{ N_{1},N_{2}\right\} $, we have that

\begin{eqnarray*}
\left|q_{T_{W}^{1}\left(P_{0}\right),x}^{W}\left(A_{0}\right)-q_{T_{n}^{1}\left(P_{0}\right),x}^{Z_{n},\delta_{n}}\left(A_{0}\right)\right| & = & \left|q_{T_{W}^{1}\left(P_{0}\right),x}^{W}\left(A_{0}\right)-q_{T_{n}^{1}\left(P_{0}\right),x}^{W}\left(A_{0}\right)\right|+\left|q_{T_{n}^{1}\left(P_{0}\right),x}^{W}\left(A_{0}\right)-q_{T_{n}^{1}\left(P_{0}\right),x}^{Z_{n},\delta_{n}}\left(A_{0}\right)\right|\\
 & < & \frac{\epsilon}{2}+\frac{\epsilon}{2}=\epsilon
\end{eqnarray*}


for all $x\in A_{0}$.

By Proposition 3, there exists a sequence $\left\{ \epsilon_{n}\right\} _{n>N^{*}}$
decreasing to zero for some $N^{*}$ such that if $n>N^{*}$ .
\[
\left|q_{T_{W}^{1}\left(P_{0}\right),\hat{x}}^{W}\left(A_{0}\right)-\mbox{max}_{x\in A_{0}}q_{T_{n}^{1}\left(P_{0}\right),x}^{Z_{n},\delta_{n}}\left(A_{0}\right)\right|<\epsilon_{n}.
\]


Let $y_{n}\in\mbox{arg max}q_{T_{n}^{1}\left(P_{0}\right),y}^{Z_{n},\delta_{n}}\left(A_{0}\right)$.
Thus if $n>\mbox{max}\left\{ N^{*},N_{1},N_{2}\right\} $, then
\begin{eqnarray*}
\left|q_{T_{W}^{1}\left(P_{0}\right),\hat{x}}^{W}\left(A_{0}\right)-q_{T_{W}^{1}\left(P_{0}\right),y_{n}}^{W}\left(A_{0}\right)\right| & \leq & \left|q_{T_{W}^{1}\left(P_{0}\right),\hat{x}}^{W}\left(A_{0}\right)-q_{T_{n}^{1}\left(P_{0}\right),y_{n}}^{Z_{n},\delta_{n}}\left(A_{0}\right)\right|+\left|-q_{T_{W}^{1}\left(P_{0}\right),y_{n}}^{W}\left(A_{0}\right)+q_{T_{n}^{1}\left(P_{0}\right),y_{n}}^{Z_{n},\delta_{n}}\left(A_{0}\right)\right|\\
 & < & \epsilon_{n}+\epsilon
\end{eqnarray*}


Take $N_{3}$ such that if $n>N_{3}$, then $\epsilon_{n}<\epsilon$.
Consequently, if $n>\mbox{max}\left\{ N_{3},N_{1},N_{2},N^{*}\right\} $,
we have that
\begin{eqnarray*}
\left|q_{T_{W}^{1}\left(P_{0}\right),\hat{x}}^{W}\left(A_{0}\right)-q_{T_{W}^{1}\left(P_{0}\right),y_{n}}^{W}\left(A_{0}\right)\right| & < & \frac{q_{T_{W}^{1}\left(P_{0}\right),\hat{x}}^{W}\left(A_{0}\right)-\mbox{arg max}_{x\in A_{0}-\gamma}q_{T_{W}^{1}\left(P_{0}\right),x}^{W}\left(A_{0}\right)}{2}
\end{eqnarray*}
if $y_{n}\neq\hat{x}$, then
\[
\left|q_{T_{W}^{1}\left(P_{0}\right),\hat{x}}^{W}\left(A_{0}\right)-q_{T_{W}^{1}\left(P_{0}\right),y_{n}}^{W}\left(A_{0}\right)\right|<\frac{q_{T_{W}^{1}\left(P_{0}\right),\hat{x}}^{W}\left(A_{0}\right)-q_{T_{W}^{1}\left(P_{0}\right),y_{n}}^{W}\left(A_{0}\right)}{2}
\]
which is a contradiction. Consequently, $y_{n}=\hat{x}$ for $n$
large. Consequently, $f\left(Z_{n},\delta_{n}\right)$ converges to
$g\left(W\right)$. 


\subparagraph*{Case II.}

Now, we fix we fix $\omega\in\mathcal{L}\bigcap\left\{ M_{W}=2\right\} $.
Similarly, we take $\left\{ Z_{n}\right\} \subset D\left[0,\infty\right)^{k}$
such that $Z_{n}\rightarrow W$. Let's prove that $\tau_{1}^{Z_{n},\delta_{n}}\rightarrow\tau_{1}^{W}$.
First, we are going to prove that $ $$T_{W}^{1}\left(c\right)=\mbox{lim}_{n}T_{W}^{1}\left(c-\epsilon_{n}\right)$.
Note that $ $ $\mbox{lim}_{n}T_{W}^{1}\left(c-\epsilon_{n}\right)=\mbox{inf}_{n}T_{W}^{1}\left(c-\epsilon_{n}\right)$
and so we have to prove that for all $M>0$ there exists $n$ such
that $T_{W}^{1}\left(c-\epsilon_{n}\right)<T_{W}^{1}\left(c\right)+M.$
Equivalently, we should prove that for all $N\in\mathbb{N}$ there
exists $t\in\left(T_{W}^{1}\left(c\right),T_{W}^{1}\left(c\right)+\frac{1}{N}\right]$
such that for some $x\in A_{0}$, $q_{t,x}^{W}\left(A_{0}\right)<c$.
However, this is true by proposition 5. By Proposition 2 and 3, $T_{W}^{1}\left(c\right)=\mbox{lim}_{n}T_{n}^{1}\left(P_{0}\right).$
Using similar arguments than the given in Case I, we can conclude
that $f\left(Z_{n},\delta_{n}\right)$ converges to $g\left(W\right)$.\\


The cases $M_{W}=i$ for $k-1\geq i\geq3$ can be proved in a similar
way. Then we conclude that $\mathbb{P}\left(W\in D_{s}\right)=1.$

$\;\;\;\;\;\;\;\;\;\;\;\;\;\;\;\;\;\;\;\;\;\;\;\;\;\;\;\;\;\;\;\;\;\;\;\;\;\;\;\;\;\;\;\;\;\;\;\;\;\;\;\;\;\;\;\;\;\;\;\;\;\;\;\;\;\;\;\;\;\;\;\;\;\;\;\;\;\;\;\;\;\;\;\;\;\;\;\;\;\;\;\;\;\;\;\;\;\;\;\;\;\;\;\;\;\;\;\;\;\;\;\;\;\;\;\;\;\;\;\;\;\;\;\;\;\;\;\;\;\;\;\;\;\;\;\;\;\;\;\text{\ensuremath{\blacksquare}}$

\[
\]


By the extension of the CMT (Theorem 5.5 of Billingsley 1968), we
have the following corollary.


\paragraph*{Corollary 2.}

We have that 

\[
f\left(C\left(\delta,t\right),\delta\right)\Rightarrow g\left(W\left(t\right)\right)
\]
in distribution as $\delta\rightarrow0$.

\[
\]



\paragraph*{Theorem 1.}

If samples from system $x\in\left\{ 1\ldots,k\right\} $ are identically
distributed and independent, over time and across alternatives, then
$\mbox{lim}_{\delta\rightarrow0}Pr\left\{ \mbox{BIZ selects }k\right\} \geq P*$
provided $\mu_{k}=a_{k}\delta,\mu_{k-1}=a_{k-1}\delta,\ldots,\mu_{1}=a_{1}\delta$,
and $\mu_{k}>\mu_{k-1}>\cdots>\mu_{1}$.

Furthermore,
\[
\mbox{inf}_{a\in PZ\left(1\right)}\mbox{lim}_{\delta\rightarrow0}\mathbb{P}\left(CS_{\delta}\right)=P^{*}
\]
where $PZ\left(1\right)=\left\{ a\in\mathbb{R}^{k}:a_{k}-a_{k-1}\geq1\right\} $. 


\paragraph*{Proof.}

By the comments given at the beginning of this section, we know that
we can work with the algorithm defined by

\[
q'_{tx}\left(A\right)=q'\left(\left(Z_{\frac{t}{\delta^{2}}x},\frac{t}{\delta^{2}}:x\in A\right),\delta,A\subset\left\{ 1,\ldots,k\right\} \right)
\]
where $t\in\delta^{2}\mathbb{N}$, instead of the algorithm defined
by
\[
q'_{tx}\left(A\right)=q'\left(\left(Z_{tx},t:x\in A\right),\delta,A\subset\left\{ 1,\ldots,k\right\} \right)
\]
where $t\in\mathbb{N}$.

Now, we define
\begin{eqnarray*}
\hat{\tau}_{n+1}\left(\delta\right) & = & \mbox{inf}\left\{ t\in\left\{ \tau_{n}\delta^{2},\left(\tau_{n}+1\right)\delta^{2},\ldots\right\} :\mbox{ min}_{x\in A_{n}}q_{t/\delta^{2},x}^{'}\left(A_{n}\right)\leq c\mbox{ or }\mbox{max}_{x\in A_{n}}q_{t/\delta^{2},x}^{'}\left(A_{n}\right)\geq P_{n}\right\} 
\end{eqnarray*}
We denote the corresponding continuous hitting times by $\left(\tau_{n}\left(\delta\right)\right)_{n}$,
which are defined as $\tau_{n+1}=\mbox{inf}\left\{ t\geq\tilde{\tau}_{n}:\mbox{ min}_{x\in A_{n}}q_{tx}\left(A_{n}\right)\leq c\mbox{ or }\mbox{max}_{x\in A_{n}}q_{tx}\left(A_{n}\right)\geq P_{n}\right\} $. 

Using that $\mathcal{C}\left(\delta,\cdot\right)$ is right-continuous,
we can prove that $\hat{\tau}_{n}\left(\delta\right)-\tau_{n}\left(\delta\right)\rightarrow0$
with probability $1$ as $\delta\rightarrow0$, and so we can use
$\mathcal{C}\left(\delta,\tau_{n}\left(\delta\right)\right)$ instead
of $\mathcal{C}\left(\delta,\hat{\tau}_{n}\left(\delta\right)\right)$
in the limit.

Let $CS_{\delta}$ be the event of doing a correct selection given
the configuration $\mu_{k}=a_{k}\delta,\mu_{k-1}=a_{k-1}\delta,\ldots,\mu_{1}=a_{1}\delta$
$ $. Then by the previous argument and the Corollary 2, 
\begin{eqnarray*}
\mbox{lim}_{\delta\rightarrow0}\mathbb{P}\left(CS_{\delta}\right) & = & \mbox{lim}_{\delta\rightarrow0}\mathbb{P}\left(f\left(\mathcal{C}\left(\delta,t\right),\delta\right)=1\right)\\
 & = & \mathbb{P}\left(g\left(W\right)=1\right)\\
 & \geq & P^{*}
\end{eqnarray*}
where the last inequality follows from the Theorem 2 of Frazier \cite{key-5}.

Furthermore, by the same theorem 2,
\[
\mbox{inf}_{a\in PZ\left(1\right)}\mathbb{P}\left(g\left(W\right)=1\right)=P^{*}
\]


where $PZ\left(1\right)=\left\{ a\in\mathbb{R}^{k}:a_{k}-a_{k-1}\geq1\right\} $. 

$\;\;\;\;\;\;\;\;\;\;\;\;\;\;\;\;\;\;\;\;\;\;\;\;\;\;\;\;\;\;\;\;\;\;\;\;\;\;\;\;\;\;\;\;\;\;\;\;\;\;\;\;\;\;\;\;\;\;\;\;\;\;\;\;\;\;\;\;\;\;\;\;\;\;\;\;\;\;\;\;\;\;\;\;\;\;\;\;\;\;\;\;\;\;\;\;\;\;\;\;\;\;\;\;\;\;\;\;\;\;\;\;\;\;\;\;\;\;\;\;\;\;\;\;\;\;\;\;\;\;\;\;\;\;\;\;\;\;\;\;\text{\ensuremath{\blacksquare}\;\;\;\;}$


\section{Asymptotic Validity when the Variances are Unknown}

We use a random change of time to prove that the new $\hat{\mathcal{C}}_{x}$
defined using the sample variances also converges to a Brownian motion
in the sense of $D_{\infty}$. 


\paragraph*{Lemma 3.}

We have that
\[
\hat{\mathcal{C}}_{x}\left(\delta,\cdot\right):=\frac{Y_{\mbox{ceil}\left(\frac{\hat{\lambda}_{t/\delta^{2},x}^{2}}{\hat{\lambda}_{t/\delta^{2},z}^{2}}\left(n_{0}+\cdot\frac{1}{\delta^{2}}\right)\right),x}-\frac{\hat{\lambda}_{t/\delta^{2},x}^{2}}{\hat{\lambda}_{t/\delta^{2},z}^{2}}\left(n_{0}+\cdot\frac{1}{\delta^{2}}\right)\mu_{x}}{\frac{\hat{\lambda}_{t/\delta^{2},x}^{2}}{\hat{\lambda}_{t/\delta^{2},z}}\frac{1}{\delta}}\Rightarrow W_{x}\left(\cdot\right)
\]


in the sense of $D_{\infty}$ for each $x\in A$, where $\hat{\lambda}_{t/\delta^{2},x}^{2}=\frac{1}{n_{t/\delta^{2},x}-1}\sum_{i=1}^{n_{t/\delta^{2},x}}\left(X_{xi}-Y_{n_{t/\delta^{2},x}}\right)^{2}$.


\paragraph*{Proof. }

Fix $x\in A$. Define $\Psi_{\delta}:\left[0,\infty\right)\rightarrow\left[0,\infty\right)$
by $\Psi_{\delta}\left(t\right)=-n_{0}\delta^{2}+\frac{\lambda_{z}^{2}}{\lambda_{x}^{2}}\frac{\hat{\lambda}_{t/\delta^{2},x}^{2}}{\hat{\lambda}_{t/\delta^{2},z}^{2}}\left(t+\delta^{2}n_{0}\right)$,
then $\Psi_{\delta}\in D_{\infty}$. Now define $\varphi:D_{\infty}\times D_{\infty}\rightarrow D_{\infty}$
by
\[
\varphi\left(\mathcal{X},\mathcal{Y}\right)=\mathcal{X}\circ\mathcal{Y}
\]


Using Lemma 1, we have that $\left(\mathcal{C}_{x}\left(\delta,\cdot\right),\Psi_{\delta}\left(\cdot\right)\right)\Rightarrow\left(W_{x}\left(\cdot\right),I\left(\cdot\right)\right)$
as $\delta\rightarrow0$ in the sense of $D_{\infty}$. We can prove
that $\varphi$ is measurable, and since $W_{x}$ is continuous almost
surely, we can use a generalization of the Lemma from the chapter
Random Change of Time of Billingsley (1999) to conclude that $\varphi\left(\mathcal{C}_{x}\left(\delta,\cdot\right),\Psi_{\delta}\left(\cdot\right)\right)\Rightarrow\varphi\left(W_{x}\left(\cdot\right),I\left(\cdot\right)\right)=W_{x}\left(\cdot\right)$
in the sense of $D_{\infty}$ as we wanted to prove.

$\;\;\;\;\;\;\;\;\;\;\;\;\;\;\;\;\;\;\;\;\;\;\;\;\;\;\;\;\;\;\;\;\;\;\;\;\;\;\;\;\;\;\;\;\;\;\;\;\;\;\;\;\;\;\;\;\;\;\;\;\;\;\;\;\;\;\;\;\;\;\;\;\;\;\;\;\;\;\;\;\;\;\;\;\;\;\;\;\;\;\;\;\;\;\;\;\;\;\;\;\;\;\;\;\;\;\;\;\;\;\;\;\;\;\;\;\;\;\;\;\;\;\;\;\;\;\;\;\;\;\;\;\;\;\;\;\;\;\;\;\text{\ensuremath{\blacksquare}\;\;\;\;}$\\


In the previous proof, we only have to replace $\lambda_{x}^{2}$
by its estimators and we get the same result. We should note that
\[
\mbox{ceil}\left(\frac{\hat{\lambda}_{t/\delta^{2},x}^{2}}{\hat{\lambda}_{t/\delta^{2},z}^{2}}\left(n_{0}+\cdot\frac{1}{\delta^{2}}\right)\right)-n_{t/\delta^{2},x}\rightarrow0
\]
for any $x$. Consequently, we will have that
\[
\frac{Y_{n_{t/\delta^{2},x},x}-n_{t/\delta^{2},x}\mu_{x}}{\frac{\hat{\lambda}_{t/\delta^{2},x}^{2}}{\hat{\lambda}_{t/\delta^{2},z}}\frac{1}{\delta}}\Rightarrow W_{x}\left(\cdot\right)
\]


So, we have the following theorem.


\paragraph*{Theorem 2.}

If samples from system $x\in\left\{ 1\ldots,k\right\} $ are identically
distributed and independent, over time and across alternatives, then
$\mbox{lim}_{\delta\rightarrow0}Pr\left\{ \mbox{BIZ selects }k\right\} \geq P*$
provided $\mu_{k}=a_{k}\delta,\mu_{k-1}=a_{k-1}\delta,\ldots,\mu_{1}=a_{1}\delta$,
and $\mu_{k}>\mu_{k-1}>\cdots>\mu_{1}$.

Furthermore,
\[
\mbox{inf}_{a\in PZ\left(1\right)}\mbox{lim}_{\delta\rightarrow0}\mathbb{P}\left(CS_{\delta}\right)=P^{*}
\]
where $PZ\left(1\right)=\left\{ a\in\mathbb{R}^{k}:a_{k}-a_{k-1}\geq1\right\} $. 


\section{Probability of Good Selection}

Ideally, we would like that the difference between the chosen system
and the best system is practically insignificant. We thus define the
probability of good selection as
\[
\mbox{PGS}\left(\mu\right)=\mathbb{P}\left(\mu_{k}-\mu_{\hat{x}}\leq\delta\right).
\]


We would like to prove the PGS guarantee:
\[
\forall\mu,\mbox{ PGS}\left(\mu\right)\geq P^{*}.
\]
First, observe that if $\mu_{k}-\mu_{k-1}>\delta$, then $\mbox{PGS}\left(\mu\right)=\mbox{PCS}\left(\mu\right)$
and so $\mbox{PGS}\left(\mu\right)\geq P^{*}$. Consequently, we only
need to prove that if $\mu_{k}-\mu_{k-1}\leq\delta$, then we have
that $\mbox{PGS}\left(\mu\right)\geq P^{*}$. 

Inspired on Frazier (2014), we first construct a probability measure
$Q$ as follows. Let $X^{*}$ be chosen uniformly at random from among
$1,\ldots,k$ and let $\theta_{X^{*}}=\delta$ and $\theta_{x}=0$
if $x\neq X^{*}$. We then define a family of probability measures
$Q_{u}$ that i


\section{Simulations}


\section{Conclusion}

We have proved the asymptotic validity of the Bayes-inspired Zone
procedure (Frazier 2014) when the variances are known, which is a
new sequential elimination procedure. This algorithm is relevant because
it takes fewer samples than other IZ procedures, especially for problems
with large numbers of alternatives. Even though this proof does not
guarantee that the algorithm will work for any sample, we know that
it will work if the alternatives are not very different, which are
the most difficult cases. 


\section*{Appendix A: Skorohod topology}

We are going to define the Skorohod topology on $D[0,\infty)$ by
defining a metric on the space. The Skorohod metric $d_{t}$ on $D\left[0,t\right]$
for each $t\geq0$ is:
\[
d_{t}\left(\mathcal{X},\mathcal{Y}\right)=\mbox{inf}_{\lambda\in\Lambda_{t}}\left\{ \left\Vert \lambda-I\right\Vert \vee\left\Vert \mathcal{X}-\mathcal{Y}\circ\lambda\right\Vert \right\} 
\]
where $\Lambda_{t}$ is the set of strictly increasing, continuous
mappings of $\left[0,t\right]$ onto itself, and $\left\Vert \cdot\right\Vert $
is the uniform norm, and $I$ is the identity map. Note that uniform
convergence on $\left[0,t\right]$ implies Skorohod convergence.

We define the Skorohod topology on $D\left[0,\infty\right)$. For
$\mathcal{X}\in D\left[0,\text{\ensuremath{\infty}}\right)$, let
$\mathcal{X}^{m}$ be the element of $D_{\infty}:=D[0,\infty)$ defined
by
\[
\mathcal{X}^{m}\left(t\right)=g_{m}\left(t\right)\mathcal{X}\left(t\right)
\]
where 
\[
g_{m}\left(t\right)=\begin{cases}
1 & \mbox{if }t\leq m-1,\\
m-t & \mbox{if }m-1\leq t\leq m,\\
0 & \mbox{if }t\geq m.
\end{cases}
\]
And now take
\[
d_{\infty}\left(\mathcal{X},\mathcal{Y}\right)=\sum_{m=1}^{\infty}2^{-m}\left(1\wedge d_{m}\left(\mathcal{X}^{m},\mathcal{Y}^{m}\right)\right)
\]
which is the Skorohod metric on $D\left[0,\infty\right)$. By Theorem
16.2 of Billingsley1999, there is convergence $d_{\infty}\left(x_{n},x\right)\rightarrow0$
in $D_{\infty}$ if and only if $d_{t}\left(x_{n},x\right)\rightarrow0$
for each continuity point $t$ of $x$. 

We can also define

\[
d_{\infty}^{\circ}\left(\mathcal{X},Y\right)=\sum_{m=1}^{\infty}2^{-m}\left(1\wedge d_{m}^{\circ}\left(\mathcal{X}^{m},\mathcal{Y}^{m}\right)\right)
\]
where $d_{m}^{\circ}\left(\mathcal{X}^{m},\mathcal{Y}^{m}\right)=\mbox{inf}_{\lambda\in\Lambda_{m}}\left\{ \mbox{sup}_{s<t}\left|\mbox{log}\frac{\lambda t-\lambda s}{t-s}\right|\vee\left\Vert \mathcal{X}-\mathcal{Y}\circ\lambda\right\Vert \right\} $.
This is also a metric and the proof is in Bilingsley 1999 and these
two metrics are equivalent.


\subparagraph*{Theorem A.1}

We have that $d_{\infty}\left(\mathcal{X}_{n},\mathcal{Y}_{n}\right)\rightarrow0$
if and only if $d_{\infty}^{\circ}\left(\mathcal{X}_{n},\mathcal{Y}_{n}\right)\rightarrow0$
and $d_{t}\left(\mathcal{X}_{n},\mathcal{Y}_{n}\right)\rightarrow0$
if and only if $d_{t}^{\circ}\left(\mathcal{X}_{n},\mathcal{Y}_{n}\right)\rightarrow0$
for all $t\geq0$.


\subparagraph*{Theorem A.2}

Suppose we have two sequences of random paths $\left\{ \mathcal{X}_{n}=\left(X_{n}\left(t\right):0\leq t<\infty\right)\right\} _{n\geq0}^{\infty}$,$\left\{ \mathcal{Y}_{n}=\left(Y_{n}\left(t\right):0\leq t<\infty\right)\right\} _{n\geq1}^{\infty}$
such that $\mathcal{X}_{n},\mathcal{Y}_{n}:\mathcal{F}\rightarrow D_{\infty}$
for all $n\geq0$ where $\left(\Omega,\mathcal{F},\mathbb{P}\right)$
is our space of probability. Suppose that $\mathcal{X}_{n}\Rightarrow\mathcal{X}_{0}$
in the sense of $D\left[0,\infty\right)$ and $\mathcal{X}_{n}-\mathcal{Y}_{n}\rightarrow0$
uniformly in $\left[0,m\right]$ for all $m\in\mathbb{N}$ with probability
1, then $\mathcal{Y}_{n}\Rightarrow\mathcal{X}_{0}$.


\subparagraph*{Proof.}

We only need to prove that $d_{\infty}\left(\mathcal{X}_{n},\mathcal{Y}_{n}\right)\rightarrow0$
almost surely and the result will follow by Theorem 3.1 of Billingsley
1999. Then we only need to prove that $d_{m}\left(\mathcal{Y}_{n}^{m},\mathcal{X}_{n}^{m}\right)\rightarrow0$
almost surely for all $m\in\mathbb{N}$. Fix $\omega\in\Omega$ and
$m\in\mathbb{N}$ such that $\mathcal{X}_{n}-\mathcal{Y}_{n}\rightarrow0$
uniformly in $\left[0,m\right]$. Observe that 
\begin{eqnarray*}
d_{m}\left(\mathcal{Y}_{n}^{m},\mathcal{X}_{n}^{m}\right) & = & \mbox{inf}_{\lambda\in\Lambda_{m}}\left\{ \left\Vert \lambda-I\right\Vert \vee\left\Vert \mathcal{Y}_{n}^{m}-\mathcal{X}_{n}^{m}\circ\lambda\right\Vert \right\} \\
 & \leq & \left\Vert \mathcal{Y}_{n}^{m}-\mathcal{X}_{n}^{m}\right\Vert \\
 & = & \left\Vert g_{m}\mathcal{Y}_{n}-g_{m}\mathcal{X}_{n}\right\Vert \\
 & = & \mbox{sup}_{t\leq m-1}\left\Vert Y_{n}\left(t\right)-X_{n}\left(t\right)\right\Vert \lor\mbox{sup}_{t>m-1}\left\Vert Y_{n}\left(t\right)-X_{n}\left(t\right)\right\Vert \left(m-t\right)\\
 & \leq & \mbox{sup}_{t\leq m}\left\Vert Y_{n}\left(t\right)-X_{n}\left(t\right)\right\Vert \\
 & \rightarrow & 0
\end{eqnarray*}
as we wanted to prove.

$\;\;\;\;\;\;\;\;\;\;\;\;\;\;\;\;\;\;\;\;\;\;\;\;\;\;\;\;\;\;\;\;\;\;\;\;\;\;\;\;\;\;\;\;\;\;\;\;\;\;\;\;\;\;\;\;\;\;\;\;\;\;\;\;\;\;\;\;\;\;\;\;\;\;\;\;\;\;\;\;\;\;\;\;\;\;\;\;\;\;\;\;\;\;\;\;\;\;\;\;\;\;\;\;\;\;\;\;\;\;\;\;\;\;\;\;\;\;\;\;\;\;\;\;\;\;\;\;\;\;\;\;\;\;\;\;\;\;\;\text{\ensuremath{\blacksquare}}$

${\color{white}sssssssssssssssssssssssssssssssssssssssssssssssssssssssssssssssssssssssssssssssssssssssssssss}$$\;\;\;\;\;\;\;\;\;\;\;\;\;\;\;\;\;\;\;\;\;\;\;\;\;\;\;\;\;\;\;\;\;\;\;\;\;\;\;\;\;\;\;\;\;\;\;\;\;\;\;\;\;\;\;\;\;\;\;\;\;\;\;\;\;\;\;\;\;\;\;\;\;\;\;\;\;\;\;\;\;\;\;\;\;\;\;\;\;\;\;\;\;\;\;\;\;\;\;\;\;\;\;\;\;\;\;\;\;\;\;\;\;\;\;\;\;\;\;\;\;\;\;\;\;\;\;\;\;\;\;\;\;\;\;\;\;\;\;\text{\ensuremath{\blacksquare}}$


\section*{Appendix B: The BIZ Procedure}


\subsection*{B.1 The BIZ Procedure with Known Common Variance}

We first define the Bayes-inspired indifference zone (BIZ) procedure
for the case of known common variance, when $\lambda_{x}^{2}=\sigma^{2}<\infty$
for all $x$. Frazier (2014) showed that this procedure satisfies
the IZ guarantee when the systems follow the normal distribution,
with tight bounds on worst-case preference-zone in continuous time.

BIZ depends on $P^{*}\in\left(1/k,1\right)$ and $\delta>0$ for which
we desire an IZ guarantee, and a parameter $c$ such that
\begin{eqnarray*}
c & \in & \left[0,1-\left(P^{*}\right)^{\frac{1}{k-1}}\right]\mbox{ if }k>2\\
c & = & 0\mbox{ if }k=2
\end{eqnarray*}
The parameter $c$controls how aggressively the alternatives are eliminated.
It is recommended to choose $c$ equal to $1-\left(P^{*}\right)^{\frac{1}{k-1}}$. 

For each $t$, $x\in\left\{ 1,\ldots,k\right\} $, and subset $A\subset\left\{ 1,\ldots,k\right\} $,
we define a function
\[
\hat{q}_{tx}\left(A\right)=\mbox{exp}\left(\frac{\delta}{\sigma^{2}}Y_{tx}\right)\left/\sum_{x'\in A}\mbox{exp}\left(\frac{\delta}{\sigma^{2}}Y_{tx'}\right)\right.
\]
where $Y_{tx}$ is the sum of the samples observed from alternative
$x$ by time $t$.

Then, the BIZ procedure for known common variance is defined by Alg.
1.

   
\paragraph{Algorithm 1: BIZ for known common sampling variance, in discrete time.}    
\begin{algorithmic}[1]   
\label{alg:hetero-BIZ}   
\REQUIRE $c \in [0,\cmax]$, $\delta>0$, $P^*\in(1/k,1)$, common sampling variance $\sigma^2>0$.     
\STATE Let $A \leftarrow \{ 1,\ldots, k\}$, $\upthresh \leftarrow P^*$, $Y_{0x} \leftarrow 0$ for each $x$.
\WHILE{$\mbox{max}_{x\in A} \hat{q}_{tx}\left(A\right)<P$}
\WHILE{$\mbox{min}_{x\in A} \hat{q}_{tx}\left(A\right) \le c$}
 \STATE Let $x\in\mbox{arg min}_{x\in A} \hat{q}_{tx}\left(A\right)$.
    \STATE Let $\upthresh \leftarrow \upthresh/(1-\hat{q}_{tx}\left(A\right))$.     
\STATE Remove $x$ from $A$.
\ENDWHILE
  \STATE Sample from each $x \in A$ and add this sample to $Y_{tx}$ to obtain $Y_{t+1,x}$. Then increment $t$.        
 \ENDWHILE
  \STATE Select $\xhat \in\mbox{arg max}_{x\in A} Y_{tx}$ as our estimate of the best.

   
\end{algorithmic}   



\subsection*{B.2 The BIZ Procedure with Heterogeneous and Unknown Sampling Variances}

Frazier (2014) showed that this procedure retains the IZ guarantee
when the systems follow the normal distribution, and the variances
are known and are integer multiples of a common value. The continuous
time version of this procedure also satisfies the IZ guarantee, with
a tight worst-case preference-zone PCS bound.

The discrete-time BIZ procedure for unknown and/or heterogeneous sampling
variances is given in Alg. 2. It takes a variable number of samples
from each alternative, and $n_{tx}$ is this number. This algorithm
depends on a collection of integers $B_{1},\ldots,B_{k}$, $P^{*},c,\delta$
and $n_{0}$. $n_{0}$ is the number of samples to use in the first
stage of samples, and $100$ is the recommended value for $n_{0}$.
$B_{x}$ controls the number of samples taken from system $x$ in
each stage. 

For each $t$, $x\in\left\{ 1,\ldots,k\right\} $, and subset $A\subset\left\{ 1,\ldots,k\right\} $,
we define a function
\[
q_{tx}\left(A\right)=\mbox{exp}\left(\delta\beta_{t}\frac{Z_{tx}}{n_{tx}}\right)\left/\sum_{x'\in A}\mbox{exp}\left(\delta\beta_{t}\frac{Z_{tx'}}{n_{tx'}}\right),\right.\mbox{ }\beta_{t}=\frac{\sum_{x'\in A}n_{tx'}}{\sum_{x'\in A}\hat{\lambda}_{tx'}^{2}}
\]
where $\hat{\lambda}_{tx'}^{2}$ is the sample variance of all samples
from alternative $x$ thus far and $Z_{tx}=Y_{n_{tx},x}$.

   
\paragraph{Algorithm 2: Discrete-time implementation of BIZ, for unknown and/or heterogeneous variances.}    
\begin{algorithmic}[1]   
\label{alg:hetero-BIZ}   
\REQUIRE $c \in [0,\cmax]$, $\delta>0$, $P^*\in(1/k,1)$, $n_0\ge0$ an integer, $B_1,\ldots,B_k$ strictly positive integers.  Recommended choices are $c=\cmax$, $B_1=\cdots=B_k=1$ and $n_0$ between $10$ and $30$.     If the sampling variances $\lambda^2_x$ are known, replace the estimators     
$\lambdahat^2_{tx}$ with the true values $\lambda^2_x$, and set $n_0=0$.     

\STATE For each $x$, sample alternative $x$ $n_0$ times and set $n_{0x} \leftarrow n_0$.     
Let $W_{0x}$ and $\lambdahat^2_{0x}$ be the sample mean and sample variance respectively of these samples.     Let $t\leftarrow 0$.     
\STATE Let $A \leftarrow \{ 1,\ldots, k\}$, $\upthresh \leftarrow P^*$.
\WHILE{$x\in\mbox{max}_{x\in A} q_{tx}\left(A\right)<P$}
\WHILE{$\mbox{min}_{x\in A} q_{tx}\left(A\right) \le c$}
 \STATE Let $x\in\mbox{arg min}_{x\in A} q_{tx}\left(A\right)$.
    \STATE Let $\upthresh \leftarrow \upthresh/(1-q_{tx}\left(A\right))$.     
\STATE Remove $x$ from $A$.
\ENDWHILE
  \STATE Let $z \in \mbox{arg min}_{x\in A} n_{tx} / \lambdahat^2_{tx}$.     
\STATE For each $x\in A$, let      $n_{t+1,x} = \ceil\left( \lambdahat^2_{tx} (n_{tz} + B_z) / \lambdahat^2_{tz} \right)$.     \STATE For each $x\in A$, if $n_{t+1,x}>n_{tx}$, take $n_{t+1,x}-n_{tx}$ additional samples from alternative $x$.  Let $W_{t+1,x}$ and $\lambdahat^2_{t+1,x}$ be the sample mean and sample variance respectively of all samples from alternative $x$ thus far.    
\STATE Increment $t$.
 \ENDWHILE
  \STATE Select $\xhat \in\mbox{arg max}_{x\in A} W_{tx} / n_{tx}$ as our estimate of the best.

   
\end{algorithmic}   



\subsection*{B.3 Generalization to Continuous Time}

We let $\left(Y_{tx}:t\in\mathbb{R}_{+}\right)$ be a Brownian motion
under $\mathbb{P}_{\mu,\lambda}$ starting from $0$, with drift $\mu_{x}$,
volatility $\lambda_{x}$ and independence across $x$.


\subsubsection*{B.3.1 Generalization to Continuous Time: The BIZ Procedure with Common
Variance}

In this section, we assume $\lambda_{x}^{2}=\sigma^{2}$ for all $x$,
with $\sigma^{2}$ known. To define this procedure, we recursively
define the stopping times $0=\tau_{1}\leq\tau_{2}\leq\cdots\leq\tau_{k-1}\leq\infty$,
random variables $Z_{1},\ldots,Z_{k-1}$, and $P_{0},\ldots,P_{k-1}$,
and random sets $A_{0},A_{1},\ldots,A_{k-1}$. We first define $\tau_{0},A_{0}$
and $P_{0}$ as 
\[
\tau_{0}=0,\mbox{ }A_{0}=\left\{ 1,\ldots,k\right\} ,\mbox{ },P_{0}=P^{*}.
\]
Then, for each $n=0,1,\ldots,k-2$, we define
\begin{eqnarray*}
\tau_{n+1} & = & \mbox{inf }\left\{ t\in\mathbb{R}_{+}\bigcap\left[\tau_{n},\infty\right):\mbox{min}_{x\in A_{n}}\hat{q}_{tx}\left(A_{n}\right)\leq c\mbox{ or}\mbox{max}_{x\in A_{n}}\hat{q}_{tx}\left(A_{n}\right)\geq P\right\} \\
Z_{n+1} & \in & \mbox{arg min}_{x\in A_{n}}\hat{q}_{\tau_{n+1},x}\left(A_{n}\right)\\
A_{n+1} & = & A_{n}\left\backslash \left\{ Z_{n+1}\right\} \right.\\
P_{n+1} & = & P_{n}\left/\left(1-\mbox{min}_{x\in A_{n}}\hat{q}_{\tau_{n+1},x}\left(A_{n}\right)\right)\right.
\end{eqnarray*}
Finally, $\hat{x}$ is the single alternative in $A_{k-1}$.

We also define
\[
M=\mbox{inf}\left\{ n=1,\ldots,k-1:\mbox{max}_{x\in A_{n-1}}\hat{q}_{\tau_{n},x}\left(A_{n-1}\right)\geq P_{n-1}\right\} .
\]


Frazier (2014) showed that $\tau_{M}=\tau_{M+1}=\cdots=\tau_{k-1}$,
and that the one alternative remaining in $A_{k-1}$ is the one whose
$\hat{q}_{tx}\left(A_{n}\right)$ satisfied $\hat{q}_{tx}\left(A_{n}\right)\geq P_{n}$
at time $\tau_{M}$. He also showed that $\tau_{M}<\infty$ almost
surely under any $\mathbb{P}_{\mu,\sigma}$. 


\subsubsection*{B.3.2 Generalization to Continuous Time: The BIZ Procedure for the
Heterogeneous Variance Setting}

This procedure is similar than the one given in 3.4. For each $x$
let $n_{x}\left(t\right)=\gamma\lambda_{x}^{2}t$. Now, we define
a stochastic process $\left(Y'_{tx}:t\geq0\right)$ as $Y'_{tx}=Y_{n_{x}\left(t\right),x}/\gamma\lambda_{x}^{2}$.
We have that $\left(Y'_{tx}:t\geq0\right)$ is a Brownian motion with
drift $\mu_{x}$ and volatility $1/\gamma$.

The procedure is defined by first defining $\tau_{0},A_{0}$ and $P_{0}$
as 
\[
\tau_{0}=0,\mbox{ }A_{0}=\left\{ 1,\ldots,k\right\} ,\mbox{ },P_{0}=P^{*},
\]
then defining recursively, for $n=0,1,\ldots,k-2$, 

\begin{eqnarray*}
\tau_{n+1} & = & \mbox{inf }\left\{ t\in\mathbb{R}_{+}\bigcap\left[\tau_{n},\infty\right):\mbox{min}_{x\in A_{n}}q'_{tx}\left(A_{n}\right)\leq c\mbox{ or}\mbox{max}_{x\in A_{n}}q'_{tx}\left(A_{n}\right)\geq P\right\} \\
Z_{n+1} & \in & \mbox{arg min}_{x\in A_{n}}q'_{\tau_{n+1},x}\left(A_{n}\right)\\
A_{n+1} & = & A_{n}\left\backslash \left\{ Z_{n+1}\right\} \right.\\
P_{n+1} & = & P_{n}\left/\left(1-\mbox{min}_{x\in A_{n}}q'_{\tau_{n+1},x}\left(A_{n}\right)\right)\right.
\end{eqnarray*}
where $q'_{tx}\left(A\right)$ is defined by
\[
q'_{tx}\left(A\right)=\mbox{exp}\left(\gamma\delta Y'_{tx}\right)\left/\sum_{x'\in A}\mbox{exp}\left(\gamma\delta Y'_{tx'}\right)\right.=\mbox{exp}\left(\frac{\delta}{\lambda_{x}^{2}}Y_{n_{x}\left(t\right),x}\right)\left/\sum_{x'\in A}\mbox{exp}\left(\frac{\delta}{\lambda_{x'}^{2}}Y_{n_{x'}\left(t\right),x'}\right)\right..
\]
Frazier (2014) showed that this procedure satisfies the IZ guarantee,
and and its worst-case preference-zone PCS bound is tight in continuous
time.

\[
\]

\begin{thebibliography}{References}
\bibitem{key-3}Billingsley, P. 1968. \textit{Convergence of Probability
Measures. }John Wiley and Sons. New York. 

\bibitem{key-8}Billingsley, P. 1999. \textit{Convergence of Probability
Measures, }second edition. .John Wiley and Sons. New York. 

\bibitem{key-5}Frazier, P. I. A Fully Sequential Elimination Procedure
for Indifference-Zone Ranking and Selection with Tight Bounds on Probability
of Correct Selection. \textit{Operations Research,} to appear.

\bibitem{key-9}Karatzas, I., Shreve, S. E. 1991. \textit{Brownian
Motion and Stochastic Calculus}, second edition. Springer. New York.

\end{thebibliography}
\[
\]
\[
\]

\end{document}
