%% LyX 2.0.0 created this file.  For more info, see http://www.lyx.org/.
%% Do not edit unless you really know what you are doing.
\documentclass[11pt,english]{article}
\usepackage[T1]{fontenc}
\usepackage{geometry}
\geometry{verbose,tmargin=1.5cm,bmargin=1.5cm,lmargin=1.5cm,rmargin=1.5cm}
\usepackage{color}
\usepackage{amsmath}
\usepackage{amssymb}
\usepackage{setspace}
\onehalfspacing

\makeatletter
%%%%%%%%%%%%%%%%%%%%%%%%%%%%%% User specified LaTeX commands.
%\documentclass[opre,blindrev]{informs3} % current default for manuscript submission

% current default line spacing
%%\OneAndAHalfSpacedXII 
%%\DoubleSpacedXII
%%\DoubleSpacedXI

% If hyperref is used, dvi-to-ps driver of choice must be declared as
%   an additional option to the \documentclass. For example
%\documentclass[dvips,opre]{informs3}      % if dvips is used 
%\documentclass[dvipsone,opre]{informs3}   % if dvipsone is used, etc. 

%%% OPRE uses endnotes
\usepackage{endnotes}
\let\footnote=\endnote
\let\enotesize=\normalsize
\def\notesname{Endnotes}%
\def\makeenmark{\hbox to1.275em{\theenmark.\enskip\hss}}
\def\enoteformat{\rightskip0pt\leftskip0pt\parindent=1.275em
  \leavevmode\llap{\makeenmark}}


% Private macros here (check that there is no clash with the style)
\newcommand{\lambdahat}{\widehat{\lambda}}
\newcommand{\xtilde}{\tilde{x}}
\newcommand{\qhat}{\widehat{q}}
\newcommand{\as}{\ a.s.}
\newcommand{\zap}[1]{}
\newcommand{\Ncal}{\mathcal{N}}
\newcommand{\sigmahat}{\hat{\sigma}}
\newcommand{\Nat}{\mathbb{N}}
\renewcommand{\P}{\mathbb{P}}
\newcommand{\Pb}[2]{\mathbb{P}_{#1}\left\{#2\right\}}
\newcommand{\Ybar}{\overline{Y}}
\newcommand{\Z}{\mathbb{Z}}
\newcommand{\Q}{\mathbb{Q}}
\newcommand{\R}{\mathbb{R}}
% \newcommand{\argmax}{\operatornamewithlimits{arg\,max}}
% \newcommand{\argmin}{\operatornamewithlimits{arg\,min}}
\newcommand{\g}{\,\vert\,}
\newcommand{\ind}[1]{1_{\left\{#1\right\}}}
\newcommand{\Fcal}{\mathcal{F}}
\newcommand{\Gcal}{\mathcal{G}}
\newcommand{\E}{\mathbb{E}}
\newcommand{\xhat}{\hat{x}}
\newcommand{\xwig}{\tilde{x}}
\newcommand{\PZ}{\mbox{PZ}}
\newcommand{\PCS}{\mbox{PCS}}
\newcommand{\PGS}{\mbox{PGS}}
\newcommand{\uLFC}{\underline{u}}
\newcommand{\e}[1]{\left\{ #1 \right\}}
\newcommand{\T}{\mathbb{T}} % Time index set.
\newcommand{\CS}{\mbox{CS}}
\newcommand{\Ft}{\mathcal{F}_t}
\newcommand{\F}[1]{\mathcal{F}_{#1}}
\newcommand{\Ftau}{\mathcal{F}_\tau}
\newcommand{\Rcal}{\mathcal{R}}
\newcommand{\upthresh}{P}
\newcommand{\xstar}{X^*}
\newcommand{\cmax}{1-(P^*)^{\frac1{k-1}}}
\newcommand{\NewN}{M}
\newcommand{\thetavec}{\vec{\theta}}
\newcommand{\uvec}{\vec{u}}
\newcommand{\sigmavec}{\lambda}
\newcommand{\sigmacom}{\sigma}
\newcommand{\sigmascal}{\lambda}
\newcommand{\PstarB}{\mathcal{P}^*_B}
\newcommand{\Ymod}{Y'}
\newcommand{\ceil}{\mathrm{ceil}}
% How to state the common variance assumption at the beginning of lemmas, propositions and theorems
\newcommand{\homog}{Suppose $\lambda^2_x = \sigma^2>0\ \forall x$.  }
\newcommand{\algref}[1]{Alg.~\ref{#1}}


\usepackage{algorithm,algorithmic,enumerate}




%% Setup of theorem styles. Outcomment only one. 
%% Preferred default is the first option.
  % Preferred (Theorem 1, Lemma 1, Theorem 2)
%\TheoremsNumberedByChapter  % (Theorem 1.1, Lema 1.1, Theorem 1.2)

\makeatother

\usepackage{babel}
\usepackage{xunicode}
\begin{document}

\title{On the Asymptotic Validity of a Fully Sequential Elimination Procedure
for Indifference-Zone Ranking and Selection with Tight Bounds on Probability
of Correct Selection}

\maketitle
We prove the validity of the sequential elimination IZ procedure proposed
by Frazier \cite{key-5} when $\delta$ goes to $0$. Specifically,
we analyze Algorithm 2, when $B_{1}=\cdots=B_{k}=1$:

   
\paragraph{Algorithm 2: Discrete-time implementation of BIZ, for unknown and/or heterogeneous variances.}    
\begin{algorithmic}[1]   
\label{alg:hetero-BIZ}   
\REQUIRE $c \in [0,\cmax]$, $\delta>0$, $P^*\in(1/k,1)$, $n_0\ge0$ an integer, $B_1,\ldots,B_k$ strictly positive integers.  Recommended choices are $c=\cmax$, $B_1=\cdots=B_k=1$ and $n_0$ between $10$ and $30$.     If the sampling variances $\lambda^2_x$ are known, replace the estimators     
$\lambdahat^2_{tx}$ with the true values $\lambda^2_x$, and set $n_0=0$.     
To compute $\qhat_{tx}(A)$, use \begin{equation}   q'_{t,x}(A) =    \exp\left(\gamma \delta Y'_{tx}\right) \bigg/ \sum_{x'\in A} \exp\left(\gamma \delta Y'_{tx'}\right)   = \exp\left(\frac{\delta}{\lambda^2_x} Y_{n_x(t),x}\right) \bigg/ \sum_{x'\in A} \exp\left(\frac{\delta}{\lambda^2_{x'}} Y_{n_x'(t),x'}\right), \end{equation}   
where $Y_{n_x(t),x}$ is the sum of the first $n_x(t)$ samples.
 

\STATE For each $x$, sample alternative $x$ $n_0$ times and set $n_{0x} \leftarrow n_0$.     
Let $W_{0x}$ and $\lambdahat^2_{0x}$ be the sample mean and sample variance respectively of these samples.     Let $t\leftarrow 0$.     
\STATE Let $A \leftarrow \{ 1,\ldots, k\}$, $\upthresh \leftarrow P^*$, $t \leftarrow 1$.
\WHILE{$x\in\mbox{max}_{x\in A}\qhat_{tx}(A)<P$}
\WHILE{$\mbox{min}_{x\in A} \qhat_{tx}(A) \le c$}
 \STATE Let $x\in\mbox{arg min}_{x\in A}\qhat_{tx}(A)$.
    \STATE Let $\upthresh \leftarrow \upthresh/(1-\qhat_{tx}(A))$.     
\STATE Remove $x$ from $A$.
\ENDWHILE
  \STATE Let $z \in \mbox{arg min}_{x\in A} n_{tx} / \lambdahat^2_{tx}$.     
\STATE For each $x\in A$, let      $n_{t+1,x} = \ceil\left( \lambdahat^2_{tx} (n_{tz} + B_z) / \lambdahat^2_{tz} \right)$.     \STATE For each $x\in A$, if $n_{t+1,x}>n_{tx}$, take $n_{t+1,x}-n_{tx}$ additional samples from alternative $x$.  Let $W_{t+1,x}$ and $\lambdahat^2_{t+1,x}$ be the sample mean and sample variance respectively of all samples from alternative $x$ thus far.    
\STATE Increment $t$.
 \ENDWHILE
  \STATE Select $\xhat \in\mbox{arg max}_{x\in A} W_{tx} / n_{tx}$ as our estimate of the best.

   
\end{algorithmic}   



\section{Introduction}

This paper is organized as follows: In $\text{§}2$, we present the
proof of the validity of the algorithm when the variances are known.
In $\text{§}3$, we prove the case when the variances are unknown.

To prove the case when the variances are known, we use a Functional
Central Limit Theorem that shows how to standardize the output data
to make them behave like Brownian motion processes in the limit. We
also use an extension of the Continuous Mapping Theorem (Theorem 5.5
of Billingsley 1968) to see that the algorithm behaves like a sequential
elimination IZ procedure with a Brownian motion process instead of
the standardize of the sum of the output data in the limit. Finally,
we use the results of the paper of Frazier \cite{key-5} to prove
the validity of this algorithm in the limit.


\section{Asymptotic Validity when the Variances are Known}

Without loss of generality, suppose that the true means of the systems
are indexed so that $\mu_{k}\geq\mu_{k-1}\geq\cdots\geq\mu_{1}$.
We suppose that samples from system $x\in\left\{ 1\ldots,k\right\} $
are identically distributed and independent, over time and across
alternatives. We also define $\lambda_{z}^{2}:=\max_{i\in\left\{ 1\ldots,k\right\} }\lambda_{i}^{2}$.
We suppose that $\mbox{min}{}_{i\in\left\{ 1\ldots,k\right\} }\lambda_{i}^{2}>0$
and $c\neq1/k$.

Now we are going to see that the standardize of the sum of the output
data converges to a Brownian motion in $D[0,\infty)$, which is the
set of functions from $\left[0,\infty\right)$ to $\mathbb{R}$ that
are right-continuous and have left-hand limits, with the Skorohod
topology. The Skorohod metric $d_{t}$ on $D\left[0,t\right]$ is:
\[
d_{t}\left(X,Y\right)=\mbox{inf}_{\lambda\in\Lambda_{t}}\left\{ \left\Vert \lambda-I\right\Vert \vee\left\Vert X-Y\circ\lambda\right\Vert \right\} 
\]
where $\Lambda_{t}$ is the set of strictly increasing, continuous
mappings of $\left[0,t\right]$ onto itself, and $\left\Vert \cdot\right\Vert $
is the uniform norm, and $I$ is the identity map. Note that uniform
convergence on $\left[0,t\right]$ implies Skorohod convergence.

For $X\in D\left[0,\text{\ensuremath{\infty}}\right)$, let $X^{m}$
be the element of $D[0,\infty)$ defined by
\[
X^{m}\left(t\right)=g_{m}\left(t\right)X\left(t\right)
\]
where 
\[
g_{m}\left(t\right)=\begin{cases}
1 & \mbox{if }t\leq m-1,\\
m-t & \mbox{if }m-1\leq t\leq m,\\
0 & \mbox{if }t\geq m.
\end{cases}
\]
And now take
\[
d_{\infty}\left(X,Y\right)=\sum_{m=1}^{\infty}2^{-m}\left(1\wedge d_{m}\left(X,Y\right)\right)
\]
which is the Skorohod metric on $D\left[0,\infty\right)$.


\paragraph*{Lemma 1.}

If $x\in\left\{ 1\ldots,k\right\} $, then
\[
C_{x}\left(\delta,\cdot\right):=\frac{Y_{\mbox{ceil}\left(\frac{\lambda_{x}^{2}}{\lambda_{z}^{2}}\left(n_{0}+\cdot\frac{1}{\delta^{2}}\right)\right),x}-\frac{\lambda_{x}^{2}}{\lambda_{z}^{2}}\left(n_{0}+\cdot\frac{1}{\delta^{2}}\right)\mu_{x}}{\frac{\lambda_{x}^{2}}{\lambda_{z}}\sqrt{\frac{1}{\delta^{2}}}}\Rightarrow W_{x}\left(\cdot\right)
\]
in the sense of $D[0,\infty)$, where $Y_{n,x}$ is the sum of the
first $n$ samples and $W_{x}$ is a standard Brownian motion.


\paragraph*{Proof.}

By the Theorem 19.1 of Billingsley 1999 and the sandwich theorem,
\[
\frac{Y_{\mbox{ceil}\left(\frac{\lambda_{x}^{2}}{\lambda_{z}^{2}}\left(\cdot\frac{1}{\delta^{2}}\right)\right),x}-\mbox{ceil}\left(\frac{\lambda_{x}^{2}}{\lambda_{z}^{2}}\left(\cdot\frac{1}{\delta^{2}}\right)\right)\mu_{x}}{\frac{\lambda_{x}^{2}}{\lambda_{z}}\sqrt{\frac{1}{\delta^{2}}}}\Rightarrow W_{x}\left(\cdot\right).
\]


Since $\frac{\frac{\lambda_{x}^{2}}{\lambda_{z}^{2}}t\frac{1}{\delta^{2}}-ceil\left(\frac{\lambda_{x}^{2}}{\lambda_{z}^{2}}t\frac{1}{\delta^{2}}\right)}{\frac{\lambda_{x}^{2}}{\lambda_{z}}\sqrt{\frac{1}{\delta^{2}}}}\rightarrow0$
uniformly on $[0,r]$ for every $r$, then it also converges to $0$
on $D[0,\infty)$ and so 
\[
\frac{Y_{\mbox{ceil}\left(\frac{\lambda_{x}^{2}}{\lambda_{z}^{2}}\left(\cdot\frac{1}{\delta^{2}}\right)\right),x}-\mbox{}\left(\frac{\lambda_{x}^{2}}{\lambda_{z}^{2}}\left(\cdot\frac{1}{\delta^{2}}\right)\right)\mu_{x}}{\frac{\lambda_{x}^{2}}{\lambda_{z}}\sqrt{\frac{1}{\delta^{2}}}}\Rightarrow W_{x}\left(\cdot\right).
\]


Observe that for $\epsilon>0$ and $\delta$ sufficiently small
\[
\left|\frac{-Y_{\mbox{ceil}\left(\frac{\lambda_{x}^{2}}{\lambda_{z}^{2}}t\frac{1}{\delta^{2}}\right),x}+Y_{\mbox{ceil}\left(n_{0}\frac{\lambda_{x}^{2}}{\lambda_{z}^{2}}+\frac{\lambda_{x}^{2}}{\lambda_{z}^{2}}t\frac{1}{\delta^{2}}\right),x}}{\frac{\lambda_{x}^{2}}{\lambda_{z}}\sqrt{\frac{1}{\delta^{2}}}}\right|<\epsilon\left(n_{0}\frac{\lambda_{x}^{2}}{\lambda_{z}^{2}}+2\right)
\]
and then 
\[
C_{x}\left(\delta,\cdot\right)\Rightarrow W_{x}\left(\cdot\right).
\]


$\;\;\;\;\;\;\;\;\;\;\;\;\;\;\;\;\;\;\;\;\;\;\;\;\;\;\;\;\;\;\;\;\;\;\;\;\;\;\;\;\;\;\;\;\;\;\;\;\;\;\;\;\;\;\;\;\;\;\;\;\;\;\;\;\;\;\;\;\;\;\;\;\;\;\;\;\;\;\;\;\;\;\;\;\;\;\;\;\;\;\;\;\;\;\;\;\;\;\;\;\;\;\;\;\;\;\;\;\;\;\;\;\;\;\;\;\;\;\;\;\;\;\;\;\;\;\;\;\;\;\;\;\;\;\;\;\;\;\;\text{\ensuremath{\blacksquare}}$

${\color{white}sssssssssssssssssssssssssssssssssssssssssssssssssssssssssssssssssssssssssssssssssssssssssssss}$$\;\;\;\;\;\;\;\;\;\;\;\;\;\;\;\;\;\;\;\;\;\;\;\;\;\;\;\;\;\;\;\;\;\;\;\;\;\;\;\;\;\;\;\;\;\;\;\;\;\;\;\;\;\;\;\;\;\;\;\;\;\;\;\;\;\;\;\;\;\;\;\;\;\;\;\;\;\;\;\;\;\;\;\;\;\;\;\;\;\;\;\;\;\;\;\;\;\;\;\;\;\;\;\;\;\;\;\;\;\;\;\;\;\;\;\;\;\;\;\;\;\;\;\;\;\;\;\;\;\;\;\;\;\;\;\;\;\;\;\text{\ensuremath{\blacksquare}}$

Now we are going to define new algorithms that are almost the same
than the one proposed by Frazier, but instead of $q_{tx}^{\delta}\left(A\right)$,
these algorithms use new functions $q_{tx}^{Y,\delta}\left(A\right)$
which depend on a function $Y$ that is in $D\left[0,\infty\right)^{k}$. 

First we are going to suppose that $\delta>0$ and $\mu_{k}=\delta$,$\mu_{k-1}=\cdots=\mu_{\text{1}}=0$.

Let$ $ $Y\in D\left[0,\infty\right)^{k}$ , we define
\begin{eqnarray*}
q_{tx}^{Y,\delta}\left(A\right): & = & \mbox{exp}\left(\frac{Y_{x}\left(t\right)}{\lambda_{z}}+\delta^{2}\beta_{t\frac{1}{\delta^{2}}}I_{\left\{ x=k\right\} }\right)/\sum_{x^{'}\in A}\mbox{exp}\left(\frac{Y_{x^{'}}\left(t\right)}{\lambda_{z}}+\delta^{2}\beta_{t\frac{1}{\delta^{2}}}I_{\left\{ x^{'}=k\right\} }\right)
\end{eqnarray*}
where $\beta_{t\frac{1}{\delta^{2}}}=\frac{\left(n_{0}+t\frac{1}{\delta^{2}}\right)}{\lambda_{z}^{2}}$.

We define
\begin{eqnarray*}
T_{Y,\delta}^{0} & = & 0\\
A_{0}^{Y,\delta} & = & \left\{ 1,\ldots,k\right\} \\
P_{0}^{Y,\delta} & = & P^{*}\\
T_{Y,\delta}^{n+1}\left(P_{n}^{Y,\delta}\right) & = & \mbox{inf}\left\{ t\geq T_{Y,\delta}^{n}:\mbox{ min}_{x\in A_{n}^{Y,\delta}}q_{tx}^{Y,\delta}\left(A_{n}^{Y,\delta}\right)\leq c\mbox{ or }\mbox{max}_{x\in A_{n}^{Y,\delta}}q_{tx}^{Y,\delta}\left(A_{n}^{Y,\delta}\right)\geq P_{n}^{Y,\delta}\right\} \\
Z_{n+1}^{Y,\delta} & \in & \mbox{arg min}_{x\in A_{n}^{Y,\delta}}q_{T_{Y,\delta}^{n+1},x}^{Y,\delta}\left(A_{n}^{Y,\delta}\right)\\
A_{n+1}^{Y,\delta} & = & A_{n}-\left\{ Z_{n+1}^{Y,\delta}\right\} \\
P_{n+1}^{Y,\delta} & = & P_{n}^{Y,\delta}/\left(1-\mbox{min}_{x\in A_{n}^{Y,\delta}}q_{T_{Y,\delta}^{n+1}x}^{Y,\delta}\left(A_{n}^{Y,\delta}\right)\right).
\end{eqnarray*}
Now, let
\[
M_{Y,\delta}=\mbox{inf}\left\{ n=1,\ldots,k-1:\mbox{max}_{x\in A_{n-1}^{Y,\delta}}q_{T_{Y,\delta}^{n},x}\left(A_{n-1}^{Y,\delta}\right)\geq P_{n-1}^{Y,\delta}\right\} 
\]


and

\begin{eqnarray*}
f\left(Y,\delta\right) & = & \begin{cases}
1 & \mbox{if }k\mbox{ is chosen}\\
0 & \mbox{otherwise}
\end{cases}.
\end{eqnarray*}


Now, we also define$ $
\[
q_{tx}^{Y}\left(A\right):=\mbox{exp}\left(\frac{Y_{x}\left(t\right)}{\lambda_{z}}+\frac{1}{\lambda_{z}^{2}}tI_{\left\{ x=k\right\} }\right)/\sum_{x^{'}\in A}\mbox{exp}\left(\frac{Y_{x'}\left(t\right)}{\lambda_{z}}+\frac{1}{\lambda_{z}^{2}}tI_{\left\{ x'=k\right\} }\right)
\]


\begin{eqnarray*}
T_{Y}^{0} & = & 0\\
A_{0}^{Y} & = & \left\{ 1,\ldots,k\right\} \\
P_{0}^{Y} & = & P^{*}\\
T_{Y}^{n+1}\left(P_{n}^{Y}\right) & = & \mbox{inf}\left\{ t\geq T_{Y}^{n}:\mbox{ min}_{x\in A_{n}^{Y}}q_{tx}^{Y}\left(A_{n}^{Y}\right)\leq c\mbox{ or }\mbox{max}_{x\in A_{n}^{Y}}q_{tx}^{Y}\left(A_{n}^{Y}\right)\geq P_{n}^{Y}\right\} \\
Z_{n+1}^{Y} & \in & \mbox{arg min}_{x\in A_{n}^{Y}}q_{T_{Y}^{n+1},x}^{Y}\left(A_{n}^{Y}\right)\\
A_{n+1}^{Y} & = & A_{n}^{Y}-\left\{ Z_{n+1}^{Y}\right\} \\
P_{n+1}^{Y} & = & P_{n}^{Y,\delta}/\left(1-\mbox{min}_{x\in A_{n}^{Y,}}q_{T_{Y}^{n+1}x}^{Y}\left(A_{n}^{Y}\right)\right).
\end{eqnarray*}


Now, let
\[
M_{Y}=\mbox{inf}\left\{ n=1,\ldots,k-1:\mbox{max}_{x\in A_{n-1}^{Y}}q_{T_{Y}^{n},x}\left(A_{n-1}^{Y}\right)\geq P_{n-1}^{Y}\right\} 
\]
and
\[
\]
\begin{eqnarray*}
g\left(Y\right) & = & \begin{cases}
1 & \mbox{if }k\mbox{ is chosen}\mbox{ }\\
0 & \mbox{otherwise}
\end{cases}.
\end{eqnarray*}


We want to prove that 
\[
f\left(C\left(\delta,\cdot\right),\delta\right)\Rightarrow g\left(W\right).
\]


In order to prove this, we will prove the following lemma which will
allow us to use the Theorem 5.5 of Billingsley 1968 that implies the
desired result.


\paragraph*{Lemma 2.}

Let $\left\{ \delta_{n}\right\} \subset\left(0,\infty\right)$ such
that $\delta_{n}\rightarrow0$. If $D_{s}\equiv\{x\in D\left[0,\infty\right)^{k}:\mbox{ for all sequences }\left\{ x_{n}\right\} \subset D\left[0,\infty\right)^{k},\mbox{ such that }$
$lim_{n}d\left(x_{n},x\right)=0$ the sequence $\left\{ f\left(x_{n},\delta_{n}\right)\right\} $
converges to $\left\{ g\left(x\right)\right\} $$\left.\right\} $$ $,
then $ $$\mathbb{P}\left(W\text{\ensuremath{\in}}D_{s}\right)=1$.


\paragraph*{Proof.}

We will use the following property: if $T$ is a stopping time, then
by the local version of the law of the iterated logarithm for Brownian
motion 
\begin{equation}
\mbox{lim sup}_{u\rightarrow0^{+}}\frac{\left[W_{x}\left(T+u\right)-W_{x}\left(T\right)\right]}{\sqrt{2u\mbox{ln}\left[\mbox{ln}\left(1/u\right)\right]}}=1\label{eq:1.1}
\end{equation}
almost surely for each system $x$. Furthermore, note that $W$ is
almost surely continuous on $\left[0,t\right]^{k}$ if $t>0$ and
so $W$ is also almost surely uniformly continuous on $\left[0,t\right]^{k}$.

Let $\left\{ Z_{n}\right\} \subset D\left[0,\infty\right)^{k}$ such
that $Z_{n}\rightarrow W$. Note that $\delta_{n}^{2}\frac{\left(n_{0}+t\frac{1}{\delta_{n}^{2}}\right)}{\lambda_{z}^{2}}\rightarrow\frac{t}{\lambda_{z}^{2}}$
in $D\left[0,\infty\right)$ and $\frac{\delta_{n}\sqrt{\frac{1}{\delta_{n}^{2}}}}{\lambda_{z}}\rightarrow\frac{1}{\lambda_{z}}$
in $D\left[0,\infty\right)$ because uniformly convergence implies
convergence in the Skorohod topology. Consequently, for each $s>0$
there exist functions $\lambda_{s}^{n}$ in $\Lambda$ such that 
\[
\mbox{lim}_{n}Z_{n}\left(\lambda_{s}^{n}t\right)=W\left(t\right)
\]
uniformly in $t$ and
\[
\mbox{lim}_{n}\lambda_{s}^{n}t=t
\]
uniformly in $t$. Then
\[
\mbox{lim}_{n}\frac{\delta_{n}\sqrt{\frac{1}{\delta_{n}^{2}}}}{\lambda_{z}}Z_{n}\left(\lambda_{s}^{n}t\right)+\delta_{n}^{2}\frac{\left(n_{0}+\lambda_{s}^{n}\left(t\right)\frac{1}{\delta_{n}^{2}}\right)}{\lambda_{z}^{2}}=W\left(t\right)\frac{1}{\lambda_{z}}+\frac{t}{\lambda_{z}^{2}}
\]
uniformly in $t$, and so
\[
\mbox{lim}_{n}\mbox{exp}\left(\frac{\delta_{n}\sqrt{\frac{1}{\delta_{n}^{2}}}}{\lambda_{z}}Z_{n}\left(\lambda_{n}t\right)+\delta_{n}^{2}\frac{\left(n_{0}+\lambda_{s}^{n}\left(t\right)\frac{1}{\delta_{n}^{2}}\right)}{\lambda_{z}^{2}}\right)=\mbox{exp}\left(W\left(t\right)\frac{1}{\lambda_{z}}+\frac{t}{\lambda_{z}^{2}}\right)
\]
uniformly in $t$ since $\mbox{exp}$ is uniformly continuous in $\left[0,s\right]$.
Consequently,
\[
q_{\lambda_{s}^{n}\left(t\right)x}^{Z_{n},\delta_{n}}\left(A\right)\rightarrow q_{tx}^{W}\left(A\right)
\]
uniformly in $t\in\left[0,s\right]$. Thus, $q_{\cdot x}^{Z_{n},\delta_{n}}\left(A\right)\rightarrow q_{\cdot x}^{W}\left(A\right)$
in $D\left[0,s\right]$ for any set $A\subset\left\{ 1,\ldots,k\right\} $
and $s\geq0$. Consequently, $q_{\cdot x}^{Z_{n},\delta_{n}}\left(A\right)\rightarrow q_{\cdot x}^{W}\left(A\right)$
in $D\left[0,\infty\right)$. Then for each $m\geq0$ there exists
$\lambda_{n,A}\in\Lambda_{\infty}$ such that $\mbox{sup}_{t<\infty}\left\Vert \lambda_{n,A}\left(t\right)-t\right\Vert \leq d\left(q_{\cdot x}^{Z_{n},\delta_{n}}\left(A\right),q_{\cdot x}^{W}\left(A\right)\right)+\frac{1}{n}$
and $\mbox{sup}_{t\leq m}\left\Vert q_{tx}^{Z_{n},\delta_{n}}\left(A\right)-q_{\lambda_{n,A}\left(t\right)x}^{W}\left(A\right)\right\Vert \leq d\left(q_{\cdot x}^{Z_{n},\delta_{n}}\left(A\right),q_{\cdot x}^{W}\left(A\right)\right)+\frac{1}{n}$.
Taking $g_{n}^{A}\equiv\mbox{sup}_{t\leq m}\left\Vert q_{tx}^{W}\left(A\right)-q_{\lambda_{n}\left(t\right)x}^{W}\left(A\right)\right\Vert $,
we see from the uniform continuity of $W$ on $\left[0,m\right]^{k}$
and the definition of $g_{n}^{A}$ that $\mbox{lim}_{n\rightarrow\infty}g_{n}^{A}=0.$
Moreover, if we take $\epsilon_{n}^{A}=3n^{-1}+3\mbox{sup}\left\{ d\left(q_{\cdot x}^{Z_{l},\delta_{l}}\left(A\right),q_{\cdot x}^{W}\left(A\right)\right)+g_{l}^{A}:l=n,n+1,\ldots\right\} $,
then $\left\{ \epsilon_{n}^{A}\right\} $ is a monotonically decreasing
sequence of positive numbers with limit zero.

From the definition of $\epsilon_{n}$ we have $d\left(q_{\cdot x}^{Z_{n},\delta_{n}}\left(A\right),q_{\cdot x}^{W}\left(A\right)\right)<\epsilon_{n}/2$
and $g_{n}^{A}<\epsilon_{n}/2$ for $n=1,2,\ldots$ Consequently,
we have 
\begin{eqnarray*}
\left\Vert q_{tx}^{Z_{n},\delta_{n}}\left(A\right)-q_{tx}^{W}\left(A\right)\right\Vert  & \leq & \left\Vert q_{tx}^{Z_{n},\delta_{n}}\left(A\right)-q_{\lambda_{n}\left(t\right)x}^{W}\left(A\right)\right\Vert +\left\Vert q_{\lambda_{n}\left(t\right)x}^{W}\left(A\right)-q_{tx}^{W}\left(A\right)\right\Vert \\
 & < & \epsilon_{n}^{A}
\end{eqnarray*}
for all $t\in\left[0,m\right]$ and $x\in A$.

We will show that $\mathbb{P}\left(W\text{\ensuremath{\in}}D_{s}\mid M_{W}=i\right)=1$
for $i\in\left\{ 1,\ldots,k-1\right\} $ so that the desired conclusion
follows.

Suppose first that $M_{W}=1$. Let's prove that $T_{Z_{n},\delta_{n}}^{1}\left(P\right)\rightarrow T_{W}^{1}\left(P\right)$
as $n\rightarrow\infty$. Since $M_{W}=1$, then $\mbox{max}_{x\in A}q_{T_{W}^{1}\left(P\right)x}^{W}\left(A\right)=P$
almost surely because $W$ is continuous almost surely. Let $q_{x}^{*}=\mbox{min}_{t\in\left[0,T_{W}^{1}\left(P\right)\right]}q_{tx}^{W}$
(the minimum exists because $q_{tx}^{W}$ is continuous) and let $q^{*}=\mbox{min}_{x}q_{x}^{*}$.
Let $N$ such that if $n>N$ then $\epsilon_{n}^{A}<q^{*}-c$ (note
that in the last equation $m=T_{W}^{1}\left(P\right)$ and $q^{*}-c>0$
because $M_{W}=1$ and $N$ is a random variable). If $n>N$ and $t\leq T_{W}^{1}\left(P\right)$
then
\begin{eqnarray*}
q_{tx}^{W}\left(A\right) & < & q_{tx}^{Z_{n},\delta_{n}}\left(A\right)+q^{*}-c\\
\Rightarrow c & < & q_{tx}^{Z_{n},\delta_{n}}\left(A\right)+q^{*}-q_{tx}^{W}\left(A\right)\leq q_{tx}^{Z_{n},\delta_{n}}\left(A\right).
\end{eqnarray*}


Case 1. $T_{W}^{1}\left(P\right)=0$. Note that $q_{0x}^{W}\left(A\right)=\frac{1}{k}<P$
and so this case is not possible. 

Case 2. $0<T_{W}^{1}\left(P\right)$. 

Note that $T_{Z_{n},\delta_{n}}^{1}\left(P\right)\leq T_{Z_{n},\delta_{n}}^{1}\left(P+\delta_{n}\right)\leq T_{W}^{1}\left(P+\delta_{n}+\epsilon_{n}\right)$
because if $t<T_{Z_{n},\delta_{n}}^{1}\left(P+\delta_{n}\right)$
(for the sequence $\left\{ \epsilon_{n}\right\} $ we take $m=\mbox{sup}_{n}P+\delta_{n}$)
\[
q_{tx}^{W}\left(A\right)-\epsilon_{n}<q_{tx}^{Z_{n},\delta_{n}}\left(A\right)<P+\delta_{n}.
\]
Furthermore, 
\begin{eqnarray*}
T_{W}^{1}\left(P\right) & \leq & \mbox{lim inf}_{n}T_{W}^{1}\left(P+\delta_{n}+\epsilon_{n}\right)\\
 & \leq & \mbox{lim sup}_{n}T_{W}^{1}\left(P+\delta_{n}+\epsilon_{n}\right).
\end{eqnarray*}
 Now, let $x=\mbox{arg max}q_{T_{W}^{1}x}^{W}\left(A\right)$. By
(\ref{eq:1.1}), for any $\varepsilon\in\left(0,1\right)$, there
exists a monotonically decreasing sequence $\left\{ s_{k}:\mbox{ }k=1,2,\ldots\right\} \subset\left(T_{W}^{1}\left(P\right),\infty\right)$
such that $\mbox{lim}_{n}s_{n}=T_{W}^{1}\left(P\right)$; and for
$k=1,2,\ldots$ we have

\begin{eqnarray*}
W_{x}\left(s_{k}\right)-W_{x}\left(T_{W}^{1}\left(P\right)\right) & >v_{k}\equiv & \left(\varepsilon\right)\sqrt{2\left(s_{k}-T_{W}^{1}\left(A\right)\right)\mbox{ln}\mbox{ln}\left(1/\left(s_{k}-T_{W}^{1}\left(P\right)\right)\right)}.
\end{eqnarray*}


Notice that $v_{k}>0$ and $\mbox{lim}_{k}v_{k}=0$. Let $\alpha>0$.
Let $\delta$ such that if $\delta_{3}<\delta$, then $\alpha>-\mbox{log}\left(1-\delta_{3}\right)$.
Pick $\delta_{3}<\delta$. Let $K$ such that if $k>K$, $\frac{\sum_{x^{'}\in A}\mbox{exp}\left(\frac{W_{^{x'}}\left(T_{W}^{1}\left(P\right)\right)}{\lambda_{z}}+\frac{1}{\lambda_{z}^{2}}T_{W}^{1}\left(P\right)I_{\left\{ x^{'}=k\right\} }\right)}{\sum_{x^{'}\in A}\mbox{exp}\left(\frac{W_{x'}\left(s_{k}\right)}{\lambda_{z}}+\frac{1}{\lambda_{z}^{2}}s_{k}I_{\left\{ x'=k\right\} }\right)}>1-\delta_{3}$.
Pick $k>K$. Let $N_{4}$ such that if $n>N_{4}$, then $\frac{\epsilon_{n}}{P}<v_{k}\frac{1}{\lambda_{z}}+\alpha$
and $\frac{\delta_{n}}{P}<\left(v_{k}\frac{1}{\lambda_{z}}+\alpha\right)^{2}/2$.
$ $Observe that
\begin{eqnarray*}
q_{s_{k}x}^{W} & = & \mbox{exp}\left(\frac{W_{x}\left(s_{k}\right)}{\lambda_{z}}+\frac{1}{\lambda_{z}^{2}}s_{k}I_{\left\{ x=k\right\} }\right)/\sum_{x^{'}\in A}\mbox{exp}\left(\frac{W_{x'}\left(s_{k}\right)}{\lambda_{z}}+\frac{1}{\lambda_{z}^{2}}s_{k}I_{\left\{ x'=k\right\} }\right)\\
 & > & \frac{\mbox{exp}\left(\frac{W_{x}\left(T_{W}^{1}\left(P\right)\right)+v_{k}}{\lambda_{z}}+\frac{1}{\lambda_{z}^{2}}T_{W}^{1}\left(P\right)I_{\left\{ x=k\right\} }\right)}{\sum_{x^{'}\in A}\mbox{exp}\left(\frac{W_{^{x'}}\left(T_{W}^{1}\left(P\right)\right)}{\lambda_{z}}+\frac{1}{\lambda_{z}^{2}}T_{W}^{1}\left(P\right)I_{\left\{ x^{'}=k\right\} }\right)}\times\frac{\sum_{x^{'}\in A}\mbox{exp}\left(\frac{W_{^{x'}}\left(T_{W}^{1}\left(P\right)\right)}{\lambda_{z}}+\frac{1}{\lambda_{z}^{2}}T_{W}^{1}\left(P\right)I_{\left\{ x^{'}=k\right\} }\right)}{\sum_{x^{'}\in A}\mbox{exp}\left(\frac{W_{x'}\left(s_{k}\right)}{\lambda_{z}}+\frac{1}{\lambda_{z}^{2}}s_{k}I_{\left\{ x'=k\right\} }\right)}\\
 & \geq & q_{T_{W}^{1}\left(P\right)x}^{W}\left(A\right)\mbox{exp}\left(v_{k}\frac{1}{\lambda_{z}}\right)\left(1-\delta_{3}\right)\geq q_{T_{W}^{1}\left(P\right)x}^{W}\left(A\right)\mbox{exp}\left(v_{k}\frac{1}{\lambda_{z}}\right)\mbox{exp}\left(\alpha\right)\\
 & \geq & P\left(1+v_{k}\frac{1}{\lambda_{z}}+\alpha+\frac{\left(v_{k}\frac{1}{\lambda_{z}}+\alpha\right)^{2}}{2}\right)>\left(P+\epsilon_{n}+\delta_{n}\right).
\end{eqnarray*}
Consequently, 
\[
s_{k}>T_{W}^{1}\left(P+\epsilon_{n}+\delta_{n}\right)
\]
if $n>N_{4}$, and so 
\[
s_{k}\geq\mbox{lim sup}_{n}T_{W}^{1}\left(P+\delta_{n}+\epsilon_{n}\right).
\]


Since $s_{k}\rightarrow T_{W}^{1}$, we must have that
\begin{eqnarray*}
T_{W}^{1}\left(P\right) & \geq & \mbox{lim sup}_{n}T_{W}^{1}\left(P+\delta_{n}+\epsilon_{n}\right)\\
 & \geq & \mbox{lim inf}_{n}T_{W}^{1}\left(P+\delta_{n}+\epsilon_{n}\right)\\
 & \geq & T_{W}^{1}\left(P\right)
\end{eqnarray*}
Consequently, 
\[
T_{W}^{1}\left(P\right)=\mbox{lim}_{n}T_{W}^{1}\left(P+\delta_{n}+\epsilon_{n}\right).
\]


Now,
\[
T_{Z_{n},\delta_{n}}^{1}\left(P\right)\geq T_{Z_{n},\delta_{n}}^{1}\left(P-\delta_{n}\right)\geq T_{W}^{1}\left(P-\delta_{n}-\epsilon_{n}\right)
\]


because if $t<T_{W}^{1}\left(P-\delta_{n}-\epsilon_{n}\right)$ 
\[
q_{tx}^{Z_{n},\delta_{n}}\left(A\right)-\epsilon_{n}<q_{tx}^{W}\left(A\right)<P-\delta_{n}-\epsilon_{n}.
\]
Similarly we can prove that 
\[
\mbox{lim}_{n}T_{W}^{1}\left(P-\delta_{n}-\epsilon_{n}\right)=T_{W}^{1}\left(P\right).
\]
Consequently,
\[
T_{W}^{1}\left(P\right)=\mbox{lim}_{n}T_{Z_{n},\delta_{n}}^{1}\left(P\right).
\]


Let $x=\mbox{arg max}_{x}q_{T_{W}^{1}\left(P\right),x}^{W}\left(A\right)$.
Let $\epsilon>0$. Let $N$ such that if $n>N$, then $2\epsilon_{n}+2\epsilon<-\mbox{max}_{y\in A-\left\{ x\right\} }q_{T_{W\left(P\right)}^{1},y}^{W}\left(A\right)+q_{T_{W}^{1}\left(P\right),x}^{W}\left(A\right)$
\[
\left|q_{T_{W}^{1}\left(P\right),x}^{W}\left(A\right)-q_{T_{Z_{n},\delta_{n}}^{1}\left(P\right),x}^{Z_{n},\delta_{n}}\left(A\right)\right|<\epsilon+\epsilon_{n}
\]
for all $x\in A$, and so if $z\in A-\left\{ x\right\} $,
\begin{eqnarray*}
q_{T_{Z_{n},\delta_{n}}^{1}\left(P\right),z}^{Z_{n},\delta_{n}}\left(A\right) & < & \epsilon+\epsilon_{n}+q_{T_{W}^{1}\left(P\right),z}^{W}\left(A\right)\\
 & < & -\epsilon-\epsilon_{n}+q_{T_{W}^{1}\left(P\right),x}^{W}\left(A\right)\\
 & < & q_{T_{Z_{n},\delta_{n}}^{1}\left(P\right),x}^{Z_{n},\delta_{n}}\left(A\right)
\end{eqnarray*}
and so $x=\mbox{arg max}_{x}q_{T_{Z_{n},\delta_{n}}^{1}\left(P\right),x}^{Z_{n},\delta_{n}}\left(A\right)$.
\[
\]


Now, we suppose that $ $$M_{W}=2$. Let's prove that $T_{W}^{1}\left(A\right)=\mbox{lim}_{n}T_{Z_{n},\delta_{n}}^{1}$. 

Case 1. $0<T_{W}^{1}\left(P\right)$. Like in the previous proof,
we conclude that 
\[
T_{W}^{1}\left(P\right)=\mbox{lim}_{n}T_{Z_{n},\delta_{n}}^{1}\left(P\right).
\]


Furthermore, if $x=\mbox{arg min}_{x\in A_{1}^{W}}q_{T_{W}^{1}\left(P\right),x}^{W}\left(A\right)$.
Let $N$ such that if $n>N$, then $2\epsilon_{n}+2\epsilon<\mbox{min}_{y\in A_{1}^{W}-\left\{ x\right\} }q_{T_{W}^{1}\left(P\right),y}^{W}\left(A\right)-q_{T_{W}^{1}\left(P\right),x}^{W}\left(A\right)$
for some $\epsilon>0$ and
\[
\left|q_{T_{Z_{n},\delta_{n}}^{1}\left(P\right)z}^{Z_{n},\delta_{n}}-q_{T_{W}^{1}\left(P\right),z}^{W}\right|<\epsilon
\]
for all $z\in A-\left\{ x\right\} $.

Then if \textbf{\large $z\in A-\left\{ x\right\} $, 
\begin{eqnarray*}
q_{T_{Z_{n},\delta_{n}}^{1}\left(P\right),z}^{Z_{n},\delta_{n}}\left(A\right) & > & -\epsilon_{n}-\epsilon+q_{T_{W}^{1}\left(P\right),z}^{W}\left(A\right)>q_{T_{W}^{1}\left(P\right),x}^{W}\left(A\right)+\epsilon_{n}+\epsilon\\
 & > & q_{T_{Z_{n},\delta_{n}}^{1}\left(P\right),x}^{Z_{n},\delta_{n}}\left(A\right)
\end{eqnarray*}
}{\large \par}

for $n$ sufficiently large and so $x=\mbox{arg min}_{x\in A_{n}^{Y}}q_{T_{Z_{n,\delta_{n}}}^{1}\left(P\right),x}^{Z_{n},\delta_{n}}\left(A\right)$.
Consequently, $A_{1}^{Z_{n},\delta_{n}}=A_{1}^{W}$ for $n$ sufficiently
large. 

Case 2. $T_{W}^{1}\left(P\right)=0$, then $ $$\mbox{ min}_{x\in A_{n}^{Y}}q_{0x}^{W}\left(A_{n}^{W}\right)=1/k\leq c$.
Suppose $c>\frac{1}{k}$. Let $N$ such that if $n>N$, then $\epsilon_{n}<c-\frac{1}{k}$.
Thus if $n>N$, 
\[
q_{0x}^{Z_{n},\delta_{n}}<\epsilon_{n}+\frac{1}{k}<c
\]


and so $T_{Z_{n},\delta_{n}}^{1}=0$. We can also see that $A_{1}^{Z_{n},\delta_{n}}=A_{1}^{W}$
for $n$ sufficiently large.

Now, let's prove that $T_{W}^{2}\left(P_{1}^{W}\right)=\mbox{lim}_{n}T_{Z_{n},\delta_{n}}^{2}\left(P_{1}^{Z_{n},\delta_{n}}\right)$.
By the above argument, we know that there exists $N$ such that if
$n>N$, then $A_{1}^{Z_{n},\delta_{n}}=A_{1}^{W}$ and
\[
P_{1}^{Z_{n},\delta_{n}}<P_{1}^{W}.
\]


Case 1. $T_{W}^{2}\left(P_{1}^{W}\right)=0=T_{W}^{1}\left(P\right)$.
This is impossible because $P_{1}^{W}\leq\frac{1}{k}\leq c$ and $ $$P_{1}^{W}\geq P^{*}>\frac{1}{k}$.

Case 2. $0<T_{W}^{2}\left(P_{1}^{W}\right)$.

Note that $T_{Z_{n},\delta_{n}}^{2}\left(P_{1}^{Z_{n}}\right)\leq T_{Z_{n},\delta_{n}}^{2}\left(P_{1}^{Z_{n}}+\delta_{n}\right)\leq T_{W}^{2}\left(P_{1}^{Z_{n}}+\delta_{n}+\epsilon_{n}\right)$
because if $T_{Z_{n},\delta_{n}}^{1}\left(P\right)\leq t<T_{Z_{n},\delta_{n}}^{2}\left(P_{1}^{Z_{n}}+\delta_{n}\right)$
\begin{eqnarray*}
q_{tx}^{W}\left(A_{1}^{W}\right)-\epsilon_{n} & < & q_{tx}^{Z_{n},\delta_{n}}\left(A_{1}^{W}\right)<P_{1}^{Z_{n}}+\delta_{n}
\end{eqnarray*}
Now, let $x=\mbox{arg max}q_{T_{W}^{2}\left(P_{1}^{W}\right)x}^{W}\left(A_{1}^{W}\right)$.
By (\ref{eq:1.1}), for any $\varepsilon\in\left(0,1\right)$, there
exists a monotonically decreasing sequence $\left\{ s_{k}:\mbox{ }k=1,2,\ldots\right\} \subset\left(T_{W}^{2}\left(P_{1}^{W}\right),1\right)$
such that $\mbox{lim}_{n}s_{n}=T_{W}^{2}\left(P_{1}^{W}\right)$;
and for $k=1,2,\ldots$ we have

\begin{eqnarray*}
W_{x}\left(s_{k}\right)-W_{x}\left(T_{W}^{2}\left(P_{1}^{W}\right)\right) & >v_{k}\equiv & \left(\varepsilon\right)\sqrt{2\left(s_{k}-T_{W}^{2}\left(P_{1}^{W}\right)\right)\mbox{ln}\mbox{ln}\left(1/\left(s_{k}-T_{W}^{2}\left(P_{1}^{W}\right)\right)\right)}.
\end{eqnarray*}
 Notice that $v_{k}>0$ and $\mbox{lim}_{k}v_{k}=0$. Let $\alpha>0$.
Let $\delta$ such that if $\delta_{3}<\delta$, then $\alpha>-\mbox{log}\left(1-\delta_{3}\right)$.
Pick $\delta_{3}<\delta$. Let $K$ such that if $k>K$, $\frac{\sum_{x^{'}\in A_{1}^{W}}\mbox{exp}\left(\frac{W_{^{x'}}\left(T_{W}^{2}\left(P_{1}^{W}\right)\right)}{\lambda_{z}}+\frac{1}{\lambda_{z}^{2}}T_{W}^{2}\left(P_{1}^{W}\right)I_{\left\{ x^{'}=k\right\} }\right)}{\sum_{x^{'}\in A_{1}^{W}}\mbox{exp}\left(\frac{W_{x'}\left(s_{k}\right)}{\lambda_{z}}+\frac{1}{\lambda_{z}^{2}}s_{k}I_{\left\{ x'=k\right\} }\right)}>1-\delta_{3}$.
Pick $k>K$. Let $N_{4}$ such that if $n>N_{4}$, then $\frac{\epsilon_{n}}{P_{1}^{W}}<v_{k}\frac{1}{\lambda_{z}}+\alpha$
and $\frac{\delta_{n}}{P_{1}^{W}}<\left(v_{k}\frac{1}{\lambda_{z}}+\alpha\right)^{2}/2$.
$ $Observe that
\begin{eqnarray*}
q_{s_{k}x}^{W}\left(A_{1}^{W}\right) & = & \mbox{exp}\left(\frac{W_{x}\left(s_{k}\right)}{\lambda_{z}}+\frac{1}{\lambda_{z}^{2}}s_{k}I_{\left\{ x=k\right\} }\right)/\sum_{x^{'}\in A_{1}^{W}}\mbox{exp}\left(\frac{W_{x'}\left(s_{k}\right)}{\lambda_{z}}+\frac{1}{\lambda_{z}^{2}}s_{k}I_{\left\{ x'=k\right\} }\right)\\
 & > & \frac{\mbox{exp}\left(\frac{W_{k}\left(T_{W}^{2}\left(P_{1}^{W}\right)\right)+v_{k}}{\lambda_{z}}+\frac{1}{\lambda_{z}^{2}}T_{W}^{2}\left(P_{1}^{W}\right)I_{\left\{ x=k\right\} }\right)}{\sum_{x^{'}\in A_{1}^{W}}\mbox{exp}\left(\frac{W_{^{x'}}\left(T_{W}^{2}\left(P_{1}^{W}\right)\right)}{\lambda_{z}}+\frac{1}{\lambda_{z}^{2}}T_{W}^{2}\left(P_{1}^{W}\right)I_{\left\{ x^{'}=k\right\} }\right)}\\
 &  & \times\frac{\sum_{x^{'}\in A_{1}^{W}}\mbox{exp}\left(\frac{W_{^{x'}}\left(T_{W}^{2}\left(P_{1}^{W}\right)\right)}{\lambda_{z}}+\frac{1}{\lambda_{z}^{2}}T_{W}^{2}\left(P_{1}^{W}\right)I_{\left\{ x^{'}=k\right\} }\right)}{\sum_{x^{'}\in A_{1}^{W}}\mbox{exp}\left(\frac{W_{x'}\left(s_{k}\right)}{\lambda_{z}}+\frac{1}{\lambda_{z}^{2}}s_{k}I_{\left\{ x'=k\right\} }\right)}\\
 & \geq & q_{T_{W}^{2}\left(P_{1}^{W}\right)x}^{W}\mbox{exp}\left(v_{k}\frac{1}{\lambda_{z}}\right)\left(1-\delta_{3}\right)\geq q_{T_{W}^{2}\left(P_{1}^{W}\right)x}^{W}\left(A_{1}^{W}\right)\mbox{exp}\left(v_{k}\frac{1}{\lambda_{z}}\right)\mbox{exp}\left(\alpha\right)\\
 &  & \geq P_{1}^{W}\left(1+v_{k}\frac{1}{\lambda_{z}}+\alpha+\frac{\left(v_{k}\frac{1}{\lambda_{z}}+\alpha\right)^{2}}{2}\right)>P_{1}^{W}+\epsilon_{n}+\delta_{n}.
\end{eqnarray*}
Consequently, 
\[
s_{k}>T_{W}^{2}\left(P_{1}^{W}+\epsilon_{n}+\delta_{n}\right)
\]
if $n>N_{4}$, and so 
\[
s_{k}\geq\mbox{lim sup}_{n}T_{W}^{2}\left(P_{1}^{W}+\delta_{n}+\epsilon_{n}\right).
\]
Since $s_{k}\rightarrow T_{W}^{1}$, we must have that
\begin{eqnarray*}
T_{W}^{2}\left(P_{1}^{W}\right) & \geq & \mbox{lim sup}_{n}T_{W}^{2}\left(P_{1}^{W}+\delta_{n}+\epsilon_{n}\right)\\
 & \geq & \mbox{lim inf}_{n}T_{W}^{2}\left(P_{1}^{W}+\delta_{n}+\epsilon_{n}\right)\\
 & \geq & T_{W}^{2}\left(P_{1}^{W}\right)
\end{eqnarray*}
Consequently, 
\[
T_{W}^{2}\left(P_{1}^{W}\right)=\mbox{lim}_{n}T_{W}^{2}\left(P_{1}^{W}+\delta_{n}+\epsilon_{n}\right).
\]


Now,
\[
T_{Z_{n},\delta_{n}}^{2}\left(P_{1}^{Z_{n}}\right)\geq T_{Z_{n},\delta_{n}}^{2}\left(P_{1}^{Z_{n}}-\delta_{n}\right)\geq T_{W}^{2}\left(P_{1}^{Z_{n}}-\delta_{n}-\epsilon_{n}\right)
\]


because if $t<T_{W}^{2}\left(P_{1}^{Z_{n}}-\delta_{n}-\epsilon_{n}\right)$
\[
q_{tx}^{Z_{n},\delta_{n}}\left(A_{1}^{W}\right)-\epsilon_{n}<q_{tx}^{W}\left(A_{1}^{W}\right)<P_{1}^{Z_{n}}-\delta_{n}-\epsilon_{n}.
\]
Similarly we can prove that 
\[
\mbox{lim}_{n}T_{W}^{2}\left(P_{1}^{W}-\delta_{n}-\epsilon_{n}\right)=T_{W}^{2}\left(P_{1}^{W}\right).
\]
Consequently,
\[
T_{W}^{2}\left(P_{1}^{W}\right)=\mbox{lim}_{n}T_{Z_{n},\delta_{n}}^{2}\left(P_{1}^{W}\right).
\]


By a similar argument than before, we can see that $\mbox{arg max}_{x}q_{T_{Z_{n},\delta_{n}}^{2}\left(P_{1}^{W}\right),x}^{Z_{n},\delta_{n}}\left(A_{1}^{W}\right)=\mbox{arg max}_{x}q_{T_{W}^{2}\left(P_{1}^{W}\right),x}^{W}\left(A_{1}^{W}\right)$
for $n$ sufficiently large.

The cases $M_{W}=i$ for $k-1\geq i\geq3$ can be proved in a similar
way. 

Since almost surely $M_{Y}\in\left\{ 1,\ldots,k-1\right\} $ by Frazier
\cite{key-5}, we conclude that

\[
\mathbb{P}\left(W\text{\ensuremath{\in}D\ensuremath{\left[0,1\right]^{k}}-}D_{s}\right)=1.
\]


$\;\;\;\;\;\;\;\;\;\;\;\;\;\;\;\;\;\;\;\;\;\;\;\;\;\;\;\;\;\;\;\;\;\;\;\;\;\;\;\;\;\;\;\;\;\;\;\;\;\;\;\;\;\;\;\;\;\;\;\;\;\;\;\;\;\;\;\;\;\;\;\;\;\;\;\;\;\;\;\;\;\;\;\;\;\;\;\;\;\;\;\;\;\;\;\;\;\;\;\;\;\;\;\;\;\;\;\;\;\;\;\;\;\;\;\;\;\;\;\;\;\;\;\;\;\;\;\;\;\;\;\;\;\;\;\;\;\;\;\text{\ensuremath{\blacksquare}}$

\[
\]


By the extension of the CMT (Theorem 5.5 of Billingsley 1968), we
have the following corollary.


\paragraph*{Corollary 1.}

We have that 

\[
f\left(C\left(\delta,t\right),\delta\right)\Rightarrow g\left(W\left(t\right)\right)
\]
in distribution as $\delta\rightarrow0$.

\[
\]



\paragraph*{Theorem 1.}

If samples from system $x\in\left\{ 1\ldots,k\right\} $ are identically
distributed and independent, over time and across alternatives, then
$\mbox{lim}_{\delta\rightarrow0}Pr\left\{ \mbox{BIZ selects }k\right\} \geq P*$
provided $\mu_{k}=\delta,\mu_{k-1}=\cdots=\mu_{1}=0$. We also suppose
$B_{1}=\cdots=B_{k}=1$ and $c\neq\frac{1}{k}$.


\paragraph*{Proof.}

Let 
\[
\hat{T}_{n}\left(\delta\right)=\min\left\{ t\in\left\{ 0,\delta^{2},2\delta^{2},\ldots\right\} :\mbox{ min}_{x\in A_{n}^{Y,\delta}}q_{tx}^{C\left(\delta,\cdot\right),\delta}\left(A_{n}^{C\left(\delta,\cdot\right),\delta}\right)\leq c\mbox{ or }\mbox{max}_{x\in A_{n}^{Y,\delta}}q_{tx}^{C\left(\delta,\cdot\right),\delta}\left(A_{n}^{C\left(\delta,\cdot\right),\delta}\right)\geq P_{n}^{C\left(\delta,\cdot\right),\delta}\right\} 
\]
and $T_{n}\left(\delta\right)$ the usual stopping times of the algorithm.
Then $T_{n}\left(\delta\right)=\hat{T}_{n}\left(\delta\right)/\delta^{2}$.
Now, we can prove that $\hat{T}_{n}\left(\delta\right)-T_{C\left(\delta,\cdot\right),\delta}^{n}\left(P_{n}^{C\left(\delta,\cdot\right),\delta}\right)\rightarrow0$
with probability $1$ as $\delta\rightarrow0$ using that $C\left(\delta,\cdot\right)$
is right-continuous and $\delta^{2}\rightarrow0$. Consequently, we
can use $C\left(\delta,T_{C\left(\delta,\cdot\right),\delta}^{n}\left(P_{n}^{C\left(\delta,\cdot\right),\delta}\right)\right)$
instead of $C\left(\delta,\hat{T}_{n}\left(\delta\right)\right)$.

Let $CS_{\delta}$ be the event of doing a correct selection given
the configuration $\mu_{k}=\delta,\mu_{k-1}=\cdots=\mu_{1}=0$. Then
\begin{eqnarray*}
\underline{lim}_{\delta\rightarrow0}\mathbb{P}\left(CS_{\delta}\right) & = & \underline{lim}_{\delta\rightarrow0}\mathbb{P}\left(f\left(C\left(\delta,t\right),\delta\right)=1\right)\\
 & = & \mathbb{P}\left(g\left(W\right)=1\right)\\
 & \geq & P^{*}
\end{eqnarray*}
where the last inequality follows from the paper of Frazier \cite{key-5}. 

$\;\;\;\;\;\;\;\;\;\;\;\;\;\;\;\;\;\;\;\;\;\;\;\;\;\;\;\;\;\;\;\;\;\;\;\;\;\;\;\;\;\;\;\;\;\;\;\;\;\;\;\;\;\;\;\;\;\;\;\;\;\;\;\;\;\;\;\;\;\;\;\;\;\;\;\;\;\;\;\;\;\;\;\;\;\;\;\;\;\;\;\;\;\;\;\;\;\;\;\;\;\;\;\;\;\;\;\;\;\;\;\;\;\;\;\;\;\;\;\;\;\;\;\;\;\;\;\;\;\;\;\;\;\;\;\;\;\;\;\;\text{\ensuremath{\blacksquare}\;\;\;\;}$


\paragraph*{Theorem 2.}

If samples from system $x\in\left\{ 1\ldots,k\right\} $ are identically
distributed and independent, over time and across alternatives, then
$\mbox{lim}_{\delta\rightarrow0}\mathbb{P}\left(CS_{\delta}\right)\geq P*$
provided $\mu_{k}-\mu_{k-1}\geq\delta$. We suppose $B_{1}=\cdots=B_{k}=1$
and $c\neq\frac{1}{k}$.


\paragraph*{Proof.}

Suppose $X_{1},\ldots,X_{k}$ are the observations of the systems
$1,\ldots,k$, respectively. Consider, $\hat{X}_{i}=X_{i}-\mu_{i}e$
if $i\neq k$ and $\hat{X}_{k}=X_{k}-\mu_{k}e+\delta e$. Then
\begin{eqnarray*}
\underline{lim}_{\delta\rightarrow0}\mathbb{P}_{\text{\ensuremath{\mu}}}\left(CS_{\delta}\mid\mathbf{X}\right) & = & \underline{lim}_{\delta\rightarrow0}\mathbb{P}_{\left(0,\ldots,0,\delta\right)}\left(CS_{\delta}\mid\hat{\mathbf{X}}\right)\\
 & = & \underline{lim}_{\delta\rightarrow0}\mathbb{P}\left(g\left(W\right)=1\right)\\
 & \geq & P^{*}.
\end{eqnarray*}


$\;\;\;\;\;\;\;\;\;\;\;\;\;\;\;\;\;\;\;\;\;\;\;\;\;\;\;\;\;\;\;\;\;\;\;\;\;\;\;\;\;\;\;\;\;\;\;\;\;\;\;\;\;\;\;\;\;\;\;\;\;\;\;\;\;\;\;\;\;\;\;\;\;\;\;\;\;\;\;\;\;\;\;\;\;\;\;\;\;\;\;\;\;\;\;\;\;\;\;\;\;\;\;\;\;\;\;\;\;\;\;\;\;\;\;\;\;\;\;\;\;\;\;\;\;\;\;\;\;\;\;\;\;\;\;\;\;\;\;\text{\ensuremath{\blacksquare}}$
\begin{thebibliography}{References}
\bibitem{key-3}Billingsley, P. 1968. \textit{Convergence of Probability
Measures. }John Wiley and Sons. New York. 

\bibitem{key-5}Frazier, P. I. A Fully Sequential Elimination Procedure
for Indifference-Zone Ranking and Selection with Tight Bounds on Probability
of Correct Selection. \textit{Operations Research,} to appear.

\end{thebibliography}
\[
\]
\[
\]

\end{document}
