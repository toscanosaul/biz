%% LyX 2.1.2.1 created this file.  For more info, see http://www.lyx.org/.
%% Do not edit unless you really know what you are doing.
\documentclass[12pt,english]{article}
\usepackage[LGR,T1]{fontenc}
\usepackage[latin9]{inputenc}
\usepackage{geometry}
\geometry{verbose,tmargin=1.5cm,bmargin=1.5cm,lmargin=1.5cm,rmargin=1.5cm}
\usepackage{textcomp}

\makeatletter

%%%%%%%%%%%%%%%%%%%%%%%%%%%%%% LyX specific LaTeX commands.
\DeclareRobustCommand{\greektext}{%
  \fontencoding{LGR}\selectfont\def\encodingdefault{LGR}}
\DeclareRobustCommand{\textgreek}[1]{\leavevmode{\greektext #1}}
\DeclareFontEncoding{LGR}{}{}
\DeclareTextSymbol{\~}{LGR}{126}

\makeatother

\usepackage{babel}
\begin{document}

\title{WSC15 paper}

\maketitle

\section*{Reviewer 1.}
\begin{enumerate}
\item \textit{Frazier 2014 was cite lots of times, modify the cite commands
accordingly to make sure the correct format. }Now, it should be fine.
\item \textit{Second corrections. }They're now correct.
\item \textit{Use Section 2, 3, 4 instead just 2,3,4 when referring to those
sections.} I changed this part, but I'm not sure if we have to write
the word section before the number.
\item Page 3, 3rd paragraph. Now it is correct.
\item Page 3, 4th paragraph. Now it is correct.
\item Page 3, 1st paragraph in Section 3. Now it is correct.
\item 3rd paragraph in Section 3. I think that now it is clear. \textit{Says
100 is the recommended value for n\_0, but later in the algorithm
and experiments used something else. }I added that this value is recommended
when the variances are unknown.
\item End of page 3. Now it should be clear.
\item Section 4, 3rd paragraph. Now it is correct.
\item Section 6. Now it is correct.
\end{enumerate}

\section*{Reviewer 2.}
\begin{enumerate}
\item \textit{It seems that the choice of n0 is relatively arbitrary. In
practice, can problem properties be used to obtain some understanding
of how large/small n0 should be? }I don't now if we should say something
in section 3 about this because it is difficult to say something if
we don't now the variances of the alternatives.
\item \textit{It does seem to me that we choose a far too stringent threshold,
the resulting problem may be rendered infeasible since \textmu {[}k
are exogenously specified and presumably unavailable a prior and \textgreek{d}
is a user-specified parameter. In re-reading the paper, it seems that
the alternatives may have nearly the same means, in which case, the
likelihood of infeasibility may be modest. }Should we say that if
the configuration is not in the preference zone, then we can't say
anything about the alternative selected?
\item \textit{While the BIZ procedure is asymptotically valid, naturally
in practice, the schemes have to be terminated after a finite number
of samples. Are there error bounds available, even in a restricted
regime, that can govern the choice of when the procedure may be terminated.
}I said in section 3 that the algorithm will end in a finite number
of steps. However, I think there are no error bounds available if
we just stop the algorithm before time. What should we say about this?
\item \textit{In the numerical experiments, the choice of \textmu \textquoteright s
and \textgreek{d} ensures feasibility. Suppose \textmu 100 is set
to \textgreek{d} + 1, then unless I have misunderstood how the scheme
works, the threshold constraint would render the problem infeasible.
In this case, how would the scheme respond? }This can't happen because
we are supposing that $\mu=a\delta$. Should we say something else?
\item \textit{It might be useful to see how the BIZ procedure is indeed
less conservative in practice compared to its counterparts in the
R\&S literature. The authors claim that the proposed procedure requires
less samples \textendash{} presumably this is only an empirical observation.
If so, please emphasize that this is the case; else it might be worth
pointing to the formal statement that guarantees this claim. }This
observation is based on empirical observations and that the BIZ procedure
is less conservative. Should we say something else in the introduction
to make this more clear?
\item \textit{Finally, some quick observations regarding the references.
}I think that now they are fine. I had to cite the two editions of
Billingsley's book because I used the two editions. \end{enumerate}

\end{document}
