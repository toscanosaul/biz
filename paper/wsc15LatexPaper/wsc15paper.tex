%**************************************************************************
%*
%*  Paper: ``INSTRUCTIONS FOR AUTHORS OF LATEX DOCUMENTS''
%*
%*  Publication: 2015 Winter Simulation Conference Author Kit
%*
%*  Filename: wsc15paper.tex
%*
%*  Date: January 31, 2001   Time:  9:45 PM
%*      BASE of current version: Feb 01, 2010 (primary WSC'10 LaTeX file)
%*
%*  Word Processing System: TeXnicCenter and MiKTeX
%*
%*
%*  All files need the following
\input{wsc15style.tex}     % download from author kit.  Style files for wsc formatting. Don't remove this line - required for generating the final paper!

\documentclass{wscpaperproc}
\usepackage{latexsym}
%\usepackage{caption}
\usepackage{graphicx}
\usepackage{mathptmx}

%
%****************************************************************************
% AUTHOR: You may want to use some of these packages. (Optional)
\usepackage{amsmath}
\usepackage{amsfonts}
\usepackage{amssymb}
\usepackage{amsbsy}
\usepackage{amsthm}
% Private macros here (check that there is no clash with the style)
\newcommand{\lambdahat}{\widehat{\lambda}}
\newcommand{\xtilde}{\tilde{x}}
\newcommand{\qhat}{\widehat{q}}
\newcommand{\as}{\ a.s.}
\newcommand{\zap}[1]{}
\newcommand{\Ncal}{\mathcal{N}}
\newcommand{\sigmahat}{\hat{\sigma}}
\newcommand{\Nat}{\mathbb{N}}
\renewcommand{\P}{\mathbb{P}}
\newcommand{\Pb}[2]{\mathbb{P}_{#1}\left\{#2\right\}}
\newcommand{\Ybar}{\overline{Y}}
\newcommand{\Z}{\mathbb{Z}}
\newcommand{\Q}{\mathbb{Q}}
\newcommand{\R}{\mathbb{R}}
% \newcommand{\argmax}{\operatornamewithlimits{arg\,max}}
% \newcommand{\argmin}{\operatornamewithlimits{arg\,min}}
\newcommand{\g}{\,\vert\,}
\newcommand{\ind}[1]{1_{\left\{#1\right\}}}
\newcommand{\Fcal}{\mathcal{F}}
\newcommand{\Gcal}{\mathcal{G}}
\newcommand{\E}{\mathbb{E}}
\newcommand{\xhat}{\hat{x}}
\newcommand{\xwig}{\tilde{x}}
\newcommand{\PZ}{\mbox{PZ}}
\newcommand{\PCS}{\mbox{PCS}}
\newcommand{\PGS}{\mbox{PGS}}
\newcommand{\uLFC}{\underline{u}}
\newcommand{\e}[1]{\left\{ #1 \right\}}
\newcommand{\T}{\mathbb{T}} % Time index set.
\newcommand{\CS}{\mbox{CS}}
\newcommand{\Ft}{\mathcal{F}_t}
\newcommand{\F}[1]{\mathcal{F}_{#1}}
\newcommand{\Ftau}{\mathcal{F}_\tau}
\newcommand{\Rcal}{\mathcal{R}}
\newcommand{\upthresh}{P}
\newcommand{\xstar}{X^*}
\newcommand{\cmax}{1-(P^*)^{\frac1{k-1}}}
\newcommand{\NewN}{M}
\newcommand{\thetavec}{\vec{\theta}}
\newcommand{\uvec}{\vec{u}}
\newcommand{\sigmavec}{\lambda}
\newcommand{\sigmacom}{\sigma}
\newcommand{\sigmascal}{\lambda}
\newcommand{\PstarB}{\mathcal{P}^*_B}
\newcommand{\Ymod}{Y'}
\newcommand{\ceil}{\mathrm{ceil}}
% How to state the common variance assumption at the beginning of lemmas, propositions and theorems
\newcommand{\homog}{Suppose $\lambda^2_x = \sigma^2>0\ \forall x$.  }
\newcommand{\algref}[1]{Alg.~\ref{#1}}


\usepackage{algorithm,algorithmic}
%****************************************************************************



%
%****************************************************************************
% AUTHOR: If you do not wish to use hyperlinks, then just comment
% out the hyperref usepackage commands below.

%% This version of the command is used if you use pdflatex. In this case you
%% cannot use ps or eps files for graphics, but pdf, jpeg, png etc are fine.

\usepackage[pdftex,colorlinks=true,urlcolor=blue,citecolor=black,anchorcolor=black,linkcolor=black]{hyperref}

%% The next versions of the hyperref command are used if you adopt the
%% outdated latex-dvips-ps2pdf route in generating your pdf file. In
%% this case you can use ps or eps files for graphics, but not pdf, jpeg, png etc.
%% However, the final pdf file should embed all fonts required which means that you have to use file
%% formats which can embed fonts. Please note that the final PDF file will not be generated on your computer!
%% If you are using WinEdt or PCTeX, then use the following. If you are using
%% Y&Y TeX then replace "dvips" with "dvipsone"

%%\usepackage[dvips,colorlinks=true,urlcolor=blue,citecolor=black,%
%% anchorcolor=black,linkcolor=black]{hyperref}
%****************************************************************************



		



%
%****************************************************************************
%*
%* AUTHOR: YOUR CALL!  Document-specific macros can come here.
%*
%****************************************************************************

% If you use theoremes
\newtheoremstyle{wsc}% hnamei
{3pt}% hSpace abovei
{3pt}% hSpace belowi
{}% hBody fonti
{}% hIndent amounti1
{\bf}% hTheorem head fontbf
{}% hPunctuation after theorem headi
{.5em}% hSpace after theorem headi2
{}% hTheorem head spec (can be left empty, meaning `normal')i

\theoremstyle{wsc}
\newtheorem{theorem}{Theorem}
\renewcommand{\thetheorem}{ \arabic{theorem}}
\newtheorem{corollary}[theorem]{Corollary}
\renewcommand{\thecorollary}{\arabic{corollary}}
\newtheorem{definition}{Definition}
\renewcommand{\thedefinition}{\arabic{definition}}


%#########################################################
%*
%*  The Document.
%*
\begin{document}

%***************************************************************************
% AUTHOR: AUTHOR NAMES GO HERE
% FORMAT AUTHORS NAMES Like: Author1, Author2 and Author3 (last names)
%
%		You need to change the author listing below!
%               Please list ALL authors using last name only, separate by a comma except
%               for the last author, separate with "and"
%
\WSCpagesetup{Toscano, Frazier}

% AUTHOR: Enter the title, all letters in upper case
\title{ASYMPTOTIC VALIDITY OF A FULLY SEQUENTIAL ELIMINATION PROCEDURE FOR INDIFFERENCE-ZONE RANKING AND SELECTION WITH TIGHT BOUNDS ON PROBABILITY OF CORRECT SELECTION\LaTeX}

% AUTHOR: Enter the authors of the article, see end of the example document for further examples
\author{Saul Toscano\\ [12pt]
School of Operations Research and Information Engineering\\
Cornell University \\ 
206 Rhodes Hall \\
Ithaca, NY 14853, USA\\
% Multiple authors are entered as follows.
% You may also need to adjust the titlevbox size in the preamble - search for titlevboxsize
\and
Peter I. Frazier\\ [12pt]
School of Operations Research and Information Engineering \\
Cornell University \\ 
206 Rhodes Hall \\
Ithaca, NY 14853, USA
}






\maketitle

\section*{ABSTRACT}
We consider the indifference-zone (IZ) formulation of the ranking
and selection problem. Conservatism leads classical IZ procedures
to take too many samples in problems with many alternatives. The Bayes-inspired
Indifference Zone (BIZ) procedure, proposed in Frazier (2014), is
less conservative than previous procedures, but its proof of validity
requires strong assumptions. In this paper, we present a new proof
of asymptotic validity that relaxes these assumptions.

\section{INTRODUCTION}
\label{sec:intro}

One common problem in simulation is that of choosing the best among
several simulated systems. The problem of deciding how many samples
to use from each alternative to support our selection as the best
is the ranking and selection problem. An efficient solution to this
problem has to balance between the time spent simulating and the quality
of the selection.

This paper will consider the indifference-zone (IZ) formulation of
the ranking and selection problem, in which we choose the best system
with probability larger than a pre-specified threshold, whenever the
distance between the best system and the others is sufficiently large.
We say that a sampling procedure having this property satisfies the
IZ guarantee and the set of system configurations under which the
best alternative is better than the second best by at least some given
$\delta>0$ is called the preference zone (PZ). The seminal work dates
back to Bechhofer (1954), with early work compiled in the monograph
Bechhofer et al. (1968). The progress in the area has been summarized
in Bechhofer et al. (1995), Swisher et al. (2003), Kim and Nelson
(2006, 2007) and Frazier (2014).

The goal of an IZ algorithm is to take as few samples as possible
while the IZ guarantee is satisfied. The first IZ procedures presented
in Bechhofer (1954), Paulson (1964), Fabian (1974), Rinott (1978),
Hartmann (1988, 1991), Paulson (1994) satisfy the IZ guarantee, but
they usually take too many samples when there are many alternatives,
in part because their probability of correct selection (PCS) is much
larger than the probability specified by the user. A reason of this
is the use of the Bonferroni's inequality which leads to sample more
than necessary. More recent algorithms in Kim and Nelson (2001), Goldsman
et al. (2002), Hong (2006) improve the performance but they still
use the Bonferroni's inequality, and so the methods are inefficient
when there are several systems. Procedures in Kim and Dieker (2011),
Dieker and Kim (2012) do not use the Bonferroni's inequality only
when compare groups of three alternatives. 

Since the classic IZ procedures take too many samples with many alternatives,
these methods are unpopular when there are more than a few hundred
alternatives. However, Frazier (2014) presented a new sequential elimination
IZ procedure, called BIZ (Bayes-inspired Indifference Zone), whose
lower bound on worst-case probability of correct selection in the
preference zone is tight in continuous time, and almost tight in the
discrete time. In numerical experiments, the number of samples required
by BIZ is significantly smaller than that of the most popular IZ procedures,
especially on problems with many alternatives. Unfortunately, the
proof that the BIZ procedure satisfies the IZ guarantee for the discrete-time
case assumes that variances are known and have an integer multiple
structure which is not very realistic. In practice, variances are
unknown. However, asymptotically we can use a central limit theorem
that allows us to prove the asymptotic validity of the BIZ procedure
for the discrete-time case. Moreover, we only need to assume that
the systems are independent, identically distributed and have finite
variance.

Kim et al. (2006) also proves the asymptotical validity of a IZ procedure.
Our proof is larger because the BIZ procedure is more sophisticated
and the bound for the PCS is tighter. To prove the case when the variances
are known, we use a theorem for Ergodic processes that shows how to
standardize the output data to make them behave like Brownian motion
processes in the limit. We also use an extension of the Continuous
Mapping Theorem (Theorem 5.5 of Billingsley 1968) to see that the
algorithm behaves like a sequential elimination IZ procedure with
a Brownian motion process instead of the standardized sum of the output
data in the limit, and then we use the results of the paper of Frazier
\cite{key-5} to prove the validity of this algorithm in the limit.
Finally, we use a random change of time argument to prove the case
when the variances are unknown.

This paper is organized as follows: In $\text{ß}2$, we recall the
indifference-zone ranking and selection problem. In $\text{ß}3$,
we recall the Bayes-inspired IZ (BIZ) procedure. In $\text{ß}4$,
we present the proof of the validity of the algorithm when the variances
are known. In $\text{ß}5$, we prove the case when the variances are
unknown. In $\text{ß}6$, we present some simulations showing the
effectiveness of the algorithm. In $\text{ß}7$, we conclude.



\section{INDIFFERENCE-ZONE RANKING AND SELECTION}

Ranking and Selection is a procedure for selecting the best system
among a finite set of alternatives, i.e. the system with the largest
mean. The method selects a system as the best based on the samples
that are observed sequentially over the time. If the best system is
selected, we say that the procedure has made the \emph{correct selection}
(CS). We define the \emph{probability of correct selection} as 
\[
\mbox{PCS}\left(\mu,\lambda\right)=\mathbb{P}_{\mu,\lambda}\left(\hat{x}\in\mbox{arg max}_{x}\mu_{x}\right)
\]
where $\hat{x}$ is the alternative chosen by the procedure and $\mathbb{P}_{\mu,\lambda}$
is the probability measure under which samples from system $x$ have
mean $\mu_{x}$ and finite variance $\lambda_{x}^{2}$.

In the Indifference-Zone Ranking and Selection, the procedure is indifferent
in the selection of a system whenever the means of the populations
are nearly the same. Formally, let $\mu=\left[\mu_{k},\ldots,\mu_{1}\right]$
be the vector of the true means, the \emph{indifference zone} is defined
as the set $\left\{ \mu\in\mathbb{R}^{k}:\mu_{\left[k\right]}-\mu_{\left[k-1\right]}<\delta\right\} $.
The complement of the indifference zone is called the \emph{preference
zone} (PZ) and $\delta>0$ is called the indifference zone parameter.
We say that a procedure meets the \emph{indifference-zone (IZ) guarantee
}at $P^{*}\in\left(1/k,1\right)$ and $\delta>0$ if
\[
\mbox{PCS}\left(\mu,\lambda\right)\geq P^{*}\mbox{ for all }\mu\in\mbox{PZ}\left(\delta\right).
\]
We assume $P^{*}>1/k$ because IZ guarantees can be meet by choosing
$\hat{x}$ uniformly at random from among $\left\{ 1,\ldots,k\right\} $.

\section{THE BAYES-INSPIRED IZ (BIZ) PROCEDURE}

BIZ is an elimination procedure. This procedure maintains a set of
alternatives that are in contention, and it takes samples from each
alternative in this set at each point in time. At beginning, all alternatives
are in contention, and over the time alternatives are eliminated.
The procedure ends when there is only one alternative in the contention
set and this remain alternative is chosen as the best. 

Frazier (2014) showed that the BIZ procedure with known common variance
satisfies the IZ guarantee when the systems follow the normal distribution
, with tight bounds on worst-case preference-zone in continuous time.
He also proved that this procedure retains the IZ guarantee when the
systems follow the normal distribution, and the variances are known
and are integer multiples of a common value. The continuous time version
of this procedure also satisfies the IZ guarantee, with a tight worst-case
preference-zone PCS bound.

The discrete-time BIZ procedure for unknown and/or heterogeneous sampling
variances is given in Alg. 2. It takes a variable number of samples
from each alternative, and $n_{tx}$ is this number. This algorithm
depends on a collection of integers $B_{1},\ldots,B_{k}$, $P^{*},c,\delta$
and $n_{0}$. $n_{0}$ is the number of samples to use in the first
stage of samples, and $100$ is the recommended value for $n_{0}$.
$B_{x}$ controls the number of samples taken from system $x$ in
each stage. 

For each $t$, $x\in\left\{ 1,\ldots,k\right\} $, and subset $A\subset\left\{ 1,\ldots,k\right\} $,
we define a function
\[
q_{tx}\left(A\right)=\mbox{exp}\left(\delta\beta_{t}\frac{Z_{tx}}{n_{tx}}\right)\left/\sum_{x'\in A}\mbox{exp}\left(\delta\beta_{t}\frac{Z_{tx'}}{n_{tx'}}\right),\right.\mbox{ }\beta_{t}=\frac{\sum_{x'\in A}n_{tx'}}{\sum_{x'\in A}\hat{\lambda}_{tx'}^{2}}
\]
where $\hat{\lambda}_{tx'}^{2}$ is the sample variance of all samples
from alternative $x$ thus far and $Z_{tx}=Y_{n_{tx},x}$.

\vspace{10mm}
   
\paragraph{Algorithm: Discrete-time implementation of BIZ, for unknown and/or heterogeneous variances.}
\vspace{0.1mm}    
\begin{algorithmic}[1]   
\label{alg:hetero-BIZ}   
\REQUIRE $c \in [0,\cmax]$, $\delta>0$, $P^*\in(1/k,1)$, $n_0\ge0$ an integer, $B_1,\ldots,B_k$ strictly positive integers.  Recommended choices are $c=\cmax$, $B_1=\cdots=B_k=1$ and $n_0$ between $10$ and $30$.     If the sampling variances $\lambda^2_x$ are known, replace the estimators     
$\lambdahat^2_{tx}$ with the true values $\lambda^2_x$, and set $n_0=0$.     

\STATE For each $x$, sample alternative $x$ $n_0$ times and set $n_{0x} \leftarrow n_0$.     
Let $W_{0x}$ and $\lambdahat^2_{0x}$ be the sample mean and sample variance respectively of these samples.     Let $t\leftarrow 0$.     
\STATE Let $A \leftarrow \{ 1,\ldots, k\}$, $\upthresh \leftarrow P^*$.
\WHILE{$x\in\mbox{max}_{x\in A} q_{tx}\left(A\right)<P$}
\WHILE{$\mbox{min}_{x\in A} q_{tx}\left(A\right) \le c$}
 \STATE Let $x\in\mbox{arg min}_{x\in A} q_{tx}\left(A\right)$.
    \STATE Let $\upthresh \leftarrow \upthresh/(1-q_{tx}\left(A\right))$.     
\STATE Remove $x$ from $A$.
\ENDWHILE
  \STATE Let $z \in \mbox{arg min}_{x\in A} n_{tx} / \lambdahat^2_{tx}$.     
\STATE For each $x\in A$, let      $n_{t+1,x} = \ceil\left( \lambdahat^2_{tx} (n_{tz} + B_z) / \lambdahat^2_{tz} \right)$.     \STATE For each $x\in A$, if $n_{t+1,x}>n_{tx}$, take $n_{t+1,x}-n_{tx}$ additional samples from alternative $x$.  Let $W_{t+1,x}$ and $\lambdahat^2_{t+1,x}$ be the sample mean and sample variance respectively of all samples from alternative $x$ thus far.    
\STATE Increment $t$.
 \ENDWHILE
  \STATE Select $\xhat \in\mbox{arg max}_{x\in A} Z_{tx} / n_{tx}$ as our estimate of the best.

   
\end{algorithmic} 

\vspace{1mm}
This algorithm generalizes the BIZ procedure with known common variance. In that case, we have that$B_1=\cdots=B_k=1$ and $n_{tx}=t$. The algorithm 2 can be generalized to the continuous case (See appendix B and Frazier (2014)). This procedure is a slightly modification of the original BIZ procedure 

\section{ASYMPTOTIC VALIDITY WHEN THE VARIANCES ARE KNOWN}


\subsection{Language}

The paper should be prepared using U.S. English in the interest of consistency across the proceedings. Please carefully check the spelling of words before you submit your paper. There are spell checkers for \LaTeX\ as well.
Some examples of software which supports spell checking are TexnicCenter, TexMaker, and TexClipse.

\subsection{Objectivity}
The content of the paper should be objective and without any appearance of commercialism.  In general, comparisons of commercial software should be avoided unless they are central to the topic.  If a comparison of commercial software is included, it should be based on objective analysis that includes criteria, description of ranking methodology on each criteria, and the rankings themselves to arrive at the conclusion.
If an approach other than a detailed objective analysis is used to select the simulation software used for the study being reported, such as, availability of the software, or the familiarity of the analyst with the software, it should be clearly identified.

\subsection{Paper Submission}
You will submit both the source files (text, graphics, bib) for your paper, as well as the paper in Portable Document Format ({\tt .pdf}), electronically at the \href{http://www.wintersim.org}{conference website} \cite{WSC}.

Your source file(s) should be submitted either as a single {\tt .tex} file if you only use
one file, or as a zipped folder containing ALL files needed to compile.

{\em In addition to the source file ({\tt .tex}), include all files necessary to generate your paper, including figures, bibliography files, and any non-standard packages that you use (please try to avoid the use of non-standard packages).} Section~\ref{sec:graphics} discusses the inclusion of figures in your paper, as well as requirements for the submission of figure files.
Section~\ref{sec:submitbib} discusses requirements for submitting bibliography files ({\tt .bib}), if you choose to use \BibTeX. If you use any non-standard packages, then please include them as well, since they will be needed to generate the final pdf versions of your paper.
The final pdfs are generated by the conference proceedings editors.

Please do not include folders like (.svn (Subversion folders) or \_\_MACOSX) and try to avoid submitting unnecessary additional files.
Please do not include previous versions of your tex file.

Zipped folders can be
produced using the {\tt tar} and {\tt compress} commands on Unix systems, by right-clicking on
a folder in Windows XP or later (``send to zip folder''), by right-clicking and using the compress command in Mac OS, or by using an application such as,
Winzip, Ultimatezip or WinRAR.
Make sure that the files submitted can be compiled without any error.
The {\tt .pdf} file submitted by you allows the editors to ensure that the edited version of your paper conforms reasonably to the appearance
that you intended.

You will also need to transfer the copyright of your article to the WSC using the copyright transfer form that will be available via the conference web site at the appropriate time. {\em In order for your paper to be published by the WSC, you must complete the transfer of copyright.}
When you have successfully transferred the copyright, you will receive a receipt ({\tt .pdf}) .
Please email a copy of this receipt to \href{mailto://wsc15proceedings@gmail.com}{the proceedings editors}. If you are unable to satisfy these requirements then you should contact \href{mailto://wsc15proceedings@gmail.com}{the proceedings editors}.

\subsection{Length Constraints}

\subsubsection{The Abstract and Keywords}
The abstract should be at most 150 words. Since abstracts of all papers accepted for publication in the proceedings will also appear in the final program, the length limit of 150 words for each abstract will be strictly enforced. The abstract should consist of a single paragraph, and it should not
contain references or mathematical symbols. Do not include a list of keywords as they are not used in the WSC proceedings.

\subsubsection{Length of the Paper}
The page size in the proceedings is 8.5 inches by 11 inches (21.6 cm by 27.9 cm). The overall length of the paper should be at least 5 proceedings pages. Papers should be at most 12 pages, except for introductory tutorials, advanced tutorials, and panel sessions, for which the limit is 15 pages.

\subsubsection{Font Specification and Spacing}
The paper should be set in the Times New Roman font using a 11-point font size.
These settings are automatically applied by the class file for the WSC proceedings used for this document.
Please note that the proceedings publisher will convert all papers to the Times New Roman font -- thus you are asked not to change these settings in your paper.
If you want to use bold Greek symbols you should use the {\tt bm} package.
The paper should be single spaced---that is, 6 lines per inch.

\subsubsection{Margins}
The width of the text area is 6.5 inches (16.0 cm). The left and right margins should be 1 inch (2.54 cm) on each page. Except for the first page, the top and bottom margins should be 1 inch (2.54 cm).

\subsubsection{Justification}
Headings of sections, subsections, and subsubsections should be left-justified. One-line captions for figures or tables should be centered.
A multiline caption for a figure or table should be fully justified. All other text should be fully justified across the page (that is, the text should line up on the right-hand and left-hand sides of the page).

\subsection{Headings of Sections, Subsections, and Subsubsections}
Section, subsection, and subsubsection headings should appear flush left, set in the bold font style, and numbered as shown in this document. The WSC style will take care of this if not modified.
The headings for the Abstract, Acknowledgments, References and Author Biographies sections are not numbered.
To suppress the section numbers, use the {\tt $\backslash$section*\{\}} command.
Section headings should be set in {\bf FULL CAPITALS LIKE THIS PHRASE}, while subsection and subsubsection headings should be {\bf Capitalized
in Headline Style like This Phrase}. The WSC style will take care of inserting one blank line before and after each heading.

\subsubsection{Paragraphs}
The first paragraph after a heading should not be indented; all other paragraphs should be indented by 0.25 inches. Do not insert additional space between paragraphs. The WSC style will take care of this if not modified.

\subsubsection{Footnotes}
Do not use footnotes; instead incorporate such material into the text directly or parenthetically.

\subsubsection{Page Numbers}
Do not include page numbers. Page numbers are added when the final pdfs are created.

\section{FORMATTING THE FIRST PAGE}

\subsection{Running Heads}
The running head in the upper left-hand corner of the first page should read {\vspace{3pt}\newline \noindent \em \currentCaption \newline \currentEditors, eds. \newline}
\vskip 8pt\noindent This initial running head is left-justified and set in the 11-point italic font style.
It is automatically provided by the class file.

Running heads on the second and subsequent pages should contain the last names of the authors, centered and set in the 11-point italic font style.
For example, running heads for papers with varying numbers of authors would appear like {\em Yilmaz} (single author), or {\em Yilmaz and Chan} (two authors), or {\em Yilmaz, Chan, and Moon} (three authors), or {\em Yilmaz, Chan, Moon, and Roeder} (four authors). These are created by using the macro\newline\vskip 1ex
\noindent \begin{verbatim}
\WSCpagesetup{LastName1, LastName2, and LastNameLastAuthor}

\end{verbatim}
defined in the class file.
Please use this macro to set up the running heads, as it sets further parameters important for the correct layout of the document.

The author names are listed in the same order as they appear on the title page, which is the same order the author biographies are provided.
These entries {\bf do} need to be changed by the authors in the {\tt $\backslash$WSCpagesetup} command in the source for this file.
Please give all author names, do not leave out any author names, and do not use et al.


\subsection{Title and Authors}
Center the title of the paper across the page and set it in bold {\bf FULL CAPITALS} so that the top edge of the title begins 1.5 inches from the top of the page.
The correct placement is automatically done by the class file as well.
Just use the {\tt $\backslash$title} and {\tt $\backslash$maketitle} commands as it is done in the source of this document.
Multiline titles should have about the same amount of text on each line.

There should be 2 blank lines between the title and the authors' names (will be inserted by the class file if the {\tt $\backslash$author}, {\tt $\backslash$title}, and {\tt $\backslash$maketitle} commands are used.

Each author's name should be capitalized and centered on a new line, with the author's first name first and no job title or honorific.
Insert 1 blank line between the author's name and address. The organization or institution that the author is affiliated to should be typed first.
Next type the complete street address, without abbreviations, followed by the city, standard two-letter state or province abbreviation, postal code, and country.
The address should be centered and capitalized, except for the country, which should be set in FULL CAPITALS. Do not include email addresses on the cover page; these are provided in the author biographies (See the first page of these instructions.)
For papers with multiple authors, the authors should be listed in order of decreasing contribution, with authors from the same institution grouped together if possible.
Different formats for multiple authors are shown as examples in Figures~\ref{fig: 2 same} through \ref{fig: 4 different} at the end of this document.
There should be 2 blank lines between the author names and the text of the paper.

You should use the {\tt $\backslash$author} command to enter author names, separated using the command {\tt $\backslash$and} --- see the source for this document.

\section{FORMATTING SUBSEQUENT PAGES}
For the remaining pages, the top margin should be 1 inch (2.5 cm).

\subsection{Mathematical Expressions in Text and in Displays}
Display only the most important equations, and number only the displayed equations that are explicitly referenced in the text.
To conserve space, simple mathematical expressions such as $\bar Y = n^{-1} \sum_{i=1}^n Y_i$ may be incorporated into the text.
Mathematical expressions that are more complicated or that must be referenced later should be displayed, as in
$$s^2 = \frac 1 {n-1} \sum_{i=1}^n (Y_i - \bar Y)^2.$$

If a display is referenced in the text, then enclose the equation number in parentheses and place it flush with the right-hand margin of the
column. For example, the quadratic equation has the general form

\begin{equation} \label{eq:quadratic}
ax^2 + bx + c = 0, \mbox{ where } a \ne 0.
\end{equation}

In the text, each reference to an equation number should also be enclosed in parentheses. For example, the solution to (\ref{eq:quadratic}) is given in (\ref{eq: quadratic sol}) in Appendix \ref{app:quadratic}.

If the equation is at the end of a sentence, then you should end the equation with a period. If the sentence in question continues beyond the equation, then you should end the equation with the appropriate punctuation---that is, a comma, semicolon, or no punctuation mark.

\subsection{Displayed Lists}
A displayed list is a list that is set off from the text, as opposed to a run-in list that is incorporated into the text. The bulleted list given below provides more information about the format of a displayed list.

\begin{itemize}
	\item Use standard bullets instead of checks, arrows, etc.\ for bulleted lists.
	\item For numbered lists, the labels should not be arabic numerals enclosed in parentheses because such labels cannot be distinguished from equation numbers.
\end{itemize}

\subsection{Definitions and Theorems}
Definitions, theorems, propositions, etc. should be formatted like a normal paragraph with a boldface heading as shown in the examples below. Number
these items separately and sequentially. You may choose not to separately number theorems, propositions, corollaries, etc., as opposed to the example below where corollaries and theorems are numbered together. Search the source of this document to see how these environments were defined. The key
command is {\tt $\backslash$newtheorem}. Do not use a period after the definition, theorem, corollary or proposition number.

\begin{definition}
In colloquial New Zealand English, the term {\em dopey mongrel} is used to refer to someone who has exhibited less than stellar intelligence.
\end{definition}

\begin{theorem}
If a proceedings editor from New Zealand accidentally deletes his draft of the author kit shortly after completing it, he would be considered to be a dopey mongrel.
\end{theorem}

\begin{corollary}
One of the proceedings editors is a dopey mongrel.
\end{corollary}

\subsection{Figures and Tables}
\label{sec:graphics}
Figures and tables should be centered within the text and should not extend beyond the right and left margins of the paper.
Figures and tables can make use of color since the WSC produces electronic proceedings.
However, try to select colors that can be differentiated when printing in black and white in consideration of vast majority of people using such printers.
Figures and tables are numbered sequentially, but separately, using arabic numerals.

Each table should appear in the document after the paragraph in which the table is first referenced. However, if the table is getting split across pages, it is okay to include it after a few paragraphs from its first reference.
One-line captions are centered, while multiline captions are left justified.
The captions appear {\em above} the table. See Tables \ref{tab: first} and \ref{tab: second} for examples.

\begin{table}[htb]
\centering
\caption{Table captions appear above the table, and if they are longer than one line they are left justified. Captions are written using normal sentences with full punctuation. It is fine to have multiple-sentence captions that help to explain the table.\label{tab: first}}
\begin{tabular}{rll}
\hline
Creature & IQ & Diet\\ \hline
dog & 70 & anything\\
cat & 75 & almost nothing\\
human & 60 & ice cream \\
dolphin & 120 & fish fillet\\
\hline
\end{tabular}
\end{table}

\begin{table}[htb]
\centering
\caption{Counting in Maori.\label{tab: second}}
\begin{tabular}{r|l}
English & Maori \\ \hline
one & tahi \\
two & rua \\
three & toru \\
four & wha \\
\end{tabular}
\end{table}

Each figure should appear in the document after the paragraph in which the figure is first referenced. One-line captions are centered,
while multiline captions are left justified. Figure captions appear below the figure. See Figures \ref{fig: tahi} and \ref{fig: rua} for examples.

\begin{figure}[htb]
{
\centering
%\includegraphics[width=0.9\columnwidth]{MathExpandExpression.jpg}
\includegraphics[width=0.50\textwidth]{MathExpandExpression}
\caption{An unusual answer to a question.\label{fig: tahi}}
}
\end{figure}

\begin{figure}[htb]
{
\centering
%\includegraphics[width=0.9\columnwidth]{puzzle.png}
\includegraphics[width=0.50\textwidth]{puzzle}
\caption{The area of the square is 64 squares, while that of the rectangle is 65 squares, yet they are made of the same pieces! How
is this possible? \label{fig: rua}}
}
\end{figure}

References to tables and figures identified by number are capitalized. For example, ``We see in Table 5 that...'' and ``We see in the previous table that...'' are both correct. Be sure to use the {\tt $\backslash$label} command within the figure or table environment and refer to the associated figure or table using {\tt Table$\sim \backslash$ref\{labelgiven\}}.
Please do not use hard coded figure/table numbers. This is error prone, and the references will not be hyperlinks.

Please ensure that your graphics files use standard fonts (Times New Roman, Symbol, etc.) or that those are embedded in the final figure files.
If they are not embedded, and if the font is not available on the editor's computer, then the font will not be included in the final PDF.
This may lead to a problem with displaying the final PDF file on computers without an appropriate font.
At best you select a format which allows to embed the fonts in all non bitmap figure files.

Including graphics files in your document can be complicated. Brace yourself! In general you have 2 options. Either (a) all of your files are {\tt .ps} or {\tt .eps} files, or (b) none of your files are of those types, and instead you use {\tt .jpg}, {\tt .png} or {\tt .pdf} files.
But there are tools to convert these formats into one another. The main difference between the formats is how they store the images and how well suited they are for specific graphics. You can choose between bitmap and vector graphics.
Bitmap graphics are well suited for photographs (jpg is very common here) or for screenshots (PNG is a lossless encoding in contrast to jpg, and is thus better suited for all those cases where you have sharp edges in your graphics).
Vector graphics are the encoding to be chosen for all kinds of drawings (diagrams, charts, ...). In contrast to bitmap formats, they can be scaled to any size without any loss of sharpness. This makes it possible to read such graphics even if two pages are printed on one sheet of paper, or if the documents are read electronically.
So what to choose for your \LaTeX\ document? As a rule of thumb you should always prefer PDF or PS and EPS.
In general these three encodings can contain both, bitmap and vector graphics. But there is no need (and no use) to convert your bitmaps to any of these.

If you follow Option (a), then you should use the outdated {\tt latex - dvips - ps2pdf} route for generating a pdf file. You may run into a problem if using both the {\tt hyperref} package and the {\tt graphicx} package; there seems to be a clash there.
In that case, you might either not use the {\tt hyperref} package and continue to use {\tt graphicx}, or continue to use the {\tt hyperref} package and use the {\tt epsfig} package in place of the {\tt graphicx} package.
If you persevere with {\tt hyperref} then be sure to use the appropriate version of the {\tt $\backslash$usepackage} command; see the preamble in the source of this file for details. See also Section~\ref{sec: hyper} below. It is important to note that if you remove the {\tt hyperref} package, then you have to deal with the correct formatting of hyperlinks on your own.

If you follow Option (b), then you must use the {\tt PDFLaTeX} command to generate your pdf file, as was done with this file. PDFLaTeX is meanwhile the standard --- so option (b) should become normal. The final file format is PDF.

Whatever option you choose: if you include figures via {\tt includegraphics}, then please do so without the file ending (e.g., skip .pdf, .ps, ...).

\subsection{Hyperlinks}
\label{sec: hyper}
A {\em hyperlink} specifies a web address (URL) or an email address.
The use of hyperlinks allows authors to give readers access to external electronic information, such as a dynamic simulation or animation.
The use of hyperlinks is at the discretion of the author(s).
But please note: hyperlinks (to web pages) might not work forever (web pages might be removed), and thus using hyperlinks intensively may make a paper (or parts thereof) less useful in future.
If enough information is provided in the main body of the paper to enable searching for the cited content in any case and if the inclusion of the web address does not hurt the appearance of the paper, then the web address can be included in the main body of the paper itself.

{\bf While the use of hyperlinked text is encouraged in the main body of the paper, it is recommended that corresponding web addresses and other identifying information should be provided in list of references.}
For example, instead of spelling out the web address of the conference website, one would refer to the  \href{http://www.wintersim.org}{conference website} \cite{WSC}, and the corresponding entry in the reference section will spell out the associated web address and other relevant information such as author(s) and/or the organization that published the content.
This would allow readers to search for the content using the author(s), organization, etc.\ in case the actual web-address is changed.  This also allows for a cleaner appearance of the main body of the paper.

Each hyperlink should be set in Times New Roman, 11-point font size.
Hyperlinks are {\em not} underlined.

A live hyperlink (or hot link)---that is, a hyperlink that will activate your web browser and take it to an external web site or that will activate your email software for sending a message to a specific email address---should be colored blue. You have already seen examples of such hyperlinks in this paper.
Non-live hyperlinks �that is, the hyperlinks that are included for the reader�s information but do not actually invoke the reader�'s web browser or email software, should be colored black.
To use live hyperlinks in a proceedings paper, use the {\tt hyperref} package. If you are using {\tt PDFLaTeX} to generate your pdf file then, as
was done for this file, you should add the following as the last {\tt $\backslash$usepackage} command in the preamble.\newline


\begin{verbatim}
\usepackage[pdftex,colorlinks=true,urlcolor=blue,citecolor=black,
anchorcolor=black,linkcolor=black]{hyperref}
\end{verbatim}\vspace{5 mm}

\noindent On the other hand, if you are using the traditional {\tt latex - dvips - ps2pdf} route, then users of MiKTeX or PCTeX for Windows should add the command\newline


\begin{verbatim}
\usepackage[dvips,colorlinks=true,urlcolor=blue,citecolor=black,
anchorcolor=black,linkcolor=black]{hyperref}
\end{verbatim}\vspace{5 mm}


\noindent as the last {\tt $\backslash$usepackage} command in the preamble, while users of Y\&Y TeX should add the command\newline


\begin{verbatim}
\usepackage[dvipsone,colorlinks=true,urlcolor=blue,citecolor=black,
anchorcolor=black,linkcolor=black]{hyperref}
\end{verbatim}\vspace{5 mm}


\noindent as the last {\tt $\backslash$usepackage} command in the preamble.
(In general the {\tt $\backslash$usepackage} command above that works for MiKTeX running on a Windows system should also work for most implementations of \LaTeX\ running on a Unix or Apple system.)
Thus the hypertext link \href{http://www.wintersim.org}{conference website} \cite{WSC} to the WSC website can be established by the command\newline

\begin{verbatim}
\href{http://www.wintersim.org}{conference website}
\end{verbatim}\vspace{5 mm}


\noindent This is especially important since WSC papers are filed in the IEEE Xplore digital library, which does not allow hyperlinks, so for that purpose the hyperlinks are removed.
Therefore it is recommended to add all hypertext references to the {\tt .bib} file and to refer to them from the text as it is done in the example above.
All live hyperlinks still appear in the CD of the proceedings and in other repositories.
If the authors use hyperlinked text in the main body of the paper, they must ensure that each hyperlink includes a citation (e.g., (WSC 2015) following the hyperlinked text ``conference website'' in Section~\ref{sec:intro}), a corresponding entry is provided in the list of references, and the associated web address displayed for the hyperlink is complete and correct so that a reader who has only a hard copy of the paper can still access the cited material by typing the relevant part of the displayed text of the hyperlink into the address bar of a web browser.
If the authors opt for including the web address in the main body of the text itself, they must ensure that the hyperlink is complete and correct for the same reason.

If you use the package {\tt hyperref} as suggested here, and if you use citation commands to handle references, then your citations will
become hyperlinks (as in this document).

\subsection{Citing a Reference}
To cite a reference in the text, use the author-date method. Thus, \citeN{chi89} could also be cited parenthetically \cite{chi89}.
For a work with four or more authors, use an abbreviated form. For example, a work by Banks, Carson, Nelson, and Nicol would be cited in one of the
following ways:
\shortciteN{bcnn:simulation} or \shortcite{bcnn:simulation}.

Parenthetical citations are enclosed in parentheses $(~)$, not square brackets $[~]$.
The items in a series of such citations are usually separated by commas.
If an item in the series of parenthetical citations contains punctuation because (for example) it refers to a work with three or more coauthors, then all items should be separated by semicolons.

The following is a list of correct forms of citations:
\begin{itemize}
\item Brown and Edwards (1993),
\item (Brown and Edwards 1993),
\item (Brown and Edwards 1993, Smith 1997), and
\item (Arnold, Brown, and Edwards 1992; Brown and Edwards 1993; Smith 1997; Brown et al. 1997).
\end{itemize}

The following is a list of incorrect forms of citations:
\begin{itemize}
\item Brown and Edwards [1993],
\item (Brown and Edwards, 1993),
\item (Brown and Edwards, 1993; Smith, 1997), and
\item (Arnold Brown and Edwards 1992, Brown and Edwards 1993, Smith 1997)
\end{itemize}

For further details, please refer to {\it The Chicago Manual of Style} \cite{chicago03}. In Section~\ref{sec:bibtex} you can see how correct citations can easily be achieved by using \BibTeX .

\subsection{List of References}
Place the list of references after the appendices. The section heading is {\bf REFERENCES}, and it is not numbered. List only references that are cited in the text.
Arrange the references in alphabetical order (chronologically for a particular author or group of authors); do not number the references.
Give complete references without abbreviations.
To identify multiple references by the same authors and year, append a lower case letter to the year of publication; for example, 1984a and 1984b.

Use hanging indentation to distinguish individual entries. Do not insert additional space between references.

You can enter the references using (a) {\em \BibTeX\ as discussed in Section~\ref{sec:bibtex}}, (b) using the environment {\tt thebibliography} via the {\tt $\backslash$bibitem} and {\tt $\backslash$cite} commands, or (c) the {\tt hangref} environment as shown below.
Please note that neither (b) nor (c) are recommended. These alternatives may mean extra work for you and the editor during the editing process. Option (c) means in addition that the references will not be hyperlinks --- as the proceedings are electronic proceedings this is not recommended at all.

{\em Whatever you do: the list of entries/items need to be included in your submission! And it needs to be included in such a way that the document can be compiled by the editor.}

To use {\tt hangref} you would enter the following lines.\newline


\begin{verbatim}
\begin{hangref}
\item The first reference goes here, and if you happen to have enough
information on the line you will be able to
 see how the second and if you really have lots of text to be displayed
later lines of the reference are indented.
\item The second reference goes here,
and once again later lines are indented if you have a sufficient amount
of words in the text block.
\item Further references appear here.
\end{hangref}
\end{verbatim}\vspace{4mm}

The output looks as follows.
\begin{hangref}
\item The first reference goes here, and if you happen to have enough information
on the line you will be able to
 see how the second and if you really have lots of text to be displayed later
lines of the reference are indented.
\item The second reference goes here,
and once again later lines are indented if you have a sufficient amount of words in the text block.
\item Further references appear here.
\end{hangref}

The bibliographic style for a journal article is: \\
$<$Surname of first author$>$, $<$First author initials$>$,
$<$Initials and surnames of other authors$>$. $<$year$>$.
$<$Capitalized article title in quotes$>$. $<${\em Journal Name in
Headline Italics}$>$ $<$Volume number$>$: $<$page numbers$>$.

The format for other types of reference can be inferred from the examples in the references, which include:
\begin{itemize}
\item a technical report \cite{chi89},
\item a proceedings article \cite{cheng:input94},
\item a journal article \cite{gupta:mnormal},
\item a book by 2 authors \cite{hammersley:montecarlo},
\item a chapter in a book \cite{sch79},
\item an unpublished thesis or dissertation \cite{ste99},
\item a book with no identified authors \cite{chicago03}, and
\item a document available on the web \cite{Foundation}.
\end{itemize}

Again, please refer to {\it The Chicago Manual of Style} \cite{chicago03} for further details and examples.
Please note that the examples given in the reference section of this document are based on the 16th edition of {\it The Chicago Manual of Style}.
Authors may use the style based on the 15th edition of the manual that they have been using in the papers for the past Winter Simulation Conferences at their discretion.
However, the two styles should not be mixed.  Clarity and consistency should be your primary concern.

Be sure that references to past WSC proceedings, such as \citeN{cheng:input94} include the necessary information such as {\it Proceedings of the xxxx Winter Simulation Conference}, following by the list of editors, then the page number range for the paper and finally the publisher information, Piscataway, New Jersey: Institute of Electrical and Electronics Engineers, Inc.

Template for a bib entry of a (yyyy) Winter Simulation Conference proceedings paper:\newline


\begin{verbatim}
@Inproceedings{(!!Provide a unique key here!!),
	author = {aaaa},
	title  ={tttt},
	year = {yyyy},
	pages = {n-m},
	booktitle = {Proceedings of the yyyy Winter Simulation Conference},
	editors = {eeee},
	address = {Piscataway, New Jersey},
	publisher = {Institute of Electrical and Electronics Engineers, Inc.}
}
\end{verbatim}\vspace{5mm}

Please do not add any additional attributes.

\section{USING \BibTeX}
\label{sec:bibtex}
Using \BibTeX\ for referencing is the recommended way. Indeed, the references in this document were generated using \BibTeX, so the source for this
document serves as an example of how to use \BibTeX\ to meet the WSC formatting requirements.
One benefit of using \BibTeX\ is that bibliography formatting and referencing can be greatly simplified: the correct citation and reference list style is automatically created.
We assume that you already know how to use \BibTeX.
Software to manage \BibTeX\ files, for example JabRef (Java based), can support you on managing and creating valid {\tt bib} files.
{\em Please open your bib file with a software like JabRef BEFORE you submit your final version. Experience shows that almost all manually edited bib files contain duplicated bib keys (which means a random selection of references), broken entries which usually lead to missing bibliographic information, invalid keys, and last but not least invalid tokens in bib files. Bib files DO NOT support comments. \BibTeX\ should not report any error for your final submitted document.}

The \BibTeX\ input file {\tt wsc.bst} and the \LaTeX\ macros found in {\tt wscbib.tex} are required, but are included in {\tt wsc15papersty.tex}, so no other files (apart from your bibliography) are required.
The macros in these files have been tested with \LaTeX. They are not intended for use with \LaTeX\ 2.09, which is obsolete.
The file {\tt wsc.bst} is essentially the same as {\tt chicago.bst}, a file found on many \LaTeX\ distributions, but is
modified to be more compatible with WSC format requirements.

The simplest way to write a WSC article that uses \BibTeX\ is to take the source file for this document, and modify it to generate your article. The file {\tt wsc15paper.tex} requires the file {\tt wsc15papersty.tex}, which contains, among other things, {\tt wsc.bst} and {\tt wscbib.tex} that are needed for \BibTeX.

\subsection{Set Up the \BibTeX\ Input Files}

\BibTeX\ requires a bibliography style file (extension \texttt{.bst}) and a bibliography database file (extension \texttt{.bib}).  This is achieved
using\newline


\begin{verbatim}
\bibliographystyle{wsc}
\bibliography{demobib}
\end{verbatim}\vspace{5mm}

\noindent just before the AUTHOR BIOGRAPHY section.  The file {\tt demobib}\ in the {\tt $\backslash$bibliography} command should be replaced with the base names of your \BibTeX\ {\tt *.bib} files that you use for your bibliography.  \BibTeX\ is then run as usual to create a bibliography file ({\tt *.bbl}).

\subsection{Use the Citation Macros}
There are a number of macros available to cite references in the \LaTeX\ source document.  The {\tt $\backslash$cite} macro can be used to give a list of references in parentheses.  For example,\newline

\begin{verbatim}
\cite{law:simulationc,cheng:queuehetero}
\end{verbatim}\vspace{5mm}

\noindent results in the citation \cite{law:simulationc,cheng:queuehetero}. A reference that functions as a noun is created with the {\tt $\backslash$citeN}
macro.  For example,\newline


\begin{verbatim}
\citeN{law:simulationc} say \ldots
\end{verbatim}\vspace{5mm}

\noindent results in: \citeN{law:simulationc} say \ldots\,.

Citations within parentheses do not need the extra parentheses provided by the above citation commands.  To suppress the inclusion of extra parentheses, use the {\tt $\backslash$citeNP} macro.  To obtain (\citeNP{cheng:queuehetero}, \citeNP{law:simulationc}), for example, use:\newline


\begin{verbatim}
(\citeNP{cheng:queuehetero},
\citeNP{law:simulationc}).
\end{verbatim}\vspace{5mm}

When there are four or more authors, the name of the first author should be given along with the text ``et al.''  This can be achieved with the {\tt $\backslash$shortcite} macro. To obtain \shortcite{bcnn:simulation}, for example, use: \newline


\begin{verbatim}
\shortcite{bcnn:simulation}
\end{verbatim}\vspace{5mm}

\noindent The macros {\tt $\backslash$shortciteN} and {\tt $\backslash$shortciteNP} are also available to obtain `et al.' when a citation with many authors is used as a noun.

For further information on the available commands for citing, search for {\tt $\backslash$cite} in the file {\tt wscbib.tex}, or consult the file {\tt chicago.sty}. The commands for making \BibTeX\ work with {\tt wsc.bst} are very similar to those used in the standard \LaTeX\ file {\tt chicago.sty}.

\subsection{Generate the Bibliography File}

Run {\tt PDFLaTeX} (or \LaTeX), then \BibTeX, and then {\tt PDFLaTeX} two more times. Running {\tt PDFLaTeX} the first time creates the \texttt{.aux} file. Running \BibTeX\ creates the {\tt .bbl} file.  Running {\tt PDFLaTeX} again (twice) fixes the bibliography and citation references.

\subsection{Include the Bibliography File in Your Submission}
\label{sec:submitbib}

Be sure to include your {\tt .bib} file(s) or your {\tt .bbl} file as part of your submission. If you only include the {\tt .bbl} file, then please verify that you include the most up-to-date version reflecting changes during the editing process by rerunning \BibTeX\ one last time before submission.
Please be aware that submitting the {\tt .bbl} file instead of the {\tt .bib} file means extra work for the editing team and for you, as any changes to the reference list need to be done by you in this case.
You can use, e.g., JabRef to create a minimal {\tt .bib} file based on your bibliography and the document (see ``Tools'', ``New subdatabase based on aux file'').
Please open and save the file before every submission with a software like JabRef --- to see whether the file is correct and to check for duplicated entries and/or bib keys in the file.

\section{AUTHOR CHECKLIST}
We strive for a consistent appearance in all papers published in the proceedings. If you used the template and styles within this author's kit, then almost all of the requirements in this checklist will be automatically satisfied, and there is very little to check.

Please {\bf print a hard copy of your paper}, and go over your printed paper to make sure it adheres to the following requirements. {\em Thank you!}
\begin{enumerate}
	\item Abstract
	\begin{enumerate}
		\item 150 or fewer words.
		\item No list of keywords.
	\end{enumerate}
	\item Paper Length
	\begin{enumerate}
		\item At least 5, but no more than 12 pages (15 pages for papers in the introductory and advanced tutorial tracks, and for panels).
	\item Page size is letter size ($8.5" \times 11"$, or $216 mm \times 279 mm$).
	\end{enumerate}
	\item All text is in 11-Point Times New Roman.
	\item The paper has been spellchecked using US English.
	\item Spacing and Margins
	\begin{enumerate}
		\item Single spaced.
		\item Left and right margins are each 1 inch.
		\item Top and bottom margins are each 1 inch except first page.
		\item First page has 1.5 inch margin from the title to the top of the page, and a 1 inch bottom margin.
	\end{enumerate}
	\item Section Headings
	\begin{enumerate}
		\item Left justified and set in {\bf BOLDFACE ALL CAPS}.
		\item Numbered, except for the abstract, acknowledgments, references and author biographies.
		\item Subsection headings are not set in all capitals.
	\end{enumerate}
	\item No footnotes or page numbers.
	\item The running head on the first page is as given in the template file, and the running head on subsequent pages is the surnames of the authors.
	\item The title is in {\bf 11 POINT BOLDFACE ALL CAPS}
	\item Multiple authors are formatted correctly, with email addresses and other information in the Author Biography section.
	\item Equations are centered and any equation numbers are in parentheses and right-justified.
	\item Figures and Tables
	\begin{enumerate}
		\item All text in figures and tables is readable.
		\item Table captions appear above the table. Figure captions appear below the figure.
		\item Fonts are embedded in all non bitmap figures.
	\end{enumerate}
	\item Citations and References (using \BibTeX\ is recommended)
	\begin{enumerate}
		\item Citations are by author and year, and are enclosed in parentheses, not brackets.
		\item References are in the {\tt hangref} style, and are listed alphabetically by the last name(s) of the author(s).
	\end{enumerate}
	\item Author biographies are one paragraph per author.
	\item Hyperlinks
	\begin{enumerate}
		\item Hyperlinks will work as of the date of December 2015.
		\item Live hyperlinks are blue.
		\item URLs are given in the references section and are properly cited.
	\end{enumerate}
	\item Include all files necessary to generate your paper, including
	\begin{enumerate}
		\item Figures (either all in {\tt .ps} or {\tt .eps} format, or all in {\tt .jpg}, {\tt .png}, or {\tt .pdf} format --- see Section~\ref{sec:graphics}),
		\item Bibliography files, if used {\tt .bib} files. Please avoid providing {\tt .bbl} files --- see Section~\ref{sec:submitbib}),
		\item check whether all files included are correctly referenced by trying to compile the archived submission on your own, and
		\item any non-standard packages you use.
	\end{enumerate}
\end{enumerate}

After verifying that your paper meets these requirements, please go to the final submission page linked on the \href{http://www.wintersim.org}{conference website} \cite{WSC} and submit your paper.
Be sure to complete the transfer of copyright and email a copy of your {\tt .pdf} receipt to \href{mailto://wsc15proceedings@gmail.com}{the proceedings editors} in the process.
Thank you for contributing to the WSC!

\section*{ACKNOWLEDGMENTS}
Place the acknowledgments section, if needed, after the main text, but before any appendices and the references. The section heading is not numbered.
These instructions are adapted from instructions that have been updated and improved by proceedings editors and several other individuals, who are too numerous to name separately (our apologies, but it is necessary), since the first set of instructions were written by Barry Nelson for the 1991 WSC.

\appendix

\section{APPENDICES} \label{app:quadratic}
Place any appendices after the acknowledgments and label them
\textbf{A}, \textbf{B}, \textbf{C}, and so forth.

The solution to (1) has the form
\begin{equation} \label{eq: quadratic sol}
x = \frac{-b \pm \sqrt{b^2-4ac}}{2a} \mbox{ if } a \ne 0.
\end{equation}

\section{GETTING HELP}
If you need help in preparing your paper, contact the proceedings editors. You can reach the entire team by writing to our unified point of contact at \href{mailto://wsc15proceedings@gmail.com}{wsc15proceedings@gmail.com}.

You may also reach us individually using the contact information below.

\vspace{6pt}

\noindent Levent Yilmaz\\
Auburn University\\
Email: \href{mailto://wsc15yilmaz@gmail.com}{wsc15yilmaz@gmail.com}\\
\\
Wai Kin (Victor) Chan\\
Rensselaer Polytechnic Institute\\
Email: \href{mailto://wsc15chan@gmail.com}{wsc15chan@gmail.com}\\
\\
II-Chul Moon\\
KAIST\\
Email: \href{mailto://wsc15moon@gmail.com}{wsc15moon@gmail.com}\\
\\
Theresa M. K. Roeder\\
San Francisco State University\\
Email: \href{mailto://wsc15roeder@gmail.com}{wsc15roeder@gmail.com}\\

\section{OFTEN OBSERVED MISTAKES}

The following list comprises the most common sources of error that had to be corrected by previous editors. Please make sure to go through the following list and check that your paper is formatted correctly:

\begin{enumerate}
\item   The paper can be \textit{at most} 12 pages long (15 for tutorials and panel sessions). Longer papers cannot be published.
\item	Paper title and section titles are in ALL CAPS, subsections capitalize the first letter of important words. Please use the templates to use correct indents and spaces.
\item	Paper is letter format, not DinA 4 format. Please use the required margins (different on page 1 from the following pages).
\item	Use the correct running heads! Use the proceedings editors and chairs on page one, and use the last names separated by commas for the other pages. Don't forget that the last Last Name is preceded by ``, and ''
\item	Double check the citation format!
\item	Don't forget the ``author biographies'' section!
\item	Double-check that figures and tables are referenced in the text and have the correct caption format!
\item	Double check that the author section after the title is formatted correctly: the number of organizations defines the number of blocks, the number of blocks defines the layout.
\item	In the heading on the title page, country names should be in ALL CAPS.
\item	The first line of each paragraph is indented, with the exception of the first paragraph of a section or subsection.
\item	There should be extra lines before and after enumerations, lists, etc.
\end{enumerate}

% Please don't exchange the bibliographystyle style
\bibliographystyle{wsc}
% AUTHOR: Include your bib file here
\bibliography{demobib}

\section*{AUTHOR BIOGRAPHIES}

\noindent {\bf LEVENT YILMAZ} is Professor of Computer Science and Software Engineering at Auburn University with a joint appointment in Industrial and Systems Engineering. He holds M.S. and Ph.D. degrees in Computer Science from Virginia Tech. He is the founding organizer and General Chair of the Agent-Directed Simulation Conference series and is currently serving as the Editor-in-Chief of \textit{Simulation: Transactions of the SCS}. His email address is \email{wsc15yilmaz@gmail.com}. \\

\noindent {\bf WAI KIN (VICTOR) CHAN} is an Associate Professor of the Department of Industrial and Systems Engineering at the Rensselaer Polytechnic Institute, Troy, NY.  He holds a Ph.D. in industrial engineering and operations research from University of California, Berkeley. His research interests include discrete-event simulation, agent-based simulation, and their applications in social networks, service systems, transportation networks, energy markets, and manufacturing. He is a member of INFORMS, IIE, and IEEE. His e-mail address is \email{wsc15chan@gmail.com}.\\

\noindent {\bf II-CHUL MOON} is an assistant professor at the Department of Industrial and Systems Engineering, KAIST, Daejeon, Republic of Korea. His theoretic research focuses on the overlapping area of computer
science, management, sociology, and operations research. His practical research includes military command and control analysis, counterterrorism analysis, intelligence analysis, and disaster management; His email address is \email{wsc15moon@gmail.com}.\\

\noindent {\bf THERESA M. K. ROEDER} is an Associate Professor of Decision Sciences at San Francisco State University. She holds a PhD in Industrial Engineering and Operations Research from UC Berkeley. Her research interests lie in O.R. education and simulation modeling, especially in healthcare and higher education. Her email address is \email{wsc15roeder@gmail.com}.\\



\newpage

\begin{figure*}[htb]
{
\centering
First Name Last Name 1 \\
First Name Last Name 2 \\
\vspace{12pt}
Institution \\
Street Address 1 \\
Street Address 2 \\
City, ST Zip, COUNTRY
\caption{Example title page heading with 2 authors from the same institution.\label{fig: 2 same}}
}
\end{figure*}

\begin{figure*}[htb]
{
\centering
\begin{tabular}{ccc}
\phantom{Entries to adjust spacing - ignore} & \phantom{intermediate space} & \phantom{Entries to adjust spacing - ignore} \\
First Name Last Name 1 & & First Name Last Name 2 \\
\\
Institution 1 & & Institution 2 \\
Street Address 1 & & Street Address 1 \\
Street Address 2 & & Street Address 2 \\
City, ST Zip, COUNTRY & & City, ST Zip, COUNTRY
\end{tabular}
\caption{Example title page heading with 2 authors from different institutions.\label{fig: 2 different}}
}
\end{figure*}



\begin{figure*}[htb]
{
\centering
\begin{tabular}{ccc}
\phantom{This adjusts spacing - ignore} & \phantom{This adjusts spacing - ignore} & \phantom{This adjusts spacing - ignore} \\
First Name Last Name 1 & & First Name Last Name 2 \\
\\
Institution 1 & & Institution 2 \\
Street Address Line 1 & & Street Address Line 1 \\
Street Address Line 2 & & Street Address Line 2 \\
City, ST Zip, COUNTRY & & City, ST Zip, COUNTRY \\
\\
\\
& First Name Last Name 3 \\
\\
& Institution 3\\
& Street Address 1 \\
& Street Address 2 \\
& City, ST Zip, COUNTRY
\end{tabular}
\caption{Alternate example title page heading with 3 authors from different institutions. \label{fig: 3 different}}
}
\end{figure*}

\begin{figure*}[htb]
{
\centering
\begin{tabular}{ccc}
\phantom{Adjust spacing using these entries} & \phantom{intermediate space} & \phantom{Adjust spacing using these entries} \\
First Name Last Name 1 & & First Name Last Name 2 \\
\\
Institution 1 & & Institution 2 \\
Street Address Line 1 & & Street Address Line 1 \\
Street Address Line 2 & & Street Address Line 2 \\
City, ST Zip, COUNTRY & & City, ST Zip, COUNTRY \\
\\ \\
First Name Last Name 3 & & First Name Last Name 4 \\
\\
Institution 3 & & Institution 4 \\
Street Address Line 1 & & Street Address Line 1 \\
Street Address Line 2 & & Street Address Line 2 \\
City, ST Zip, COUNTRY & & City, ST Zip, COUNTRY
\end{tabular}
\caption{Example title page heading with 4 authors from different institutions.\label{fig: 4 different}}
}
\end{figure*}



\end{document}

